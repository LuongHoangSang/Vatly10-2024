\let\lesson\undefined
\newcommand{\lesson}{\phantomlesson{Ôn tập chương 7}}
\setcounter{section}{2}
\ANSMCQ
{	\begin{center}
		\begin{tabular}{|m{2.8em}|m{2.8em}|m{2.8em}|m{2.8em}|m{2.8em}|m{2.8em}|m{2.8em}|m{2.8em}|m{2.8em}|m{2.8em}|}
			\hline
			1.B  & 2.C  & 3.C  & 4.A  & 5.B  & 6.C  & 7.D  & 8.D  & 9.B  & 10.C  \\
			\hline
			11.B  & 12. A  & 13.C  & 14.A  & 15.A  & 16.A  & 17.B  & 18.B  & 19.C  & 20.A  \\
			\hline
			21.A  & 22.A  & 23.B  & 24.B  & 25.C  &26.D   &27.A   & 28.A  &29.A   &30.D   \\
			\hline
		\end{tabular}
	\end{center}
}
\begin{enumerate}[label=\bfseries Câu \arabic*:, leftmargin=1.5cm]
	\item \mkstar{1}
	
	
	{
		Đơn vị của động lượng bằng
		\begin{mcq}(4)
			\item N/s.
			\item N$\cdot$s. 
			\item N$\cdot$m.
			\item N$\cdot$m/s.
		\end{mcq}
	}
	
	\hideall
	{	
		\textbf{Đáp án: B.}
	}
	\item \mkstar{1}
	
	
	{Điều nào sau đây \textbf{sai} khi nói về động lượng?
		\begin{mcq}
			\item Động lượng của một vật có độ lớn bằng tích khối lượng và tốc độ của vật.
			\item Trong hệ kín, động lượng của hệ được bảo toàn.
			\item Động lượng của một vật có độ lớn bằng tích khối lượng và bình phương vận tốc.
			\item Động lượng của một vật là một đại lượng véc tơ.
		\end{mcq}
	}
	
	\hideall
	{	
		\textbf{Đáp án: C}
	}
	\item \mkstar{1}
	
	
	{Chọn câu phát biểu đúng nhất?
		\begin{mcq}
			\item Véctơ động lượng của hệ được bảo toàn. 
			\item Véctơ động lượng toàn phần của hệ được bảo toàn. 
			\item Véctơ động lượng toàn phần của hệ kín được bảo toàn. 
			\item Động lượng của hệ kín được bảo toàn.
		\end{mcq}
	}
	
	\hideall
	{	
		\textbf{Đáp án: C.}
	}
	\item \mkstar{1}
	
	
	{Động lượng của vật bảo toàn trong trường hợp nào sau đây? 
		\begin{mcq}
			\item Vật đang chuyển động thẳng đều trên mặt phẳng nằm ngang.
			\item  Vật đang chuyển động tròn đều. 
			\item Vật đang chuyển động nhanh dần đều trên mặt phẳng nằm ngang không ma sát. 
			\item Vật đang chuyển động chậm dần đều trên mặt phẳng nằm ngang không ma sát.
		\end{mcq}
	}
	
	\hideall
	{	
		\textbf{Đáp án: A.}
	}
	\item \mkstar{1}
	
	
	{Vector động lượng là vector 
		\begin{mcq}(2)
			\item có phương vuông góc với vector vận tốc.
			\item cùng phương, cùng chiều với vector vận tốc.
			\item cùng phương, ngược chiều với vector vận tốc.
			\item có phương hợp với vector vận tốc một góc bất kì.
		\end{mcq}
	}
	
	\hideall
	{	
		\textbf{Đáp án: B.}
	}

\item \mkstar{2}\\
{Biểu thức nào sau đây mô tả đúng mối quan hệ giữa động lượng và động năng?
	\begin{mcq}(4)
		\item $p=\sqrt{mW_\text{đ}}$.
		\item $p=mW_\text{đ}$.
		\item $p=\sqrt{2mW_\text{đ}}$.
		\item $p=2mW_\text{đ}$.
	\end{mcq}

}
\hideall{
\textbf{Đáp án C.}\\
Ta có:
\begin{align*}
	\begin{cases}
		W_\text{đ}=\dfrac{1}{2}mv^2\\
		p=mv
	\end{cases}
\Rightarrow p^2=2mW_\text{đ}\Leftrightarrow p=\sqrt{2mW_\text{đ}}
\end{align*}
}

\item \mkstar{2}\\
Trong trường hợp nào sau đây, hệ có thể xem là hệ kín?
\begin{mcq}
	\item Hai viên bi chuyển động trên mặt phẳng nằm ngang.
	\item Hai viên bi chuyển động trên mặt phẳng nghiêng.
	\item Hai viên bi rơi thẳng đứng trong không khí.
	\item Hai viên bi chuyển động không ma sát trên mặt phẳng nằm ngang.
\end{mcq}
\hideall{
\textbf{Đáp án D.}\\
}

\item \mkstar{2}\\
{Trong các quá trình chuyển động sau đây, quá trình nào mà động lượng của vât không thay đổi?
	\begin{mcq}
		\item Vật chuyển động chạm vào vách và phản xạ lại.
		\item Vật được ném ngang.
		\item Vật đang rơi tự do.
		\item Vật chuyển động thẳng đều.
	\end{mcq}

}
\hideall{
\textbf{Đáp án D.}
}

\item \mkstar{2}\\
{Va chạm đàn hồi và va chạm mềm khác nhau ở điểm nào sau đây?
	\begin{mcq}
		\item Hệ va chạm đàn hồi có động lượng bảo toàn còn va chạm mềm thì động lượng không bảo toàn.
		\item Hệ va chạm đàn hồi có động năng không thay đổi còn va chạm mềm thì động năng thay đổi.
		\item Hệ va chạm mềm có động năng không thay đổi còn va chạm đàn hồi thì động năng thay đổi.
		\item Hệ va chạm mềm có động lượng bảo toàn còn va chạm đàn hồi thì động lượng không bảo toàn.
	\end{mcq}

}
\hideall{
\textbf{Đáp án B.}
}

\item\mkstar{2}\\
{Cho hai vật va chạm trực diện với nhau, sau va chạm, hai vật dính liền thành một khối và chuyển động với cùng vận tốc. Động năng của hệ ngay trước và sau va chạm lần lượt là $W_\text{đ}$ và $W'_\text{đ}$. Biểu thức nào dưới đây là đúng?
	\begin{mcq}(4)
		\item $W_\text{đ}=W_\text{đ'}$.
		\item $W_\text{đ}<W'_\text{đ}$.
		\item $W_\text{đ}>W'_\text{đ}$.
		\item $W_\text{đ}=2W'_\text{đ}$.
	\end{mcq}
}
\hideall{
\textbf{Đáp án C.}\\
Trong hệ va chạm mềm, cơ năng của hệ sau va chạm bé hơn cơ năng của hệ trước va chạm.
}



\item Hai vật có khối lượng $m_1$ và $m_2$ chuyển động với vận tốc lần lượt là $\vec v_1$ và $\vec v_2$. Động lượng của hệ có giá trị 
\begin{mcq}(4)
	\item $m\vec v$.
	\item $m_1\vec v_1+m_2\vec v_2$.
	\item 0.
	\item $m_1v_1+m_2v_2$.
\end{mcq}
\hideall{
\textbf{Đáp án B.}
}

\item \mkstar{2}\\
{Vật 1 khối lượng $m$ đang chuyển động với tốc độ $v_0$ đến va chạm đàn hồi với vật 2 có cùng khối lượng và đang đứng yên. Nếu khối lượng vật 2 tăng lên gấp đôi thì động năng của hệ sau va chạm
	\begin{mcq}(2)
		\item không đổi.
		\item tăng 2 lần.
		\item giảm 1,5 lần.
		\item tăng 1,5 lần.
	\end{mcq}

}
\hideall{
\textbf{Đáp án A.}\\
Vì va chạm là đàn hồi nên động năng của hệ sau va chạm bằng động năng của hệ trước va chạm và bằng động năng của vật 1 trước va chạm $W_\text{đ}=\dfrac{1}{2}m_1v^2_0$ (ban đầu vật 2 đứng yên).
}

\item \mkstar{2}\\
{Một vật chuyển động với tốc độ tăng dần thì có
	\begin{mcq}(2)
		\item động lượng không đổi.
		\item động lượng bằng không.
		\item động lượng tăng dần.
		\item động lượng giảm dần.
	\end{mcq}

}
\hideall{
\textbf{Đáp án C.}
}

	\item \mkstar{2}
	
	
	{	Quả cầu A khối lượng $m_1$ chuyển động với vận tốc $\vec v_1$ va chạm vào quả cầu B khối lượng $m_2$ đứng yên. Sau va chạm, cả hai quả cầu có cùng vận tốc $\vec v_2$. Ta có hệ thức
		\begin{mcq}(2)
			\item $m_1\vec v_1 = (m_1 + m_2) \vec v_2.$
			\item  $m_1\vec v_1 = -m_2 \vec v_2.$
			\item $m_1\vec v_1 = m_2 \vec v_2.$
			\item $m_1\vec v_1 = \dfrac{1}{2}(m_1 + m_2) \vec v_2.$
		\end{mcq}
	}
	
	\hideall
	{	
		\textbf{Đáp án: A.}
	}
	\item \mkstar{2}
	
	
	{Một vật có khối lượng $\SI{4}{kg}$ rơi tự do không vận tốc đầu trong khoảng thời gian $\SI{2,5}{s}$. Lấy $g = \SI{10}{m/s}^2$. Độ biến thiên động lượng của vật trong khoảng thời gian đó có độ lớn là
		\begin{mcq}(4)
			\item $\SI{100}{kg\cdot m/s}$.
			\item $\SI{25}{kg\cdot m/s}$.
			\item $\SI{50}{kg\cdot m/s}$.
			\item $\SI{200}{kg\cdot m/s}$.
		\end{mcq}
	}
	
	\hideall
	{	
		\textbf{Đáp án: A.}
		
		Vận tốc ban đầu của vật: 
		
		$$v_0 = \SI{0}{m/s}.$$
		
		Vận tốc của ngay trước khi chạm đất
		
		$$v =gt = \SI{25}{m/s}.$$
		
		Độ biến thiên động lượng của vật trong khoảng thời gian:
		
		$$\Delta p = p_2 - p_1 = mv -mv_0 = \SI{100}{kg \cdot m/s}.$$
		
	}
	\item \mkstar{2}
	
	
	{Người ta ném một quả bóng khối lượng $\SI{500}{g}$ cho nó chuyển động với tốc độ $\SI{20}{m/s}$. Xung lượng của lực tác dụng lên quả bóng là
		\begin{mcq}(4)
			\item $\SI{10}{N\cdot s}.$
			\item $\SI{200}{N\cdot s}.$
			\item $\SI{100}{N\cdot s}.$
			\item $\SI{20}{N\cdot s}.$
		\end{mcq}
	}
	
	\hideall
	{	
		\textbf{Đáp án: A.}
		
		Xung lượng của lực tác dụng là
		
		$$F \cdot \Delta t= \Delta p = mv = \SI{10}{N\cdot s}.$$
	}
	\item \mkstar{2}
	
	
	{Hai vật có khối lượng $m_1 = 2m_2$, chuyển động với vận tốc có độ lớn $v_1 = 2v_2$. Động lượng của hai vật có quan hệ
		\begin{mcq}(4)
			\item $p_1 = 2p_2.$
			\item $p_1 = 4p_2.$
			\item $p_2 = 4p_1.$
			\item $p_1 = p_2.$
		\end{mcq}
	}
	
	\hideall
	{	
		\textbf{Đáp án: B.}
		
		Ta có:
		
		$$\dfrac{p_1}{p_2} = \dfrac{m_1v_1}{m_2v_2} = \dfrac{2 \cdot 2 m_2v_2}{m_2v_2} = 4.$$
		
		Suy ra: $p_1 = 4p_2.$
	}
	\item \mkstar{2}
	
	
	{Một chất điểm chuyển động không vận tốc đầu dưới tác dụng của lực $F = \xsi{10^{-2}}{N}$. Động lượng chất điểm ở thời điểm $t = \SI{3}{s}$ kể từ lúc bắt đầu chuyển động là 
		\begin{mcq}(4)
			\item $\xsi{2 \cdot 10^{-2}}{kg \cdot m/s}.$
			\item $\xsi{3 \cdot 10^{-2}}{kg \cdot m/s}.$
			\item $\xsi{ 10^{-2}}{kg \cdot m/s}.$
			\item $\xsi{6 \cdot 10^{-2}}{kg \cdot m/s}.$
		\end{mcq}
	}
	
	\hideall
	{	
		\textbf{Đáp án: B.}
		
		Ta có:
		
		$$p = F\Delta t = \xsi{3\cdot 10^{-2}}{kg\cdot m/s}.$$
	}

\item \mkstar{2}


{Một vật nhỏ khối lượng $m =\SI{2}{kg}$ trượt xuống một đường dốc thẳng nhẵn tại một thời điểm xác định có vận tốc $\SI{3}{m/s}$, sau đó $\SI{4}{s}$ có vận tốc $\SI{7}{m/s}$, tiếp ngay sau đó $\SI{3}{s}$ vật có động lượng là 
	\begin{mcq}(4)
		\item $\SI{6}{kg\cdot m/s}.$ 
		\item $\SI{10}{kg\cdot m/s}.$ 
		\item $\SI{20}{kg\cdot m/s}.$  
		\item $\SI{28}{kg\cdot m/s}.$ 
	\end{mcq}
}

\hideall
{	
	\textbf{Đáp án: C.}
	
	Ta có: $$a = \dfrac{v - v_0}{4} = \SI{1}{m/s}^2.$$
	
	Vận tốc sau $\SI{3}{s}$ là:
	
	$$v' = v + at = \SI{10}{m/s}.$$
	
	Động lượng của vật
	
	$$ p =mv' = \SI{20}{kg \cdot m/s}.$$
}

\item \mkstar{2}\\
{Vật $m_1=\SI{1}{\kilogram}$ chuyển động với tốc độ $v_1$ đến va chạm mềm vào vật $m_2 =\SI{2}{\kilogram}$ đang nằm yên. Ngay sau va chạm tốc độ của vật $m_2$ là $v_2 =\SI{2}{\meter/\second}$. Tính tốc độ vật $m_1$ trước va chạm? 
\begin{mcq}(4)
	\item $v_1=\SI{6}{\meter/\second}$.
	\item $v_1=\SI{1.2}{\meter/\second}$.
	\item $v_1=\SI{5}{\meter/\second}$.
	\item $v_1=\SI{6}{\meter/\second}$.
\end{mcq}
}
\hideall{
\textbf{Đáp án A.}\\
Áp dụng định luật bảo toàn động lượng cho hai vật $m_1$, $m_2$ ngay trước và sau va chạm:
$$m_1\vec v_1=\left(m_1+m_2\right)\vec v_2$$
$$\Rightarrow\vec v_1=\dfrac{\left(m_1+m_2\right)\vec v_2}{m_1}$$
$$\Rightarrow v_1=\dfrac{\left(m_1+m_2\right)v_2}{m_1}=\SI{6}{\meter/\second}.$$
}

\item \mkstar{2}\\
{Hai vật có khối lượng $m_1 =\SI{2}{\kilogram}$ và $m_2 =\SI{5}{\kilogram}$ chuyển động với tốc độ $v_1 =\SI{5}{\meter/\second}$ và $v_2 =\SI{2}{\meter/\second}$. Tổng động lượng của hệ trong các trường hợp $\vec v_1$, và $\vec v_2$ cùng phương, ngược chiều là
\begin{mcq}(4)
	\item $\SI{0}{\kilogram\cdot\meter\second^{-1}}$.
	\item $\SI{3}{\kilogram\cdot\meter\second^{-1}}$.
	\item $\SI{6}{\kilogram\cdot\meter\second^{-1}}$.
	\item $\SI{10}{\kilogram\cdot\meter\second^{-1}}$.
\end{mcq}
}
\hideall{
\textbf{Đáp án A.}\\
Vì $\vec p_1\uparrow\downarrow\vec p_1$ nên tổng động lượng của hệ:
$$p=\left|p_2-p_1\right|=\SI{0}{\kilogram\cdot\meter\second^{-1}}.$$
}

\item \mkstar{3}\\
{Hai xe có khối lượng $m_1$ và $m_2$ chuyển động ngược chiều nhau với tốc độ $v_1 =\SI{10}{\meter/\second}$; $v_2 =\SI{4}{\meter/\second}$. Sau va chạm 2 xe bị bật trở lại với cùng tốc độ $v'_1=v'_2=\SI{5}{\meter/\second}$. Tỉ số khối lượng của 2 xe là
\begin{mcq}(4)
	\item 0,6.
	\item 0,2.
	\item $\dfrac{5}{3}$.
	\item 5.
\end{mcq}
}
\hideall{
\textbf{Đáp án A.}\\
Áp dụng định luật bảo toàn động lượng cho hai xe ngay trước và sau va chạm:
$$m_1\vec v_1+m_2\vec v_2=m_1\overrightarrow{v'}_1+m_2\overrightarrow{v'}_2$$
Chiếu phương trình bảo toàn động lượng lên chiều chuyển động ban đầu của $m_1$:
$$m_1v_1-m_2v_2=-m_1v'_1+m_2v'_2$$
$$\Rightarrow \dfrac{m_1}{m_2}=\dfrac{v_2+v'_2}{v_1+v'_1}=0,6.$$
}

\item \mkstar{3}\\
{Cho một vật khối lượng $m_1$ đang chuyển động với với vận tốc $\SI{5}{\meter/\second}$ đến va chạm với vật hai có khối lượng $\SI{1}{\kilogram}$ đang chuyển động với vận tốc $\SI{1}{\meter/\second}$, hai vật chuyển động cùng chiều. Sau va chạm 2 vật dính vào nhau và cùng chuyển động với vận tốc $\SI{2.5}{\meter/\second}$. Xác định khối lượng $m_1$.
	\begin{mcq}(4)
		\item $\SI{1}{\kilogram}$.
		\item $\SI{0.6}{\kilogram}$.
		\item $\SI{2}{\kilogram}$.
		\item $\SI{3}{\kilogram}$.
	\end{mcq}

}
\hideall{
\textbf{Đáp án B.}\\
Áp dụng định luật bảo toàn động lượng cho hệ hai viên bi ngay trước và sau va chạm:
$$m_1\vec v_1+m_2\vec v_2=\left(m_1+m_2\right)\vec v$$
Chiếu phương trình bảo toàn động lượng lên chiều chuyển động ban đầu của hai hòn bi:
$$m_1v_1+m_2v_2=\left(m_1+m_2\right)v$$
$$\Rightarrow m_1=\dfrac{m_2\left(v-v_2\right)}{v_1-v}=\SI{0.6}{\kilogram}.$$
}



\item \mkstar{3}\\
{Cho viên bi thứ nhất có khối lượng $\SI{200}{\gram}$ đang chuyển động trên mặt phẳng nằm ngang với vận tốc $\SI{5}{\meter/\second}$ tới va chạm vào viên bi thứ hai có khối lượng $\SI{400}{\gram}$ đang đứng yên, biết rằng sau va chạm viên bi thứ hai chuyển động với vận tốc $\SI{3}{\meter/\second}$, chuyển động của hai bi trên cùng một đường thẳng. Xác định độ lớn vận tốc của viên bi thứ nhất sau va chạm.
	\begin{mcq}(4)
		\item $\SI{4}{\meter/\second}$.
		\item $\SI{1}{\meter/\second}$.
		\item $\SI{6}{\meter/\second}$.
		\item $\SI{5}{\meter/\second}$.
	\end{mcq}

}
\hideall{
\textbf{Đáp án B.}\\
Chọn chiều dương là chiều chuyển động của bi thứ nhất trước lúc va chạm.\\
Áp dụng định luật bảo toàn động lượng cho hai viên bi ngay trước và sau va chạm:
$$m_1\vec v_1+m_2\vec v_2=m_1\overrightarrow{v'_1}+m_2\overrightarrow{v'_2}$$
Chiếu phương trình bảo toàn động lượng lên chiều dương:
$$m_1v_1=m_1v'_1+m_2v'_2$$
$$\Rightarrow v'_1=\dfrac{m_1v_1-m_2v'_2}{m_1}=\dfrac{\left(\SI{0.2}{\kilogram}\right)\cdot\left(\SI{5}{\meter/\second}\right)-\left(\SI{0.4}{\kilogram}\right)\cdot\left(\SI{3}{\meter/\second}\right)}{\SI{0.2}{\kilogram}}=\SI{-1}{\meter/\second}.$$
Vậy sau va chạm bi thứ nhất chuyển động với tốc độ $\SI{1}{\meter/\second}$ và chuyển động ngược chiều dương đã chọn.
}

\item \mkstar{3}\\
{Hai vật nhỏ có khối lượng khác nhau ban đầu ở trạng thái nghỉ. Sau đó, hai vật đồng thời chịu tác dụng của ngoại lực không đổi có độ lớn như nhau và bắt đầu chuyển động. Sau cùng một khoảng thời gian, điều nào sau đây là đúng?
	\begin{mcq}
		\item Động năng của hai vật như nhau.
		\item Vật có khối lượng lớn hơn có động năng lớn hơn.
		\item Vật có khối lượng lớn hơn có động năng nhỏ hơn.
		\item Không đủ dữ kiện để so sánh.
	\end{mcq}
	
}
\hideall{
	\textbf{Đáp án C.}\\
	Ta có: $\Delta \vec p=\vec F\cdot\Delta t$.\\
	Ban đầu, vật ở trạng thái nghỉ nên:
	$$\vec p=\vec F\Delta t\Rightarrow p^2=F^2\left(\Delta t\right)^2\Rightarrow2mW_\text{đ}=F^2\left(\Delta t\right)^2 \Rightarrow W_\text{đ}=\dfrac{F^2\left(\Delta t\right)^2}{2m}$$
	Như vậy, vật có khối lượng càng lớn thì động năng càng bé.
	
}

\item \mkstar{3}


{Một xe ô tô có khối lượng $m_1 = 6\ \text{tấn}$ chuyển động thẳng với vận tốc $v_1=3\ \text{m/s}$, đến tông và dính vào một xe gắn máy đang đứng yên có khối lượng $m_2 = 200\ \text{kg}$. Tính vận tốc của các xe sau va chạm.
	\begin{mcq}(4)
		\item $\SI{1.9}{\meter/s}.$
		\item $\SI{3.9}{\meter/s}.$ 
		\item $\SI{4.9}{\meter/s}.$
		\item $\SI{2.9}{\meter/s}.$
	\end{mcq}
}

\hideall
{	
	\textbf{Đáp án: D.}
	
	Xem hệ hai xe là hệ cô lập, áp dụng định luật bảo toàn động lượng của hệ: 
	$$m_1 \cdot \vec{v_1} = (m_1+m_2)\cdot \vec{v}.$$
Vì hai xe va chạm mềm nên sẽ chuyển động theo chiều cũ. Tốc độ của hai xe sau va chạm:
	$$v=\frac{m_1 \cdot v_1}{m_1+m_2}= \SI{2.9}{\meter/s}.$$
}


\item \mkstar{3}\\
{Viên đạn khối lượng $\SI{20}{\gram}$ đang bay với tốc độ $\SI{600}{\meter/\second}$ thì gặp một cánh cửa thép. Đạn xuyên qua cửa trong thời gian $\SI{0.002}{\second}$. Sau khi xuyên qua cánh cửa thì tốc độ của đạn còn $\SI{300}{\meter/\second}$. Lực cản trung bình của cửa tác dụng lên đạn có độ lớn bằng
	\begin{mcq}(4)
		\item $\SI{3000}{\newton}$.
		\item $\SI{900}{\newton}$.
		\item $\SI{9000}{\newton}$.
		\item $\SI{30000}{\newton}$.
	\end{mcq}

}
\hideall{
\textbf{Đáp án A.}\\
Lực cản trung bình của cửa tác dụng lên đạn:
$$F=\dfrac{\left|\Delta p\right|}{\Delta t}=\dfrac{m\left(v_1-v_2\right)}{\Delta t}=\SI{3000}{\newton}.$$
}
	
	\item \mkstar{4}
	
	
	{Một người có khối lượng $m_1=\SI{50}{kg}$ nhảy từ một chiếc xe có khối lượng $m_2 = \SI{80}{kg}$ đang chuyển động theo phương ngang với vận tốc $v = \SI{3}{m/s}$. Biết vận tốc nhảy của người đối với xe lúc chưa thay đổi vận tốc là $v_0 = \SI{4}{m/s}$. Vận tốc của xe sau khi người ấy nhảy ngược chiều đối với xe là
		\begin{mcq}(4)
			\item $\SI{5,5}{m/s}.$
			\item $\SI{4,5}{m/s}.$ 
			\item $\SI{0,5}{m/s}.$
			\item $\SI{1}{m/s}.$
		\end{mcq}
	}
	
	\hideall
	{	
		\textbf{Đáp án: A.}
		
		Gọi:
		\begin{itemize}
			\item (1) người;
			\item (2) xe;
			\item (0) đất.
		\end{itemize}
		Ta có: $v_{20}=\SI{3}{\meter/\second}$, $v'_{12}=\SI{4}{\meter/\second}$.\\
		Áp dụng định luật bảo toàn động lượng cho hệ người và xe ngay trước và sau khi người nhảy:
		\begin{eqnarray*}
			\left(m_1+m_2\right)\overrightarrow{v_{20}}&=&m_1\overrightarrow{v'_{10}} +m_2\overrightarrow{v'_{20}}\\
			\Leftrightarrow \left(m_1+m_2\right)\overrightarrow{v_{20}}&=&m_1\left(\overrightarrow{v'_{12}}+\overrightarrow{v_{20}}\right) +m_2\overrightarrow{v'_{20}}\\
			\Rightarrow m_2\overrightarrow{v_{20}}&=&m_1\overrightarrow{v'_{12}}+m_2\overrightarrow{v'_{20}} \quad (*)
		\end{eqnarray*}		
		Chiếu phương trình (*) lên chiều chuyển động ban đầu của xe:
		$$m_2v_{20}=-m_1v'_{12}+m_2v'_{20}\Rightarrow v'_{20}=\dfrac{m_2v_{20}+m_1v'_{12}}{m2}$$
		$$\Leftrightarrow v'_{20}=\dfrac{\left(\SI{80}{\kilogram}\right)\cdot\left(\SI{3}{\meter/\second}\right)+\left(\SI{50}{\kilogram}\right)\cdot\left(\SI{4}{\meter/\second}\right)}{\SI{80}{\kilogram}}=\SI{5.5}{\meter/\second}$$
		
		
	}
	\item \mkstar{4}
	
	
	{Tên lửa khối lượng $\SI{500}{kg}$ đang chuyển động với vận tốc  $\SI{200}{m/s}$ thì tách ra làm hai phần. Phần bị tháo rời có khối lượng $\SI{200}{kg}$ sau đó chuyển động ra phía sau với vận tốc $\SI{100}{m/s}$ so với phần còn lại. Vận tốc phần còn lại bằng
		\begin{mcq}(4)
			\item $\SI{240}{m/s}.$
			\item $\SI{266,7}{m/s}.$ 
			\item $\SI{220}{m/s}.$
			\item $\SI{400}{m/s}.$
		\end{mcq}
	}
	
	\hideall
	{	
		\textbf{Đáp án: A.}\\
		Gọi:
		\begin{itemize}
			\item (1) phần tên lửa bị tháo rời;
			\item (2) phần tên lửa còn lại;
			\item (0) mặt đất.
		\end{itemize}
		Áp dụng định luật bảo toàn động lượng cho tên lửa ngay trước và sau khi tách ra:
		\begin{equation}
			\label{eq:30-P.6}
			M\vec V=m_1\vec v_{10}+m_2\vec v_{20}=m_1\left(\vec v_{12}+\vec v_{20}\right)+m_2\vec v_{20}
		\end{equation}
		Chiếu (\ref{eq:30-P.6}) lên chiều chuyển động ban đầu của tên lửa:
		$$MV = m_1\left(v_{20}-v_{12}\right)+m_2v_{20}\Rightarrow v_{20} = \SI{240}{m/s}.$$
	}
	
	\item \mkstar{4}\\
	{Một khẩu pháo có khối lượng $m_1 =\SI{130}{\kilogram}$ được đặt trên một toa xe nằm trên đường ray biết toa xe có khối lượng $m_2 =\SI{20}{\kilogram}$ khi chưa nạp đạn. Viên đạn được bắn ra theo phương nằm ngang dọc theo đường ray biết viên đạn có khối lượng $m_3 =\SI{1}{\kilogram}$. Vận tốc của đạn khi bắn ra khỏi nòng súng thì có vận tốc $v_0 = \SI{400}{\meter/\second}$ so với súng. Hãy xác định vận tốc của toa xe sau khi bắn. Biết rằng ban đầu toa xe đang chuyển động với vận tốc $v_1 =\SI{18}{\kilo\meter/\hour}$ theo chiều bắn đạn
		\begin{mcq}(4)
			\item $\SI{3.67}{\meter/\second}$.
			\item $\SI{5.25}{\meter/\second}$.
			\item $\SI{8.76}{\meter/\second}$.
			\item $\SI{2.33}{\meter/\second}$.
		\end{mcq}
	}
\hideall{
\textbf{Đáp án D.}\\
Áp dụng định luật bảo toàn động lượng cho hệ (xe-khẩu pháo-đạn) ngay trước và sau khi bắn:
$$\left(m_1+m_2+m_3\right)\overrightarrow{v}_\text{xe/đất}=\left(m_1+m_2\right)\overrightarrow{v'}_\text{xe/đất}+m_3\overrightarrow{v}_\text{đạn/đất}$$
$$\Leftrightarrow \left(m_1+m_2+m_3\right)\overrightarrow{v}_\text{xe/đất}=\left(m_1+m_2\right)\overrightarrow{v'}_\text{xe/đất}+m_3\left(\overrightarrow{v}_\text{đạn/xe}+\overrightarrow{v}_\text{xe/đất}\right)$$
Chiếu phương trình bảo toàn động lượng lên chiều chuyển động ban đầu của xe:
$$\Leftrightarrow \left(m_1+m_2+m_3\right)v_\text{xe/đất}=\left(m_1+m_2\right)v'_\text{xe/đất}+m_3\left(v_\text{đạn/xe}+v_\text{xe/đất}\right)$$
\begin{eqnarray*}
	\Rightarrow v'_\text{xe/đất}&=&\dfrac{\left(m_1+m_2+m_3\right)v_\text{xe/đất}-m_3\left(v_\text{đạn/xe}+v_\text{xe/đất}\right)}{m_1+m_2}\\
	&=&\dfrac{\left(\SI{130}{\kilogram}+\SI{20}{\kilogram}+\SI{1}{\kilogram}\right)\cdot\left(\SI{5}{\meter/\second}\right)-\left(\SI{1}{\kilogram}\right)\cdot\left(\SI{400}{\meter/\second}+\SI{5}{\meter/\second}\right)}{\SI{130}{\kilogram}+\SI{20}{\kilogram}}\approx\SI{2.33}{\meter/\second}
\end{eqnarray*}
}

\end{enumerate}