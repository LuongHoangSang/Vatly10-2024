
\setcounter{section}{0}
\begin{enumerate}[label=\bfseries Câu \arabic*:]
	\item \mkstar{2}
	
	{
		
		Hãy tìm hiểu trên internet và cho biết một số sự kiện trong khám phá vũ trụ gần đây.
	}
	
	\hideall{
		
		Năm 2021: 
		
		- Cao điểm chinh phục hành tinh Đỏ vào tháng 2: vào năm 2020, ba quốc gia đã phóng tàu vũ trụ lên sao Hỏa và hiện nay, ba nhiệm vụ riêng biệt này đang trên đường đến hành tinh Đỏ.
		
		- Phóng kính viễn vọng không gian James Webb sau 25 năm phát triển.
		
		- Ấn Độ tiếp tục chinh phục mặt trăng.
		
		- Nói lời từ biệt với tàu vũ trụ Juno: Tàu vũ trụ Juno đã quay quanh sao Mộc từ tháng 7-2016 và sứ mệnh của nó sẽ kết thúc vào tháng 7-2021.
		
		- Phóng tàu thăm dò vũ trụ Lucy khám phá các tiểu hành tinh: Một tàu vũ trụ khác của NASA mang tên Lucy sẽ được phóng lên vũ trụ vào tháng 10 năm 2021 trong một hành trình kéo dài 12 năm để thăm tám tiểu hành tinh khác nhau.
		
		- Bước nhảy vọt trở lại mặt trăng của NASA.
		
		- SpaceX tiếp tục thực hiện tham vọng chinh phục vũ trụ.
		
		- Trạm vũ trụ của Trung Quốc.
		
		- Sứ mệnh đánh dạt các tiểu hành tinh. 
		
		
		
	}
	
	\item \mkstar{2}
	
	
	{
		Vật lý hạt cơ bản là gì? Các hạt cơ bản cấu tạo nên vật chất được phân loại như thế nào?
	}
	
	\hideall
	{
		Vật lí hạt cơ bản nghiên cứu về các hạt sơ cấp chứa trong vật chất và những tương tác giữa chúng.
		
		Phân loại hạt sơ cấp:
		
		- Photon: có $m_0 =0$.
		
		- Lepton: các hạt nhẹ như electron, muyon ($\mu^+, \mu^+$), cái hạt tau ($\tau^+; \tau^-$)
		
		- Mezon: gồm 2 nhím: mezon $\pi$ và mezon $K$.
		
		- Barion: gồm các hạt nặng có khối lượng bằng hoặc lớn hơn khối lượng proton. Có hau nhóm: nuclon và hiperon, cùng các phản hạt của chúng.
		
		Tập hợp các mezon và các barion có tên chung là hadron.
		
	}
	
	\item \mkstar{2}
	
	
	{
		Vật liệu nano là gì? Các nhà khoa học nghiên cứu vật liệu nano như thế nào? Nêu một số ứng dụng của vật liệu nano. Tại sao các vật liệu có kích cỡ nano lại có những tính chất rất khác biệt?
	}
	
	\hideall
	{
		- Vật liệu nano có cấu trúc các hạt, các sợi, ống, hay các tấm mỏng,... Có kích thước rất nhỏ khoảng từ 1 - 100 nanomet (nm).
		
		- Năm 1959, khái niệm về công nghệ nano được nhà vật lý người Mỹ Richard Feynman nhắc đến khi ông đề cập tới khả năng chế tạo vật chất ở kích thước siêu nhỏ đi từ quá trình tập hợp các nguyên tử, phân tử. Những năm 1980, nhờ sự ra đời của hàng loạt các thiết bị phân tích, trong đó có kính hiển vi đầu dò quét (SEM hay TEM) có khả năng quan sát đến kích thước vài nguyên tử hay phân tử, con người có thể quan sát và hiểu rõ hơn về lĩnh vực nano. 
		
		- Ứng dụng: 
		
		+ Trong y - sinh học các hóa chất và dược phẩm kích thước cỡ nano khi đưa vào cơ thể giúp can thiệp ở quy mô phân tử hay tế bào, dùng để hỗ trợ chuẩn đoán bệnh, dẫn truyền thuốc, tiêu diệt các tế bào ung thư,...
		
		+ Trong sản xuất năng lượng, sử dụng vật liệu nano để chế tạo các loại pin, tụ điện làm tăng tín hiệu dữ trữ điện năng.
		
		+ ...
		
		- Tính chất thú vị của vật liệu nano bắt nguồn từ kích thước của chúng rất nhỏ bé có thể so sánh với các kích thước tới hạn của nhiều tính chất hóa lí của vật liệu. Vật liệu nano nằm giữa tính chất lượng tử của nguyên tử và tính chất khối của vật liệu. Đối với vật liệu khối, độ dài tới hạn của các tính chất rất nhỏ so với độ lớn của vật liệu, nhưng đối với vật liệu nano thì điều đó không đúng nên các tính chất khác lạ bắt đầu từ nguyên nhân này.
	}
	\item \mkstar{2}
	
	
	{
		Hãy nêu một số ưu điểm của laser so với ánh sáng thông thường.
	}
	
	\hideall
	{
		Laser phát ra chùm ánh sáng có tính định hướng cao, có nghĩa là các sóng ánh sáng thành phần cùng truyền theo một đường thẳng mà hầu như không phân tán cách xa nhau. Các nguồn sáng thông thường phát ra sóng ánh sáng lan truyền theo mọi hướng. Các sóng ánh sáng trong chùm tia laser đều có cùng màu sắc (tính chất này được gọi là tính đơn sắc). Ánh sáng thông thường (chẳng hạn như ánh sáng từ đèn huỳnh quang) nói chung là sự pha trộn của một số màu sắc kết hợp với nhau và kết quả là có màu trắng.
	}
	\item \mkstar{2}
	
	{
		Laser là gì? Hãy nêu ứng dụng của laser trong đời sống.
	}
	
	\hideall{
		
		- Thuật ngữ “laser” có nguồn gốc là chữ viết tắt của “light amplification by stimulated emission of radiation” (khuếch đại ánh sáng bằng bức xạ kích thích). Tia laser là một loại ánh sáng đặc biệt được tạo ra nhờ “Sự khuếch đại ánh sáng bằng bức xạ phát ra khi kích hoạt cao độ các phần tử của môi trường”. Laser được tạo ra nhờ thuyết lượng tử của Albert Einstein.
		
		- Ứng dụng của tia laser trong hoạt động đời sống.
		
		+  Ứng dụng trong y học.
		
		Trong y học người ta sử dụng tia laser như là một dao mỗ vì nó có dễ điều khiển, diện tích nhỏ và dễ kiểm soát. Laser có thể dùng để chữa hoặc phục hồi các bệnh về mắ như cận thị. Ngoài ra người ta còn kết hợp với sợi quang học để chuẩn đoán ung thư…
		
		+ Ứng dụng trong thẩm mỹ
		
		Tia laser còn được người ta sử dụng nhiều trong các dịch vụ chăm sóc da như tái tạo bề mặt da, làm săn chắc da, xóa bỏ xăm…
		
		+ Ứng dụng trong các ngành công nghệ.
		
		Tia laser được sử dụng để làm nóng chảy các bề mặt kim loại. Chính vì thế nên được ứng dụng rất nhiều trong ngành công nghiệp nặng được biết đến với một số sản phẩm như:
		
		máy cắt laser dùng các nguồn laser công suất cao để làm nóng chảy kim loại tạo ra các hình thù hoa văn trên bề mặt kim loại.
		
		Máy hàn laser giúp gắn, đính các khoảng trống, khớp nối ở kim loại..
		
		+ Ứng dụng trong một số ngành khác.
		
		Sử dụng tia laser để thuận tiện cho việc bán hàng như máy quét mã vạch. Hay ứng dụng trong các máy đo đạc, máy cân bằng, thủy bình…
		
		
	}
	
	\item \mkstar{2}
	
	
	{
		Hãy kể tên một số ứng dụng của vật lí bán dẫn trong đời sống và khoa học, kĩ thuật.
	}
	
	\hideall
	{
		- Vật liệu bán dẫn có thể chế tạo được các linh kiện rất nhỏ, vì vậy người ta đã dùng để chế tạo ra các mạch tổ hợp (mạch IC) hoặc các mạch IC siêu lớn.
		
		- Nhờ tính nhạy sáng và nhiệt độ của vật kiệu bán dẫn, người ta có thể chế tạo các thiết bị cảm biến dùng trong các hệ thống điều khiển tự động.
		
	}
	\item \mkstar{2}
	
	
	{
		Đặc điểm nào của tia X là cơ sở cho phương pháp chụp X - quang?
	}
	
	\hideall
	{
		- Tính truyền thẳng và đâm xuyên.
		
		- Tính bị hấp thụ.
		
		- Tính chất quang học.
		
		- Tác dụng sinh học.
		
		Máy X-quang sử dụng loại bức xạ ánh sáng hoặc sóng vô tuyến. Một ống đặc biệt bên trong máy sẽ phát ra các chùm tia X có bức xạ cao, được các mô trong cơ thể hấp thụ ở những mức độ khác nhau. Các mô dày đặc như xương sẽ chặn hầu hết tia bức xạ, trong khi các mô mềm như mỡ hoặc cơ, chặn ít hơn.
		
		Sau khi đi qua cơ thể, chùm tia X chiếu vào một tấm phim hoặc máy dò đặc biệt. Các mô chặn lượng bức xạ cao, chẳng hạn như xương, hiển thị dưới dạng vùng trắng trên nền đen. Các mô mềm ngăn chặn ít bức xạ hơn được hiển thị với màu xám. Những khối u thường dày đặc hơn các mô xung quanh, vì vậy chúng có màu xám nhạt hơn. Các cơ quan chủ yếu là không khí (chẳng hạn như phổi) thường có màu đen.
	}
	\item \mkstar{2}
	
	
	{
		Hãy tìm hiểu trên internet về công nghệ hiện tại cũng như sự phát triển các công nghệ mới trong vật lí bán dẫn.
	}
	
	\hideall
	{
		GaN - vật liệu nắm giữ tương lai ngành bán dẫn
		
		Sở hữu hiệu suất vượt trội, những con chip sử dụng Gallium Nitride (GaN) được cho là sẽ mở ra một kỷ nguyên mới trong lĩnh vực điện tử công suất.
		
		Cuộc khủng hoảng chất bán dẫn toàn cầu đang làm trì hoãn việc sản xuất mọi thứ từ tủ lạnh, lò vi sóng đến máy chơi game, smartphone. Các chuyên gia cho rằng có thể mất nhiều tháng để ngành công nghiệp bán dẫn phục hồi, nhưng trên thực tế, tình trạng thiếu hụt chip này đang thay đổi ngành điện tử tiêu dùng mãi mãi.
		
		Trong nhiều thập kỷ, silicon luôn là vật liệu thống trị trong ngành công nghiệp bán dẫn, nhưng khoảng thời gian ngặt nghèo thiếu chip silicon gần đây đang giúp cho một vật liệu mới lên ngôi với tiềm năng làm cho các thiết bị điện tử thân thiện với môi trường, hiệu quả và nhỏ hơn. Nhiều công ty đang chuyển sang sử dụng Gallium Nitride (GaN) vì nó dễ sản xuất hơn và nhanh hơn so với chip silicon.
	}
	\item \mkstar{2}
	
	
	{
		Em hãy tìm hiểu trên internet và thảo luận để tìm hiểu về các công nghệ hiện tại cũng như phát triển các công nghệ mới trong vật lí y sinh.
	}
	
	\hideall
	{
		Phạm vi nghiên cứu của lý sinh học trải từ so sánh chuỗi đến mạng thần kinh. Trong những năm gần đây, lý sinh học còn nghiên cứu đến chế tạo chi cơ học và thiết bị nano để điều hoà chức năng sinh học. Ngày nay các nghiên cứu đó thường được xem là thuộc về lĩnh vực tương ứng của công nghệ sinh học và công nghệ nano.
		
		Nghiên cứu truyền thống trong sinh học được tiến hành bằng các thí nghiệm tổng thể thống kê (statistical ensemble), dùng nồng độ mol của các đại phân tử. Vì các phân tử bên trong tế bào sống có số lượng ít, các kỹ thuật như khuếch đại PCR, thấm gel (gel blotting), gắn kết huỳnh quang và nhuộm in vivo được dùng để có thể xem kết quả thí nghiệm bằng mắt thường hoặc, it nhất, với thiết bị phóng đại quang học. Bằng các kỹ thuật này, nhà sinh học cố gắng làm sáng tỏ hệ thống tương tác phức tạp tạo ra các tiến trình cho sự sống. Lý sinh học cũng quan tâm đến những vấn đề tương tự trong sinh học, nhưng đặt ở mức độ một phân tử (nghĩa là số Reynolds thấp). Bằng cách áp dụng kiến thức và kỹ thuật thí nghiệm từ nhiều chuyên ngành, nhà lý sinh có thể quan sát gián tiếp hoặc mô hình hoá cấu trúc và tương tác của từng phân tử hay phức hợp phân tử.
	}
	\item \mkstar{2}
	
	
	{
		Vì sao có thể nói các nhà vật lí đã tạo ra loại dao phẫu thuật tốt nhất cho các bác sĩ?
	}
	
	\hideall
	{
		Khi phẫu thuật với dao mổ truyền thống, vết cắt thường sẽ khá dài và sâu, gây ảnh hưởng đến các mô trong phạm vi khá lớn xung quanh. Thời gian phẫu thuật bằng dao mổ thông thường cũng kéo dài, rất dễ xảy ra những biến chứng bất lợi. Ngay cả khi ca phẫu thuật thuận lợi, bệnh nhân vẫn cần nhiều thời gian để có thể khôi phục sau đại phẫu.
		
		Laser có thể được điều khiển tập trung vào một vùng rất nhỏ trên cơ thể để phá hủy, cắt bỏ nó mà rất ít gây tổn thương cho các mô xung quanh. Đây là cơ sở để tạo các dao phẫu thuật bằng tia laser. Những bệnh nhân được phẫu thuật bằng tia laser sẽ ít bị đau, cũng giảm sưng và tạo sẹo hơn so với bệnh nhân được phẫu thuật bằng các phương pháp truyền thống.
		
		Hiện nay, bằng cách sử dụng xung laser cường độ lớn trong thời gian 1 phần triệu của một phần tỉ giây, các nhà vật lí đã có thể làm phá hủy các cấu trúc nhỏ bé bên trong một tế bào sống mà không giết chết tế bào đó. Đây là cơ sở của phương pháp phẫu thuật nano laser, giúp giới khoa học tìm hiểu cơ chế hoạt động của tế bào cũng như thực hiện các ca phẫu thuật siêu chính xác.
	}
\end{enumerate}