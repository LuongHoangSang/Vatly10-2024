\let\lesson\undefined
\newcommand{\lesson}{\phantomlesson{Bài 15.}}
\setcounter{section}{2}
\section{Trắc nghiệm}
\begin{enumerate}[label=\bfseries Câu \arabic*:, leftmargin=1.5cm]
	
	\item \mkstar{1}
	
	
	{	Công là đại lượng
		\begin{mcq}
			\item vô hướng, có thể âm hoặc dương. 
			\item vectơ, có thể âm, dương hoặc bằng 0.
			\item vectơ, có thể âm hoặc dương.
			\item vô hướng, có thể âm, dương hoặc bằng 0.
		\end{mcq}
	}
	
	\hideall
	{	\textbf{Đáp án: D.}
		
		Công là đại lượng vô hướng, có thể âm, dương hoặc bằng 0.
	}
	\item \mkstar{1}
	
	
	{	Trường hợp nào sau đây công của lực bằng không?
		\begin{mcq}
			\item Lực vuông góc với phương chuyển động của vật. 
			\item Lực cùng phương với phương chuyển động của vật.
			\item Lực hợp với phương chuyển động một góc lớn hơn $90^\circ$.
			\item Lực hợp với phương chuyển động một góc nhỏ hơn $90^\circ$.
		\end{mcq}
	}
	
	\hideall
	{	\textbf{Đáp án: A.}
		
		Khi lực vuông góc với phương chuyển động của vật thì $\alpha=90^\circ$, khi đó $\cos \alpha =0$, dẫn đến $A=Fs\cos \alpha=0$.
	}
	
	\item \mkstar{2}
	
	
	{
		Chọn câu \textbf{sai}.
		\begin{mcq}
			\item Công của lực cản âm vì $90^\circ < \alpha < 180^\circ$.
			\item Công của lực phát động dương vì $90^\circ > \alpha > 0^\circ$.
			\item Vật dịch chuyển theo phương nằm ngang thì công của trọng lực bằng $0$.
			\item Vật dịch chuyển trên mặt phẳng nghiêng thì công của trọng lực bằng $0$.
		\end{mcq}
	}
	
	\hideall
	{	
		\textbf{Đáp án: D.}
		
		Vật dịch chuyển trên mặt phẳng nghiêng thì công của trọng lực khác $0$, vì phương của trọng lực không vuông góc với phương của mặt nghiêng.
	}
	
	\item \mkstar{2}
	
	
	{
		Sử dụng một lực $F=\SI{50}{N}$ tạo với phương ngang một góc $\alpha = 60^\circ$ kéo một vật và làm vật chuyển động thẳng đều trên mặt phẳng nằm ngang. Công của lực kéo khi vật di chuyển được một đoạn đường bằng $\SI{6}{m}$ là
		\begin{mcq}(4)
			\item $\SI{0}{J}$. 
			\item $\SI{260}{J}$.
			\item $\SI{300}{J}$.
			\item $\SI{150}{J}$.
		\end{mcq}
	}
	
	\hideall
	{	
		\textbf{Đáp án: D.}
		
		Công của lực kéo:
		$$A=Fs\cos \alpha  =\SI{150}{J}.$$
		
	}
	\item \mkstar{2}
	
	
	{
		Một người kéo một thùng gỗ trượt trên sàn nhà bằng một sợi dây hợp với phương ngang một góc $60^\circ$, lực tác dụng lên dây là $\SI{200}{N}$. Khi thùng gỗ được kéo và trượt một đoạn $\SI{10}{m}$ thì công của lực kéo là
		\begin{mcq}(4)
			\item $\SI{200}{J}$. 
			\item $\SI{1000}{J}$.
			\item $\SI{2000}{J}$.
			\item $\SI{120000}{J}$.
		\end{mcq}
	}
	
	\hideall
	{	
		\textbf{Đáp án: B.}
		
		Công của lực kéo:
		$$A=Fs\cos \alpha = \SI{1000}{J}.$$
	}
	
	\item \mkstar{2}
	
	
	{
		Vật nào sau đây \textbf{không} có khả năng sinh công?
		\begin{mcq}
			\item Vật đang rơi tự do xuống mặt đất. 
			\item Dòng nước từ trên cao đổ mạnh xuống làm quay tuabin nước.
			\item Vật đang nằm yên trên mặt đất.
			\item Viên đạn đang bay.
		\end{mcq}
	}
	
	\hideall
	{	
		\textbf{Đáp án: C.}
		
		Vật đang nằm yên trên mặt đất không có khả năng sinh công.
	}
	\item \mkstar{2}
	
	
	{Công có thể biểu thị bằng tích của:
		\begin{mcq}
			\item Năng lượng và khoảng thời gian.
			\item Lực, quãng đường đi được và khoảng thời gian.
			\item Lực và quãng đường đi được.
			\item Lực và vận tốc.
		\end{mcq}
	}
	
	\hideall
	{	
		\textbf{Đáp án: C.}
	}
	\item \mkstar{2}
	
	
	{
		Người ta kéo một cái thùng nặng $\SI{20}{kg}$ trượt trên sàn nhà bằng một dây hợp với phương nằm ngang một góc $60^\circ$, lực tác dụng lên dây là $\SI{300}{N}$. Tính công của lực đó khi thùng trượt được $\SI{10}{m}$. 
		\begin{mcq}(4)
			\item $\SI{1000}{J}.$
			\item $\SI{500}{J}.$
			\item $\SI{1500}{J}.$
			\item $\SI{100}{J}.$
		\end{mcq}
	}
	
	\hideall
	{	
		\textbf{Đáp án: C.}
		
		Công của lực $F$ kéo thùng đi được $\SI{10}{m}$ là:
		
		$$A = Fs\cos \alpha = \SI{1500}{J}.$$
	}
	
	\item \mkstar{2}
	
	
	{Một chiếc ô tô sau khi tắt máy còn đi được $\SI{10}{m}$. Biết ô tô nặng 1,5 tấn, hệ số cản bằng 0,25 (Lấy $g = \SI{9,8}{m/s}^2$). Công của lực cản có giá trị:
		\begin{mcq}(4)
			\item $-\SI{36750}{J}.$
			\item $\SI{36750}{J}.$
			\item $\SI{18375}{J}.$
			\item $-\SI{18375}{J}.$
		\end{mcq}
	}
	
	\hideall
	{	
		\textbf{Đáp án: A.}
		
		Công của lực cản là: 
		
		$$A_\text{cản} = F_\text{ms}s = \mu mgs = \SI{36750}{J}.$$
		
		Vì công cản nên $A < 0 \Rightarrow A = - \SI{36750}{J}.$
	}
	\item \mkstar{2}
	
	
	{Tác dụng lực không đổi $\SI{150}{N}$ theo phương hợp với phương ngang góc $30^\circ$ vào vật khối lượng $\SI{80}{kg}$ làm vật chuyển động được quãng $\SI{20}{m}$. Tính công của lực tác dụng.
		\begin{mcq}(4)
			\item $\SI{2598}{J}.$
			\item $\SI{598}{J}.$
			\item $\SI{298}{J}.$
			\item $\SI{258}{J}.$
		\end{mcq}
	}
	
	\hideall
	{	
		\textbf{Đáp án: A.}
		
		Công của lực tác dụng
		
		$$A=Fs\cos \alpha =\SI{2598}{J}.$$
	}
\end{enumerate}
\ANSMCQ
{		\begin{center}
		\begin{tabular}{|m{2.8em}|m{2.8em}|m{2.8em}|m{2.8em}|m{2.8em}|m{2.8em}|m{2.8em}|m{2.8em}|m{2.8em}|m{2.8em}|}
			\hline
			1.D  & 2.A  & 3.D  & 4.D  & 5.B  & 6.C  & 7.C  & 8.C  & 9.A  & 10.A  \\
			\hline
			
		\end{tabular}
	\end{center}
}
\section{Tự luận}
\begin{enumerate}[label=\bfseries Câu \arabic*:, leftmargin=1.5cm]
	\item \mkstar{2}
	
	
	{
		Một hành khách kéo đều một vali đi trong nhà ga trên sân bay trên quãng đường dài $\SI{150}{m}$ với lực kéo có độ lớn $\SI{40}{N}$ theo hướng hợp với phương ngang một góc $60^\circ$. Hãy xác định công của lực kéo của người này.
	}
	
	\hideall
	{	
		Công của lực kéo của người: $$A=Fs\cos \alpha = \SI{3000}{J}.$$
	}
	
	\item \mkstar{2}
	
	
	{
		Tính công của lực $F=\SI{1200}{N}$ tác dụng lên vật làm vật dịch chuyển quãng đường $\SI{4}{m}$, biết góc hợp bởi chiều của lực và chiều dịch chuyển là $60^\circ$.
	}
	
	\hideall
	{	
		Công của lực: $$A=Fs\cos \alpha = \SI{2400}{J}.$$
	}
	
	\item \mkstar{2}
	
	
	{
		Con ngựa kéo chiếc xe với một lực kéo $F=\SI{100}{N}$ theo phương nằm ngang. Chiếc xe chuyển động thẳng đều trên đường nằm ngang với vận tốc $\SI{8}{m/s}$ trong thời gian 5 giây. Tính công của lực kéo của con ngựa ở đoạn đường trên. 
	}
	
	\hideall
	{	
		Quãng đường con ngựa kéo xe:
		$$s=vt=\SI{40}{m}.$$
		
		Công của lực kéo:
		$$A=Fs=\SI{4000}{J}.$$
	}
	\item \mkstar{2}
	
	
	{
		Một thùng nước khối lượng $\SI{10}{kg}$ được kéo cho chuyển động thẳng đều lên cao $\SI{5}{m}$ trong thời gian 1 phút 40 giây. Tính công của lực kéo. Lấy $g=\SI{10}{m/s^2}$.
	}
	
	\hideall
	{	
		Lực kéo thùng nước để thùng chuyển động thẳng đều:
		$$F=P=mg=\SI{100}{N}.$$
		
		Công của lực kéo:
		$$A=Fs=\SI{500}{J}.$$
	}
	\item \mkstar{2}
	
	
	{
		Khi rửa gầm xe ô tô, người ta sử dụng máy nâng ô tô lên độ cao $h = \SI{160}{cm}$ so với mặt sàn. Cho biết khối lượng ô tô là $m = \text{1,5}\ \text{tấn}$. Tính công tối thiểu mà máy nâng đã thực hiện.
	}
	
	\hideall
	{	
		Để nâng được ô tô thì máy nâng phải tác dụng vào ô tô một lực có độ lớn tối thiểu bằng trọng lượng của ô tô:
		
		$$F = P =mg = \text{1,5} \cdot 10^4\ \text{N}.$$
		
		Công tối thiểu mà máy nâng đã thực hiện là:
		
		$$A = Ph = \SI{24000}{J} = \SI{24}{kJ}.$$
		
	}
	
	
\end{enumerate}