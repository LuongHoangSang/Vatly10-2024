\let\lesson\undefined
\newcommand{\lesson}{\phantomlesson{Bài 18.}}
\setcounter{section}{2}
\section{Trắc nghiệm}
\ANSMCQ
{	\begin{center}
		\begin{tabular}{|m{2.8em}|m{2.8em}|m{2.8em}|m{2.8em}|m{2.8em}|m{2.8em}|m{2.8em}|m{2.8em}|m{2.8em}|m{2.8em}|}
			\hline
			1.D  & 2.A  & 3.C  & 4.B  & 5.C  & 6.A  & 7.D  & 8.B  & 9.D  & 10.A  \\
			\hline
			11.A  & 12. C  & 13.D  & 14.A  & 15.A  & &   &   &   &   \\
			\hline
		\end{tabular}
	\end{center}
}
\begin{enumerate}[label=\bfseries Câu \arabic*:, leftmargin=1.5cm]
	\item \mkstar{2}
	
	
	{
		Phát biểu nào sau đây \textbf{sai}?
		\begin{mcq}
			\item Động lượng là một đại lượng vectơ.
			\item Xung lượng của lực là một đại lượng vectơ.
			\item Động lượng tỉ lệ với khối lượng của vật.
			\item Động lượng của vật trong chuyển động tròn đều không đổi.
		\end{mcq}
	}
	
	\hideall
	{	
		\textbf{Đáp án: D.}
		
		Vectơ vận tốc trong chuyển động tròn đều thay đổi theo thời gian nên động lượng của vật trong chuyển động tròn đều thay đổi theo thời gian.
	}
	\item \mkstar{2}
	
	
	{
		Động lượng là đại lượng vectơ
		\begin{mcq}
			\item cùng phương, cùng chiều với vectơ vận tốc. 
			\item cùng phương, ngược chiều với vectơ vận tốc.
			\item có phương vuông góc với vectơ vận tốc.
			\item có phương hợp với vectơ vận tốc một góc $\alpha$ bất kì.
		\end{mcq}
	}
	
	\hideall
	{	
		\textbf{Đáp án: A.}
		
		Động lượng là đại lượng vectơ cùng phương, cùng chiều với vectơ vận tốc.
	}
	
	
	\item \mkstar{2}\\
	Chọn câu phát biểu \textbf{sai}?
	\begin{mcq}
		\item Động lượng là một đại lượng véctơ.
		\item Động lượng luôn được tính bằng tích khối lượng và vận tốc của vật.
		\item Động lượng luôn cùng hướng với vận tốc vì vận tốc luôn luôn dương.
		\item Động lượng luôn cùng hướng với vận tốc vì khối lượng luôn luôn dương.
	\end{mcq}
\hideall{
\textbf{Đáp án C.}
}

\item\mkstar{2}\\
{Hai vật có khối lượng $m_1 = 2m_2$, chuyển động với vận tốc có độ lớn $v_1 = 2v_2$. Độ lớn động lượng của hai vật có quan hệ
	\begin{mcq}(4)
		\item $p_1=2p_2$.
		\item $p_1=4p_2$.
		\item $p_2=4p_1$.
		\item $p_1=p_2$.
	\end{mcq}
}
\hideall{
\textbf{Đáp án B.}\\
$$\dfrac{p_1}{p_2}=\dfrac{m_1v_1}{m_2v_2}=4.$$
}


\item \mkstar{2}\\
Một vật có khối lượng $\SI{500}{\gram}$ chuyển động theo chiều âm của trục toạ độ $Ox$ với tốc độ $\SI{12}{\meter/\second}$. Động lượng của vật có giá trị là
\begin{mcq}(4)
	\item $\SI{6}{\kilogram\cdot\meter/\second}$.
	\item $\SI{-3}{\kilogram\cdot\meter/\second}$.
	\item $\SI{-6}{\kilogram\cdot\meter/\second}$.
	\item $\SI{3}{\kilogram\cdot\meter/\second}$.
\end{mcq}
\hideall{
\textbf{Đáp án C.}\\
Động lượng của vật:
$$p=mv=\left(\SI{0.5}{\kilogram}\right)\cdot\left(\SI{-12}{\meter/\second}\right)=\SI{-6}{\kilogram\cdot\meter/\second}.$$
}
	
		\item \mkstar{2}
	
	
	{
		Một electron chuyển động với tốc độ $\xsi{2\cdot 10^7}{m/s}$. Biết khối lượng electron bằng $\text{9,1} \cdot 10^{-31}\ \text{kg}.$ Động lượng của electron bằng
		
		\begin{mcq}(2)
			\item $\text{1,82}\xsi{\cdot 10^{-23}}{kg.m/s}$.
			\item $\text{82}\xsi{\cdot 10^{-23}}{kg.m/s}$.
			\item $\text{1,2}\xsi{\cdot 10^{-23}}{kg.m/s}$.
			\item $\text{1,8}\xsi{\cdot 10^{-23}}{kg.m/s}$.
		\end{mcq}
		
	}
	
	\hideall
	{	
		\textbf{Đáp án: A.}
		
		Động lượng của electron
		
		$$ p  = mv = \text{1,82}\xsi{\cdot 10^{-23}}{kg.m/s}.$$
	}
	\item \mkstar{2}
	
	
	{
		Một quả bóng nặng $\SI{1}{kg}$ đang đứng yên thì cầu thủ chạy đến sút quả bóng thật mạnh. Quả bóng bay đi với vận tốc $\SI{25}{m/s}$. Tính động lượng quả bóng.
		\begin{mcq}(4)
			\item $\SI{15}{kg.m/s}$
			\item $\SI{2}{kg.m/s}$
			\item $\SI{5}{kg.m/s}$
			\item $\SI{25}{kg.m/s}$
		\end{mcq}
	}
	
	\hideall
	{	
		\textbf{Đáp án: D.}
		
		Động lượng của quả bóng: $p=mv=\SI{25}{kg.m/s}$.
	}
	
	\item \mkstar{3}\\
	{Một chất điểm chuyển động không vận tốc đầu dưới tác dụng của lực không đổi $F =\SI{1}{\newton}$. Động lượng của chất điểm ở thời điểm $t = \SI{3}{\second}$ kể từ lúc bắt đầu chuyển động là
		\begin{mcq}(4)
			\item $\SI{30}{\kilogram\cdot\meter/\second}$.
			\item $\SI{3}{\kilogram\cdot\meter/\second}$.
			\item $\SI{0.3}{\kilogram\cdot\meter/\second}$.
			\item $\SI{0.03}{\kilogram\cdot\meter/\second}$.
		\end{mcq}
	
}
\hideall{
\textbf{Đáp án B.}\\
Động lượng của chất điểm ở thời điểm $t=\SI{3}{\second}$:
$$p=F\cdot t=\SI{3}{\kilogram\cdot\meter/\second}.$$
}
	
	\item\mkstar{3}\\
	{Một quả bóng có khối lượng $m$ đang bay ngang với vận tốc $v$ thì đập vào một bức tường rồi bật trở lại với cùng tốc độ. Xung lượng của lực gây ra bởi tường lên quả bóng là
		\begin{mcq}(4)
			\item $mv$.
			\item $-mv$.
			\item $2mv$.
			\item $-2mv$.
		\end{mcq}
}
\hideall{
\textbf{Đáp án D.}\\
Xung lượng của lực gây ra bởi tường lên quả bóng:
$$F\Delta t=\Delta p=m\cdot\left(-v\right)-mv=-2mv.$$
}
	
	\item \mkstar{3}
	
	
	{
		Một quả bóng golf có khối lượng $\SI{46}{g}$ đang nằm yên, sau một cú đánh quả bóng bay lên với tốc độ $\SI{70}{m/s}$. Tính độ lớn trung bình của lực tác dụng vào quả bóng. Biết thời gian tác dụng là $\text{0,5} \cdot 10^{-3}\ \text s.$
		\begin{mcq}(4)
			\item $\SI{6400}{N}$.
			\item $\SI{600}{N}$.
			\item $\SI{400}{N}$.
			\item $\SI{3400}{N}$.
		\end{mcq}
	}
	
	\hideall
	{	\textbf{Đáp án: A.}
		
		Xung lượng của lực
		
		$$\Delta \vec p = \vec {p'} - \vec p \Rightarrow \Delta p = \SI{3,22}{kgm/s}.$$
		
		Lực tác dụng vào quả bóng
		
		$$F = \dfrac{\Delta p}{\Delta t} = \SI{6400}{N}.$$
	}

\item \mkstar{3}\\
{Viên đạn có khối lượng $\SI{20}{\gram}$ đang bay với tốc độ $\SI{600}{\meter/\second}$ thì gặp một cánh cửa thép. Đạn xuyên qua cửa trong thời gian $\SI{0.002}{\second}$. Sau khi xuyên qua cánh cửa tốc độ của đạn còn $\SI{300}{\meter/\second}$. Lực cản trung bình của cửa tác dụng lên đạn có độ lớn bằng
	\begin{mcq}(4)
		\item $\SI{3000}{\newton}$.
		\item $\SI{900}{\newton}$.
		\item $\SI{9000}{\newton}$.
		\item $\SI{30000}{\newton}$.
	\end{mcq}

}
\hideall{
\textbf{Đáp án A.}\\
Lực cản trung bình của cửa tác dụng lên đạn:
$$F_c=\dfrac{\Delta p}{\Delta t}=\dfrac{m\Delta v}{\Delta t}=\dfrac{\left(\SI{0.02}{\kilogram}\right)\cdot\left(\SI{-300}{\meter/\second}\right)}{\SI{0.002}{\second}}=\SI{-3000}{\newton}.$$
}


\item \mkstar{3}\\
{Cho hệ hai vật có khối lượng bằng nhau $m_1 = m_2 = \SI{1}{\kilogram}$. Vận tốc của vật 1 có độ lớn $v_1 =\SI{1}{\meter/\second}$, vận tốc của vật 2 có độ lớn $v_2 =\SI{2}{\meter/\second}$. Khi vectơ vận tốc của hai vật cùng hướng với nhau, tổng động lượng của hệ có độ lớn là
	\begin{mcq}(4)
		\item $\SI{1}{\kilogram\cdot\meter/\second}$.
		\item $\SI{2}{\kilogram\cdot\meter/\second}$.
		\item $\SI{3}{\kilogram\cdot\meter/\second}$.
		\item $\SI{0.5}{\kilogram\cdot\meter/\second}$.
	\end{mcq}
}
\hideall{
\textbf{Đáp án C.}\\
Động lượng của hệ:
$$\vec p=\vec p_1+\vec p_2$$
Vì $\vec p_1\uparrow\uparrow\vec p_2$ nên:
$$p=p_1+p_2=m_1v_1+m_2v_2=\SI{3}{\kilogram\cdot\meter/\second}.$$
}

\item \mkstar{3}\\
{Hai vật có khối lượng $m_1=\SI{2}{\kilogram}$ và $m_2=\SI{3}{\kilogram}$ chuyển động ngược chiều nhau với tốc độ lần lượt bằng $\SI{8}{\meter/\second}$ và $\SI{4}{\meter/\second}$. Độ lớn tổng động lượng của hệ bằng
	\begin{mcq}(4)
		\item $\SI{16}{\kilogram\cdot\meter/\second}$.
		\item $\SI{12}{\kilogram\cdot\meter/\second}$.
		\item $\SI{30}{\kilogram\cdot\meter/\second}$.
		\item $\SI{4}{\kilogram\cdot\meter/\second}$.
	\end{mcq}

}
\hideall{
\textbf{Đáp án D.}\\
Động lượng của hệ:
$$\vec p=\vec p_1+\vec p_2$$
Vì $\vec p_1 \uparrow\downarrow \vec p_2$ nên:
$$p=\left|m_2v_2-m_1v_1\right|=\SI{4}{\kilogram\cdot\meter/\second}.$$
}

\item \mkstar{3}\\
{Một hệ gồm 2 vật có khối lượng $m_1 = \SI{1}{\kilogram}$, $m_2 =\SI{4}{\kilogram}$, có vận tốc lần lượt là $v_1 = \SI{3}{\meter/\second}$, $v_2=\SI{1}{\meter/\second}$. Biết 2 vật chuyển động theo hướng vuông góc nhau. Độ lớn động lượng của hệ là
	\begin{mcq}(4)
		\item $\SI{5}{\kilogram\cdot\meter/\second}$.
		\item $\SI{10}{\kilogram\cdot\meter/\second}$.
		\item $\SI{20}{\kilogram\cdot\meter/\second}$.
		\item $\SI{14}{\kilogram\cdot\meter/\second}$.
	\end{mcq}

}
\hideall{
\textbf{Đáp án A.}\\
Động lượng của hệ:
$$\vec p=\vec p_1+\vec p_2$$
Vì $\vec p_1 \bot \vec p_2$ nên:
$$p=\sqrt{p^2_1+p^2_2}=\SI{5}{\kilogram\cdot\meter/\second}.$$
}

\item \mkstar{3}\\
{Cho hệ hai vật có khối lượng bằng nhau $m_1 = m_2 =\SI{1}{\kilogram}$. Vận tốc của vật 1 có độ lớn $v_1 =\SI{1}{\meter/\second}$, vận tốc của vật 2 có độ lớn $v_2 =\SI{2}{\meter/\second}$. Khi vectơ vận tốc của hai vật hợp với nhau một góc $\SI{60}{\degree}$ thì tổng động lượng của hệ có độ lớn là
	\begin{mcq}(4)
		\item $\SI{2.65}{\kilogram\cdot\meter/\second}$.
		\item $\SI{26.5}{\kilogram\cdot\meter/\second}$.
		\item $\SI{28.9}{\kilogram\cdot\meter/\second}$.
		\item $\SI{2.89}{\kilogram\cdot\meter/\second}$.
	\end{mcq}
}
\hideall{
\textbf{Đáp án A.}\\
Động lượng của hệ:
$$\vec p=\vec p_1+\vec p_2$$
Vì $\left(\vec p_1, \vec p_2\right)=\SI{60}{\degree}$ nên:
$$p=\sqrt{p^2_1+p^2_2+2p_1p_2\cos\SI{60}{\degree}}\approx\SI{2.65}{\kilogram\cdot\meter/\second}.$$
}
\end{enumerate}
\section{Tự luận}
\begin{enumerate}[label=\bfseries Câu \arabic*:, leftmargin=1.5cm]
	
	\item \mkstar{2}
	
	
	{
		Một vật có khối lượng $m=\SI{1}{\kilogram}$ đang chuyển động với vận tốc $v=\SI{2}{\meter/\second}$. Tính động lượng của vật.
	}
	
	\hideall
	{	
		Động lượng của vật: $$p=mv=\SI{2}{kg.m/s}.$$
	}
	
	\item \mkstar{2}
	
	
	{
		Một quả bóng có khối lượng $m=\SI{300}{\gram}$ va chạm vào tường và nảy trở lại với cùng tốc độ. Tốc độ của bóng trước va chạm là $\SI{5}{\meter/\second}$. Tìm độ biến thiên động lượng của quả bóng.
	}
	
	\hideall
	{	
		Độ biến thiên động lượng của quả bóng:
		$$\Delta \vec p  = \vec p' - \vec p.$$
		
		Vì $\vec p'$ và $\vec p$ cùng phương, ngược chiều nên $$\Delta p = \left|2p\right| = \SI{3}{kg\cdot \meter/\second}.$$
	}
	
	\item \mkstar{2}
	
	
	{
		Xác định độ biến thiên động lượng của một vật có khối lượng $\SI{4}{\kilogram}$ sau khoảng thời gian 6 giây. Biết rằng vật chuyển động trên đường thẳng và có phương trình chuyển động là $x=t^2-6t+3$.
	}
	
	\hideall
	{	
		Gia tốc của vật: $a=\SI{2}{m/s^2}$.
		
		Độ biến thiên vận tốc của vật:
		$$\Delta v =a\Delta t= \SI{12}{m/s}.$$
		
		Độ biến thiên động lượng của vật: 
		
		$$\Delta p = m\Delta v = \SI{48}{\kilogram\cdot\meter/\second}.$$
	}
	
	\item \mkstar{3}
	
	
	{
		Một toa xe khối lượng 10 tấn đang chuyển động trên đường ray nằm ngang với tốc độ không đổi $v=\SI{54}{\kilo\meter/\hour}$, người ta tác dụng lên toa xe một lực hãm theo phương ngang. Tính độ lớn trung bình của lực hãm nếu toa xe dừng lại sau
		\begin{enumerate}[label=\alph*)]
			\item 1 phút 40 giây.
			\item 10 giây.
		\end{enumerate}
	}
	
	\hideall
	{	
		\begin{enumerate}[label=\alph*)]
			\item $F=\dfrac{\Delta p}{\Delta t}=\SI{1500}{\newton}$.
			\item $F=\dfrac{\Delta p}{\Delta t}=\SI{15000}{\newton}$.
		\end{enumerate}
	}
	\item \mkstar{3}
	
	
	{
		Một viên đạn khối lượng $\SI{10}{\gram}$ đang bay với tốc độ $\SI{600}{\meter/\second}$ thì gặp một bức tường. Đạn xuyên qua tường trong thời gian $\dfrac{1}{100}\,\text{s}$. Sau khi xuyên qua tường, tốc độ của đạn còn $\SI{200}{\meter/\second}$. Tính lực cản của tường tác dụng lên viên đạn.
	}
	
	\hideall
	{	
		Độ biến thiên động lượng của viên đạn:
		$$\Delta p = m\Delta v = \SI{4}{kg.m/s}.$$
		
		Lực cản của tường tác dụng lên viên đạn: $$F=\dfrac{\Delta p}{\Delta t} = \SI{400}{N}.$$
	}
	
\end{enumerate}