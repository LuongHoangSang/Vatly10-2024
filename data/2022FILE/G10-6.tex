\newcommand{\chapter}[2][]{
	\newcommand{\chapname}{#2}
	\begin{flushleft}
		\begin{minipage}[t]{\linewidth}
			\includegraphics[height=1cm]{hdht-logo.png}
			\hspace{0pt}	
			\sffamily\bfseries\large Bài  8 + Bài 9.
			\begin{flushleft}
				\huge\bfseries #1
			\end{flushleft}
		\end{minipage}
	\end{flushleft}
	\vspace{1cm}
	\normalfont\normalsize
}
\chapter[Chuyển động biến đổi - Gia tốc \\ Chuyển động thẳng biến đổi đều]{Chuyển động biến đổi -  Gia tốc \\ Chuyển động thẳng biến đổi đều}
\section{Lý thuyết}
\subsection{Vận tốc tức thời. Chuyển động thẳng biến đổi đều}
\vspace*{-1em}
\subsubsection{Vận tốc tức thời}
	Vận tốc tức thời của một vật tại một điểm là đại lượng vector cho ta biết sự nhanh chậm của chuyển động tại điểm đó
	\begin{description}
		\item[Độ lớn của vận tốc tức thời] đo bằng thương số giữa đoạn đường rất nhỏ $\Delta s$ đi qua điểm đó và khoảng thời gian rất ngắn $\Delta t$ để vật đi hết đoạn đường đó 
			\begin{equation*}
				v=\dfrac{\Delta s}{\Delta t}.
			\end{equation*}	
		\item[Vectơ vận tốc tức thời] là một đại lượng vectơ có:
		\begin{itemize}
			\item gốc đặt ở vật chuyển động;
			\item hướng là hướng của chuyển động;
			\item độ dài biểu diễn độ lớn của vận tốc.
		\end{itemize}
	\end{description}
\subsubsection{Chuyển động thẳng biến đổi đều}

Chuyển động thẳng biến đổi đều là chuyển động có 
	\begin{itemize}
		\item quỹ đạo là đường thẳng
		\item độ lớn của vận tốc tức thời tăng đều hoặc giảm đều theo thời gian.
	\end{itemize}  

Chuyển động thẳng biến đổi đều có thể chia thành:
\begin{itemize}
	\item nhanh dần đều: độ lớn của vận tốc tức thời tăng đều theo thời gian;
	\item chậm dần đều: độ lớn của vận tốc tức thời giảm đều theo thời gian.
\end{itemize}
\subsection{Gia tốc, vận tốc, độ dời, quãng đường đi được và phương trình chuyển động trong chuyển động thẳng biến đổi đều}
\vspace{-1em}
\subsubsection{Gia tốc trong chuyển động thẳng biến đổi đều}
\begin{enumerate}[label=\alph*)]
	\item Khái niệm gia tốc
	
	Gia tốc là đại lượng đặc trưng cho sự biến thiên nhanh hay chậm của vận tốc và được đo bằng thương số giữa độ biến thiên vận tốc $\Delta v$ và khoảng thời gian vận tốc biến thiên $\Delta t$,
	
	\begin{equation*}
		a=\dfrac{\Delta v}{\Delta t}.
	\end{equation*}	
	\item Vectơ gia tốc
	
	Vì vận tốc là đại lượng vectơ nên gia tốc cũng là đại lượng vectơ:
	\begin{equation*}
		\vec{a}=\dfrac{\Delta\vec{v}}{\Delta t}=\dfrac{\vec{v}-\vec{v}_0}{t-t_0}.
	\end{equation*}
	Vectơ gia tốc là một đại lượng vectơ có:
	\begin{itemize}
		\item gốc đặt ở vật chuyển động;
		\item hướng trùng với hướng vectơ vận tốc nếu chuyển động nhanh dần đều $(a\cdot v$ $\geq$ $0)$, ngược hướng vectơ vận tốc nếu chuyển động chậm dần đều $(a\cdot v$ $\leq$ $ 0)$;
		\item độ dài tỉ lệ với độ lớn của gia tốc.
	\end{itemize}
\end{enumerate}	
\subsubsection{Vận tốc trong chuyển động thẳng biến đổi đều}
\begin{enumerate}[label=\alph*)]
	\item Công thức tính vận tốc
	
%	Một vật có vận tốc đầu là $v_0$ chuyển động với gia tốc $a$. Nếu chọn mốc thời gian tại thời điểm $t_0$ thì vận tốc của vật tại thời điểm $t$ là: 
	\begin{equation}
		v=v_0+at,
	\end{equation}
trong đó:
	\begin{itemize}
		\item $v_0$ là vận tốc ban đầu 
		\item $a$ là gia tốc 
	\end{itemize}
	\item Đồ thị vận tốc - thời gian
	
	Đồ thị biểu diễn sự biến thiên của vận tốc tức thời theo thời gian gọi là đồ thị vận tốc - thời gian.
	
	Trong chuyển động thẳng biến đổi đều với gia tốc không đổi, đồ thị vận tốc - thời gian có dạng là một nửa đường thẳng.
\end{enumerate}
\subsubsection{Công thức tính độ dời của vật trong chuyển động thẳng biến đổi đều}
\begin{equation}
	s=x-x_0=v_0t+\dfrac{1}{2}at^2.
\end{equation}
 	trong đó 
 		\begin{itemize}
 			\item $x_0$ là tọa độ của vật tại thời điểm $t=0$.
 			\item $v_0$ là vận tốc của vật tại thời điểm $t=0$.
 		\end{itemize}
\subsubsection{Công thức liên hệ giữa gia tốc, vận tốc và độ dời  trong chuyển động thẳng biến đổi đều}
\begin{equation}
	v^2-v_0^2=2as.
\end{equation}
\luuy{Đôi khi ký hiệu $s$ được sử dụng cho tổng quãng đường dịch chuyển, còn độ dời được ký hiệu là $\Delta x$. Khi đó, công thức trên phải được viết là 
$$v^2-v_0^2=2a\cdot\Delta x$$ }


\subsection{Viết phương trình chuyển động thẳng biến đổi đều}
	\begin{itemize}
	\item Chọn hệ quy chiếu gồm:
	\begin{itemize}
		\item Chiều dương (thường là chiều chuyển động của một vật);
		\item Gốc tọa độ (thường là vị trí xuất phát của một vật);
		\item Mốc thời gian (thường là thời điểm bắt đầu chuyển động của một vật).
	\end{itemize}
	\item Phương trình chuyển động của vật có dạng tổng quát như sau:
	\begin{equation*}
		x=x_0+v(t-t_0)+\dfrac{1}{2}a(t-t_0)^2\qquad\textrm{ với }(t\geq t_0),
	\end{equation*}
	trong đó:
	\begin{itemize}[label=$\circ$]
		\item $x_0$: tọa độ ban đầu của vật tại thời điểm $t_0$;
		\item $x$: tọa độ của vật tại thời điểm $t$;
		\item $v$: vận tốc của vật ($v>0$ nếu vật chuyển động cùng chiều dương, $v<0$ nếu vật chuyển động ngược chiều dương);
		\item $a$: gia tốc của vật ($a\cdot v> 0$ nếu vật chuyển động nhanh dần đều, $a\cdot v< 0$ nếu vật chuyển động chậm dần đều).
	\end{itemize}
	\item Hai vật gặp nhau khi chúng có cùng tọa độ:
	\begin{equation*}
		x_1=x_2.
	\end{equation*}
	\item Khoảng cách giữa hai vật tại thời điểm $t$ bất kì là:
	\begin{equation*}
		d=\left|x_1-x_2\right|.
	\end{equation*}
\end{itemize}	
\section{Mục tiêu bài học - Ví dụ minh họa}
\begin{dang}{Nhận biết được đặc điểm \\của chuyển động thẳng biến đổi đều}
	\viduii{1}{Chọn câu sai: 
		
		Khi một chất điểm chuyển động thẳng biến đổi đều thì nó
		\begin{mcq}
			\item có gia tốc trung bình không đổi.
			\item có gia tốc không đổi.
			\item chỉ có thể chuyển động nhanh dần hoặc chậm dần đều.
			\item có thể lúc đầu chuyển động chậm dần đều, sau đó nhanh dần đều.
		\end{mcq}
	}
	{	\begin{center}
			\textbf{Hướng dẫn giải}
		\end{center}
		
		Khi một chất điểm chuyển động thẳng biến đổi đều thì nó có thể lúc đầu chuyển động chậm dần đều, sau đó nhanh dần đều. 
		
		\textbf{Đáp án: C}.
	}
	\viduii{1}{Chuyển động thẳng biến đổi đều là chuyển động
		\begin{mcq}
			\item có quỹ đạo là đường thẳng, vectơ gia tốc bằng không.
			\item có quỹ đạo là đường thẳng, vectơ gia tốc không thay đổi trong suốt quá trình chuyển động.
			\item có quỹ đạo là đường thẳng, vectơ gia tốc và vận tốc không thay đổi trong suốt quá trình chuyển động.
			\item có quỹ đạo là đường thẳng, vectơ vận tốc không thay đổi trong suốt quá trình chuyển động.
		\end{mcq}
	}
	{	\begin{center}
			\textbf{Hướng dẫn giải}
		\end{center}
		
		Chuyển động thẳng biến đổi đều là chuyển động có quỹ đạo là đường thẳng, vectơ gia tốc không thay đổi trong suốt quá trình chuyển động.
		
		\textbf{Đáp án: B}.
	}
\end{dang}

\begin{dang}{Thực hiện xác định \\quãng đường, vận tốc, gia tốc, thời gian \\thông qua phương trình chuyển động }
	\viduii{2}{Cho phương trình chuyển động của một chất điểm dọc theo trục $Ox$ có dạng $x = 10 + 4t - \text{0,5}t^2$ (đơn vị của $x$ và $t$ lần lượt là mét và giây). Vận tốc của chuyển động sau $\SI{2}{s}$ là bao nhiêu?
	}
	{	\begin{center}
			\textbf{Hướng dẫn giải}
		\end{center}
		
		Đối chiếu phương trình đã cho
		$$x = 10 + 4t - \text{0,5}t^2.$$
		với phương trình chuyển động
		$$x=x_0+v_0t+\dfrac{1}{2}at^{2}$$
		ta suy ra $$v_0 = \SI{4}{\meter/\second};\qquad a =\SI{-1}{\meter/\second^2}.$$
		
		Thay các giá trị vào phương trình vận tốc, ta suy ra vận tốc của vật ở thời điểm  $t =\SI{2}{\second}$
		$$v=v_0+at=\SI{4}{\meter/\second}+\left(\SI{-1}{\meter/\second^{2}}\right)\cdot\SI{2}{\second}=\SI{2}{\meter/\second}.$$

	}
	\viduii{2}{Phương trình cơ bản của 1 vật chuyển động: $x = 6t^2 - 18t + 12 \ (\text{cm, s})$. Hãy xác định:
		\begin{enumerate}[label=\alph*.]
			\item Vận tốc của vật, gia tốc của chuyển động và cho biết tính chất của chuyển động.
			\item Vận tốc của vật ở thời điểm $t = \SI{2}{\second}$.
		\end{enumerate}
	}
	{	\begin{center}
			\textbf{Hướng dẫn giải}
		\end{center}
		
		\begin{enumerate}[label=\alph*.]
			\item Đối chiếu phương trình 
				$$x = 6t^2 - 18t + 12 \ \text{cm}$$
			với phương trình chuyển động
			$$x=x_0+v_0t+\dfrac{1}{2}at^{2}$$
			ta suy ra
			$$v_0 = \SI{-18}{\centi\meter/\second};\qquad a =\SI{12}{\centi\meter/\second^{2}}.$$
			
			Vật chuyển động chậm dần đều do gia tốc và vận tốc trái dấu với nhau.
			\item Thay các giá trị vận tốc và gia tốc đã tìm được vào phương trình vận tốc, ta suy ra vận tốc của vật ở thời điểm $t=\SI{2}{\second}$
			$$v =v_0 +at =\SI{-18}{\centi\meter/\second}+\SI{12}{\centi\meter/\second^{2}}\cdot\SI{2}{\second}=\SI{6}{\centi\meter/\second}.$$
		\end{enumerate}
	}
	
	\viduii{3}{Một vật chuyển động thẳng có phương trình: $x = 4t^2 + 20t\ (\text{m})$. Tính quãng đường vật đi được từ thời điểm $t_1 = \SI{2}{s}$ đến thời điểm $t_2 = \SI{5}{s}$.
	}
	{	\begin{center}
			\textbf{Hướng dẫn giải}
		\end{center}
		
		Vị trí của vật ở thời điểm  $\SI{2}{\second}$  
		
		$$x_1 = 4t_1^2 + 20t_1 =\SI{56}{\meter}.$$
		
		Vị trí của vật ở thời điểm  $\SI{5}{\second}$ 
		
		$$x_2 = 4t_2^2 + 20t_2 =\SI{200}{\meter}.$$
		
		Quãng đường vật đi được trong thời gian từ $\SI{2}{\second}$  đến $\SI{5}{\second}$ là:
		$$\Delta x = x_2 - x_1 = \SI{144}{\meter}.$$
		
	}
	\viduii{3}{Vật chuyển động thẳng có phương trình: $x = 2t^2 - 4t + 10$ (đơn vị của $x$ và $t$ lần lượt là mét và giây). Vật sẽ dừng lại tại vị trí
		\begin{mcq}(4)
			\item $\SI{6}{m}.$
			\item $\SI{4}{m}.$
			\item $\SI{10}{m}.$
			\item $\SI{8}{m}.$
		\end{mcq}
	}
	{	\begin{center}
			\textbf{Hướng dẫn giải}
		\end{center}
		
		Vật sẽ dừng lại khi vận tốc $v = 0$.
		
		Từ phương trình chuyển động ta suy ra các giá trị vận tốc ban đầu và gia tốc 
			$$v_0=\SI{-4}{\meter/\second},	\qquad a=\SI{4}{\meter/\second^{2}}.$$
		
		Sử dụng phương trình vận tốc, ta suy ra thời điểm vật dừng lại
			\begin{align*}
				v=v_0+at=0 \quad\Rightarrow\quad t=-\dfrac{v_0}{a}=-\dfrac{\SI{-4}{\meter/	\second}}{\SI{4}{\meter/\second^{2}}}=\SI{1}{\second}.
			\end{align*}		
		Thay $t =\SI{1}{\second}$ vào phương trình chuyển động ta được vị trí dừng lại của vật 
		$$x = 2t^2 - 4t + 10=\SI{8}{\meter}.$$
		
		\textbf{Đáp án: D}.
	}
\end{dang}

\begin{dang}{Áp dụng được công thức liên hệ \\giữa độ dời, vận tốc, gia tốc }
	\viduii{2}{Một xe máy đang đi với $v = \SI{50,4}{km/h}$ bỗng người lái xe thấy có ổ gà trước mắt cách xe $\SI{24,5}{m}$. Người ấy phanh gấp và xe đến ổ gà thì dừng lại.
		\begin{enumerate}[label=\alph*.]
			\item Tính gia tốc
			\item Tính thời gian phanh.
		\end{enumerate}
	}
	{	\begin{center}
			\textbf{Hướng dẫn giải}
		\end{center}
		
		Đơn vị vận tốc được đổi về hệ SI $$\SI{50,4}{\kilo\meter/\hour} =\dfrac{\SI{50.4e3}{\meter}}{\SI{3600}{\second}}= \SI{14}{m/s}.$$
		\begin{enumerate}[label=\alph*.]
			\item Gia tốc của xe máy
			$$v^2-v_0^2 =2as \Rightarrow a =\dfrac{v^2 - v_0^2}{2s}=\dfrac{\left(\SI{0}{\meter/\second}\right)^{2}-\left(\SI{14}{\meter/\second}\right)^{2}}{2\cdot\SI{24.5}{\meter}}=\SI{-4}{m/s}^2.$$
			\item Thời gian phanh
			
			$$v =v_0 +at \Rightarrow t = \dfrac{v-v_0}{a} =\dfrac{\SI{0}{\meter/\second}-\SI{14}{\meter/\second}}{\SI{-4}{\meter/\second^{2}}}= \SI{3,5}{s}.$$
		\end{enumerate}
			}
	\viduii{2}{Một đoàn tàu bắt đầu chuyển động nhanh dần đều khi đi hết $\SI{1}{km}$ đầu tiên thì đạt vận tốc $v=\SI{10}{m/s}$. Tính vận tốc đoàn tàu sau khi đi hết $\SI{2}{km}$.
	}
	{	\begin{center}
			\textbf{Hướng dẫn giải}
		\end{center}
		
		Gia tốc của đoàn tàu
		$$v^2-v_0^2 =2as_1 \Rightarrow a = \dfrac{v^2-v_0^2}{2s_1}=\dfrac{\left(\SI{10}{\meter/\second}\right)^2-(\SI{0}{\meter/\second})^2}{2\cdot\SI{1000}{\meter}} =\SI{0,05}{m/s}^2.$$
		
		Vận tốc sau khi đi hết $\SI{2}{km}$
		$$v_1^2 - v_0^2 = 2as_2 \Rightarrow  v_1 =\sqrt{2as_2+v_0^2}=\sqrt{2\cdot\SI{0.05}{\meter/\second^{2}}\cdot\SI{2}{\kilo\meter}+\left(\SI{0}{\kilo\meter/\second^{2}}\right)^{2}} =\xsi{10\sqrt{2}}{\meter/\second}.$$
	}
\end{dang}

\begin{dang}{Xây dựng phương trình\\ chuyển động thẳng biến đổi đều}
	\viduii{2}{Một vật chuyển động thẳng chậm dần đều với tốc độ ban đầu $\SI{3}{\meter/\second}$ và gia tốc có độ lớn $\SI{2}{\meter/\second^2}$. Biết thời điểm ban đầu vật ở gốc tọa độ và chuyển động ngược chiều dương của trục tọa độ. Viết phương trình chuyển động của vật.
	}
	{	\begin{center}
			\textbf{Hướng dẫn giải}
		\end{center}
		
		Chọn gốc thời gian là khi vật bắt đầu chuyển động.
		
		Vì vật chuyển động chậm dần đều ngược chiều dương nên
		\begin{equation*}
			\left\{
			\begin{array}{rcl}
				a\cdot v &<&0\\
				v &<& 0
			\end{array}
			\right.
			\quad
			\Rightarrow 
			\left\{
			\begin{array}{rcl}
				a &>& 0\\
				v &<& 0.
			\end{array}
			\right.
		\end{equation*}
		
		Kết hợp với các dữ kiện của đề bài, ta suy ra
		\begin{equation*}
			\left\{
			\begin{array}{rcr}
				a&=&\SI{2}{\meter/\second^2}\\
				v&=&\SI{-3}{\meter/\second} .
			\end{array}
			\right.
		\end{equation*}
		Do đó, phương trình chuyển động của vật có dạng
		$x=-3t+t^2$ (m, s).
	}
	\viduii{3}{Một đoạn dốc thẳng dài $\SI{62,5}{m}$, Nam đi xe đạp và khởi hành từ chân dốc đi lên với $v_0 =\SI{18}{km/h}$ chuyển động chậm dần đều với gia tốc có độ lớn $\SI{0,2}{m/s}^2$.
		\begin{enumerate}[label=\alph*.]
			\item Viết phương trình chuyển động của Nam.
			\item Nam đi hết đoạn dốc trong bao lâu?
		\end{enumerate}
	}
	{	\begin{center}
			\textbf{Hướng dẫn giải}
		\end{center}
		
		Đổi đơn vị $$\SI{18}{km/h} = \dfrac{\SI{18e3}{\meter}}{\SI{3600}{\second}}=\SI{5}{m/s}.$$
		
		Chọn gốc toạ độ tại chân dốc, chiều dương từ chân đến đỉnh dốc, gốc thời gian là khi Nam bắt đầu lên dốc.
		\begin{enumerate}[label=\alph*.]
			\item Khi nam lên dốc, Nam đi theo chiều dương nên $v>0$.
			
			Chuyển động chậm dần đều: 
			$$a\cdot v<0 \Rightarrow a<0.$$
			
			Phương trình chuyển động:
			$$x =x_0 +v_0t+\dfrac{1}{2}at^2 = 5t - \text{0,1}t^2.$$
			\item Thời gian đi hết đoạn dốc
			$$\text{62,5} =5t - \text{0,1}t^2 \Rightarrow t = \SI{25}{s}.$$
		\end{enumerate}
	}
\end{dang}

\begin{dang}{Thực hiện xác định vị trí, thời điểm hai vật chuyển động thẳng biến đổi đều gặp nhau}
	\viduii{3}{Một xe ô tô bắt đầu chuyển động thẳng nhanh dần đều với gia tốc $\SI{0,5}{\meter/\second^{2}}$ đúng lúc một xe máy chuyển động thẳng đều với vận tốc $\SI{36}{\kilo\meter/\hour}$ vượt qua nó.
		Xác định thời điểm và vị trí hai xe gặp nhau lần nữa và vận tốc xe ô tô khi đó?
		Xác định thời điểm để hai xe cách nhau một quãng đường là $\SI{100}{\meter}$.
		
	}
	{	\begin{center}
			\textbf{Hướng dẫn giải}
		\end{center}
	Chọn chiều dương là chiều chuyển động của ô tô, gốc tọa độ tại vị trí xuất phát, gốc thời gian là lúc xe ô tô khởi hành.
		 	
		 	Xe ô tô có các thông số chuyển động 
		 	$$x_{10} = \SI{0}{\meter};\qquad  v_{10} = \SI{0}{\meter/\second};\qquad a_1 = \SI{0.5}{\meter
		 	/\second^{2}}$$	
		 	nên có phương trình chuyển động 
		 	$$x_1 = x_{10}+v_{10}t+\dfrac{1}{2}a_{1}t^{2}=\dfrac{1}{2}a_1t^2.$$
		 	
		 	Xe máy có các thông số chuyển động 
		 	$$x_{20} = 0;\qquad v_{20} =\SI{36}{\kilo\meter/\hour}=\SI{10}{\meter/\second};\qquad a_{2} = \SI{0}{\meter/\second^{2}}$$	
		 	nên có phương trình chuyển động 
		 	$$x_2 =x_{20}+v_{20}t+\dfrac{1}{2}a_2t^2=v_{20}t.$$
		 	
		 	Tọa độ hai xe bằng nhau khi hai xe gặp nhau
		 	\begin{align*}
		 		x_1&=x_2\\
		 		\quad\Rightarrow\quad \dfrac{1}{2}a_1t^2&=v_{20}t\\
		 		\quad\Rightarrow\quad t&=\SI{0}{\second} \quad\vee\quad t=\dfrac{2v_{20}}{a_1}=\dfrac{2\cdot\SI{10}{\meter/\second}}{\SI{0.5}{\meter/\second^{2}}}=\SI{40}{\second}
		 	\end{align*} 
		 	trong đó nghiệm $t=0$ ứng với thời điểm hai xe gặp nhau lúc đầu, còn nghiệm $t=\SI{40}{\second}$ là nghiệm ta cần tìm. 
		 	
		 	Vị trí 2 xe gặp nhau 
		 	$$x_1=x_2=v_{20}t =\SI{10}{\meter/\second}\cdot\SI{40}{\second}= \SI{400}{m}.$$
		 	 
		 	Vận tốc ô tô khi đó 
		 	$$v_1 = v_{10}+ a_1t = \SI{0}{\meter/\second}+\SI{0.5}{\meter/\second^{2}}\cdot\SI{40}{\second}=\SI{20}{m/s}.$$  
		
	
	}
	\viduii{3}{
		Trong một chuyến từ thiện của trung tâm A thì mọi người dừng lại bên đường uống nước. Sau đó ngay thời điểm ô tô bắt đầu chuyển động nhanh dần đều với gia tốc $\SI{0,5}{m/s}^2$ thì có một xe khách vượt qua xe với vận tốc $\SI{18}{km/h}$ và gia tốc $\SI{0,3}{m/s}^2$. Hỏi ô tô đuổi kịp xe khách sau khi đi quãng đường bao xa, và tính vận tốc của ô tô lúc đó.
	}
	{	\begin{center}
			\textbf{Hướng dẫn giải}
		\end{center}
		
		Chọn chiều dương là chiều chuyển động của ô tô, gốc tọa độ tại vị trí uống nước, gốc thời gian là lúc xe ô tô khởi hành.
		
		Từ các thông số chuyển động của ô tô
		$$x_{10} = 0,\quad v_{10} = 0,\quad a_{1} = \SI{0,5}{m/s}^2,$$	
		ta suy ra phương trình chuyển động của ô tô
		$$x_1 = \text{0,25}t^2.$$
		
		Tương tự, phương trình chuyển động của xe khách cũng được suy ra từ các thông số chuyển động của xe khách
			\begin{equation*}
				\begin{gathered}
					x_{20} = 0,\quad v_{20} =\SI{18}{km/h}=\SI{5}{m/s},\quad a_{2} = \SI{0,3}{m/s}^2\\
					\Rightarrow\quad x_2 =5t+\text{0,15}t^2.
				\end{gathered}
			\end{equation*}
		
		Thời điểm hai xe gặp nhau được xác định từ phương trình 
		$$x_1=x_2 \quad\Rightarrow\quad \text{0,25}t^2=5t+\text{0,15}t^2 \quad\Rightarrow\quad t=\SI{0}{\second}\quad\vee\quad t =\SI{50}{s}.$$
		Ta chọn nghiệm $t =\SI{50}{s}$ là thời điểm gặp nhau sau khi ô tô đã xuất phát. 
		
		Vận tốc của ô tô khi đó
		$$v_1 = v_{10}+ a_1t = \SI{0}{\meter/\second}+\SI{0.5}{\meter/\second^{2}}\cdot\SI{50}{\second}=\SI{25}{m/s}.$$  
		
		Quãng đường ô tô đã đi được cho đến khi gặp nhau 
		$$s=x-x_0 =\text{0,25}t^2 = \SI{625}{\meter}.$$	
	}
\end{dang}
\begin{dang}{Thực hiện xác định vận tốc, khoảng cách giữa hai vật chuyển động thẳng biến đổi đều}
	\viduii{3}{Phương trình cơ bản của một vật chuyển động: $x=6t^2-18t+12$ (cm, s). Hãy xác định:	
		\begin{enumerate}[label=\alph*)]
			\item Vận tốc ban đầu của vật, gia tốc của chuyển động và cho biết tính chất của chuyển động.
			\item Vận tốc của vật ở thời điểm $t=\SI{2}{\second}$.
		\end{enumerate}
	}
	{	\begin{center}
			\textbf{Hướng dẫn giải}
		\end{center}
		
		\begin{enumerate}[label=\alph*)]
			\item
			Phương trình chuyển động của vật: $x=6t^2-18t+12$ (cm, s).
			
			Từ phương trình chuyển động ta suy ra được vận tốc ban đầu và gia tốc của vật là:
			\begin{equation*}
				\left\{\begin{array}{ll}{v_0=\SI{-18}{cm/\second}}&\\{a=\SI{12}{cm/\second^2}.}&\end{array}\right.
			\end{equation*}
			
			Vì $a\cdot v_0 < 0$ nên vật chuyển động chậm dần đều.
			\item 
			Vận tốc của vật ở thời điểm $t=\SI{2}{\second}$ là:
			\begin{equation*}
				v=v_0+at=\SI{6}{cm/\second}.
			\end{equation*}
		\end{enumerate}
	}
	\viduii{3}{Một xe ô tô bắt đầu chuyển động thẳng nhanh dần đều với gia tốc $\SI{0,5}{m/s}^2$ đúng lúc một xe máy chuyển động thẳng đều với vận tốc $\SI{36}{km/h}$ vượt qua nó. Xác định thời điểm để hai xe cách nhau một quãng đường là $\SI{100}{m}$.
	}
	{	\begin{center}
			\textbf{Hướng dẫn giải}
		\end{center}
		
		Chọn chiều dương là chiều chuyển động của ô tô, gốc tọa độ tại vị trí xuất phát, gốc thời gian là lúc xe ô tô khởi hành.
		
		Xe ô tô có các thông số chuyển động 
		$$x_{10} = \SI{0}{\meter};\qquad  v_{10} = \SI{0}{\meter/\second};\qquad a_1 = \SI{0.5}{\meter
			/\second^{2}}$$	
		nên có phương trình chuyển động 
		$$x_1 = x_{10}+v_{10}t+\dfrac{1}{2}a_{1}t^{2}=\dfrac{1}{2}a_1t^2=\SI{0.25}{}t^{2}.$$
		
		Xe máy có các thông số chuyển động 
		$$x_{20} = 0;\qquad v_{20} =\SI{36}{\kilo\meter/\hour}=\SI{10}{\meter/\second};\qquad a_{2} = \SI{0}{\meter/\second^{2}}$$	
		nên có phương trình chuyển động 
		$$x_2 =x_{20}+v_{20}t+\dfrac{1}{2}a_2t^2=v_{20}t=\SI{10}{}t.$$
		
		Để 2 xe cách nhau $\SI{40}{m}$ thì 
		$$|x_1-x_2| = 100.$$
		$$\Rightarrow \left[\begin{array}{ll}{x_1-x_2 =100}&\\{x_2-x_1 =100.}&\end{array}\right.$$
		
		$$\Rightarrow 
		\left[\begin{array}{ll}{\text{0,25}t^2 -10t=100 \Rightarrow t\approx\SI{48,28}{s}}&\\{10t-\text{0,25}t^2=100\Rightarrow t =\SI{20}{s}.}&\end{array}\right.$$
		
		\luuy{Đôi khi các phương trình cho ta nhiều nghiệm $t$, ta cần phân tích ý nghĩa của nghiệm và lựa chọn nghiệm phù hợp với thời điểm ta quan tâm. 
			
			Chẳng hạn trong  bài toán này, các phương trình cho ba nghiệm: $t_1\approx\SI{-8.28}{\second}, t_2=\SI{20}{\second},t_3\approx\SI{48.28}{\second}$. Nghiệm $t_1$ tương ứng với thời điểm trước khi xe hai xe gặp nhau lần đầu, lúc đó xe máy đang ở phía sau của ô tô và chuẩn bị vượt qua ô tô. Nghiệm $t_2$ ứng với thời điểm ô tô đang đuổi theo xe máy, và còn cách xe máy \SI{100}{\meter}. Nghiệm $t_3$ ứng với thời điểm ô tô đã vượt qua xe máy và đã bỏ xa xe máy \SI{100}{\meter}. Do đề bài chỉ cho ta biết về chuyển động hai xe kể từ thời điểm xe máy vượt qua ô tô, nên ta chỉ quan tâm các nghiệm $t>0$. 
		}
	}
\end{dang}
\begin{dang}{Thực hiện xác định vị trí, thời điểm gặp nhau giữa vật chuyển động thẳng đều và vật chuyển động thẳng biến đổi đều}
	\viduii{3}{Một ôtô đang chuyển động thẳng đều với vận tốc $\SI{45}{\kilo\meter/\hour}$ bỗng tăng ga chuyển động nhanh dần đều.
		\begin{enumerate}[label=\alph*.]
			\item Tính gia tốc của xe biết rằng sau $\SI{30}{\second}$ ô tô đạt vận tốc $\SI{72}{\kilo\meter/\hour}$.
			\item Trong quá trình tăng tốc nói trên, vào thời điểm nào kể từ lúc tăng tốc, vận tốc của xe là $\SI{64.8}{\kilo\meter/\hour}$?
		\end{enumerate}
	}
	{	\begin{center}
			\textbf{Hướng dẫn giải}
		\end{center}
		
		Đổi đơn vị $\SI{45}{km/h} = \SI{12,5}{m/s};\ \SI{72}{km/h} = \SI{20}{m/s}$.
		
		Chọn chiều dương là chiều chuyển động.
		
		a. Gia tốc của xe là
		$$a = \dfrac{v-v_0}{t} =\dfrac{\SI{20}{\meter/\second}-\SI{12.5}{\meter/\second}}{\SI{30}{\second}}= \SI{0,25}{m/s}^2.$$
		
		b. Đổi đơn vị $\SI{64,8}{km/h} = \SI{18}{m/s}$.
		
		Xe đạt vận tốc $\SI{64,8}{km/h}$ vào thời điểm 
		$$t= \dfrac{v'-v_0}{a}=\dfrac{\SI{18}{\meter/\second}-\SI{12.5}{\meter/\second}}{\SI{0.25}{\meter/\second^{2}}} = \SI{22}{s}.$$
		
	}
	\viduii{3}{Cùng một lúc, từ hai địa điểm A và B cách nhau $\SI{50}{m}$ có hai vật chuyển động ngược chiều để gặp nhau. Vật thứ nhất xuất phát từ A chuyển động đều với vận tốc $\SI{5}{m/s}$, vật thứ hai xuất phát từ B chuyển động nhanh dần đều không vận tốc đầu với gia tốc $\SI{2}{m/s}^2$. Chọn trục $Ox$ trùng đường thẳng AB, gốc tọa độ tại A, chiều dương từ A đến B, gốc thời gian là lúc xuất phát
		\begin{enumerate}[label=\alph*.]
			\item Viết phương trình chuyển động của mỗi vật.
			\item Xác định thời điểm và vị trí hai xặp nhau.
			\item Xác định thời điểm mà tại đó hai vật có độ lớn vận tốc bằng nhau.
		\end{enumerate}
	}
	{	\begin{center}
			\textbf{Hướng dẫn giải}
		\end{center}
		Chọn chiều dương là chiều chuyển động của xe A, gốc tọa độ ở A, gốc thời gian từ lúc xe hai xe bắt đầu chuyển động. 
		\begin{enumerate}[label=\alph*.]
			\item Phương trình chuyển động của 2 xe lần lượt là:
			
			$$x_\text{1} = x_{10}+v_{1}t = v_{1}t.$$
			
			$$x_\text{2} = x_{20} + v_{20}t +\dfrac{1}{2}a_2t^2 = x_{20} +\dfrac{1}{2}a_2 t^2.$$
			\item Hai xe gặp nhau thì:
			
			$$x_{1} = x_{2}.$$
			$$\Leftrightarrow  v_{10}t= x_{20} +\dfrac{1}{2}a_2 t^2.$$
			Thay các giá trị số 
				$$v_{1}=\SI{5}{\meter/\second},\quad x_{20}=\SI{50}{\meter},\quad a_2=\SI{-2}{\meter/\second^{2}},$$
			và giải phương trình trên, ta thu được hai nghiệm $t=\SI{0}{\second}$ và $t =\SI{5}{s}.$ Ta chọn nghiệm $t =\SI{5}{s}$ do ta chỉ xét thời gian sau khi hai xe đã chuyển động. 
			
			Vị trí gặp nhau của hai xe
				$$x_1\simeq x_2=v_{1}t=\SI{5}{\meter/\second}\cdot\SI{5}{\second}=\SI{25}{\meter}.$$
			 Vậy hai xe gặp nhau sau $\SI{5}{s}$ và cách A $\SI{25}{m}.$
			 \item Phương trình vận tốc của xe thứ hai  là
			 
			 $$v_{2} = v_{20} +a_2t=a_2t$$
			 
			 Hai xe có cùng độ lớn vận tốc
			 $$\left|v_1\right|=\left|v_2\right|\quad\Rightarrow\quad v_1=a_2t\quad\Rightarrow\quad t=\left|\dfrac{v_1}{a_2}\right|=\left|\dfrac{\SI{5}{\meter/\second}}{\SI{-2}{\meter/\second^{2}}}\right|=\SI{2.5}{\second}.$$
		\end{enumerate}
	}
	\viduii{3}{Hai vật cùng xuất phát một lúc tại A, chuyển động cùng chiều. Vật thứ nhất chuyển động đều với vận tốc $v_1 = \SI{20}{m/s}$, vật thứ hai chuyển động thẳng nhanh dần đều với vận tốc ban đầu bằng không và gia tốc $\SI{0,4}{m/s}^2$. Chọn chiều dương là chiều chuyển động, gốc tọa độ O tại A, gốc thời gian là lúc xuất phát.
		
		a.  Xác định thời điểm và vị trí hai xe gặp nhau.
		
		b.  Viết phương trình vận tốc của vật thứ hai. Xác định khoảng cách giữa hai vật tại thời điểm chúng có vận tốc bằng nhau.
		
	}
	{	\begin{center}
			\textbf{Hướng dẫn giải}
		\end{center}
		
		a. Phương trình chuyển động:
		$$x_1 = 20t.$$
		$$x_2 = \text{0,2}t^2.$$
		
		Khi hai vật gặp nhau thì:
		
		$$x_1 =x_2 \quad\Rightarrow\quad 20\;t = \text{0,2}\;t^2\quad\Rightarrow\quad t =\SI{0}{\second}\quad\vee\quad t =\SI{100}{s}.$$
		Ta chọn nghiệm $t =\SI{100}{s}$ ứng với thời điểm hai xe gặp lại nhau. 
		
		Vị trí hai xe gặp lại nhau 
			$$x_1 =x_2 =20t=\SI{2000}{m}.$$
		
		b. Phương trình vận tốc của vật thứ hai
		$$v_2 =v_{20}+a_2t=\text{0,4}\;t\ \text{m/s}.$$
		
		Thời điểm lúc hai vật có vận tốc bằng nhau: 
		$$v_2 =\text{0,4}\;t = 20 \quad\Rightarrow\quad t = \SI{50}{s}.$$
		
		Tọa độ các vật lúc đó: 
		
		$$x_1 = 20\;t = \SI{1000}{m};\quad x_2 = \text{0,2}\;t^2 = \SI{500}{m}.$$
		
		Khoảng cách giữa hai vật: 
		
		$$d= \left|x_1 - x_2\right| =\SI{500}{m}.$$
	}
	
\end{dang}


% testing git