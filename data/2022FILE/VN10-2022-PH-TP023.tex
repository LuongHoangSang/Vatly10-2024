\setcounter{section}{0}
\begin{enumerate}[label=\bfseries Câu \arabic*:]
	\item \mkstar{2}
	
	{
		
		Các nhà triết học tự nhiên Hy Lạp cổ đại sử dụng những phương pháp nghiên cứu nào để nghiên cứu thế giới tự nhiên?
		
	}
	
	\hideall{
		
		Các nhà Triết học tự nhiên Hy Lạp cổ đại tìm cách xây dựng một bức tranh khoa học tổng quát về thới giới từ những cảm nhận bằng mắt, từ những dữ kiện đơn lẻ, cụ thể để khái quát tính chất chung của thế giới tự nhiên.
	}
	
	\item \mkstar{2}
	
	
	{
		Phương pháp thực nghiệm có vai trò như thế nào đối với quá trình phát triển của Vật lí học và cuộc cách mạng công nghiệp?
		
	}
	
	\hideall
	{
		Bằng việc sử dụng các phương pháp thực nghiệm để phát hiện ra các quy luật, các định luật vật lí. Vật lí đã trở thành ngành khoa học riêng.
		
		Thế kỉ thứ 17, có nhiều bước tiến đáng kể về nhiệt học, nhiều nghiên cứu về hiện tượng dẫn nhiệt, bức xạ nhiệt, dãn nở vì nhiệt, bức xạ nhiệt, dãn nở vì nhiệt là cơ sở để sáng chế ra máy hơi nước, mở đầu cách mạng công nghiệp lần thứ nhất.
		
		Tìm ra định luật cảm ứng điện từ, là cơ sở sáng chế ra máy điện và động cơ điện, mở đầu cách mạng công nghiệp lần thứ hai.
		
		
	}
	\item \mkstar{2}
	
	
	{
		Hãy nêu vai trò của cơ học Newton đối với sự phát triển của Vật lí học.
	}
	
	\hideall
	{
		Bằng phương pháp thực nghiệm, Newton đã phát minh ra các định luật về chuyển động cơ học. Các định luật này không chỉ đặt nền móng cho cơ cổ điển nghiên cứu các chuyển động xung quanh chúng ta mà còn giúp các nhà vật lí mở rộng các nghiên cứu về thủy động lực học, điện học, từ học. Đầu thế kỉ 19, các nhà khoa học đã nghiên cứu lực điện, lực từ giống như Newton nghiên cứu lực hấp dẫn, nhờ đó họ đã chế tạo được nam châm điện, động cơ, máy phát điện...
		
	}
	\item \mkstar{2}
	
	
	{
		Tầm quan trọng của thuyết tương đối.
	}
	
	\hideall
	{
		Năm 1905 Einstein phát minh ra thuyết tương đối hẹp, mô tả không gian - thời gian. Hệ thức này mở đường cho nghiên cứu năng lượng nguyên tử và hạt nhân.
		
		Năm 1916, Einstein đưa ra thuyết tương đối rộng, quan niệm trường hấp dẫn đặc trưng bởi độ cong của không - thời gian phụ thuộc vào sự phân bố khối lượng. Thuyết tương đối rộng đã giải thích được nhiều hiện tượng trong vũ trụ và mở đường cho vật lí thiên văn hiện đại.
		
		Lí thuyết lượng tử và thuyết tương đối tác động mạnh mẽ đến mọi lĩnh vực nghiên cứu của vật lí và sáng chế các thiết bị kĩ thuật như laser, máy tính hoặc GPS.
	}
	\item \mkstar{2}
	
	
	{
		Vì sao âm học được coi là một nhánh của cơ học?
	}
	
	\hideall
	{
		Âm học nghiên cứu âm thanh, cũng được coi là một nhánh của cơ học bởi vì âm thanh là do chuyển động của các hạt hay phân tử trong không khí hoặc môi trường khác gây ra sóng âm. Một trong những nhánh quan trọng của âm học là siêu âm học, nghiên cứu sóng siêu âm với tần số cao hơn tần số mà con người nghe được.
	}
	\item \mkstar{2}
	
	{
		
		Hãy kể ra một số thành tựu khác của vật lí thực nghiệm.
	}
	
	\hideall{
		
		- Thế kỉ 19, sáng chế ra máy hơi nước, mở đầu cách mạng công nghiệp lần 1.
		
		- Những phát minh quan trong của vật lí thực nghiệm liên quan đến lĩnh vực điện như chế tạo ra pin (Galvani và Davy), cho phép các nhà khoa học nghiên cứu định lượng về tác dụng và bản chất của dòng điện.
		
		- Những thí nghiệm của Orsted và Ampere nghiên cứu về bản chất của hiện tượng điện từ. Năm 19831, Faraday đã tìm ra định luật cảm ứng điện từ, là cơ sở sáng chế ra máy phát điện và động cơ điện, mở đầu cách mạng công nghiệp lần 2.
		
		- Newton đưa ra thuyết tán sắc ánh sáng và lí thuyết hạt của ánh sáng.
		
		- Galilei chế tạo ra kính thiên văn học mở đầu cho thiên văn học khám phá vũ trụ, giúp quang hình học phát triển mạnh mẽ.
		
		- Huygens đưa ra lí thuyết về bản chất sóng của ánh sáng.
		
		- Grimaldi đã phát hiện ra hiện tượng giao thoa, nhiễu xạ.
		
		- Maxwell làm sáng tỏ bản chất sóng của ánh sáng bằng cách đưa ra hệ phương trình mô tả trường điện từ, làm cho điện từ học thống nhất với quang học.
		
		- Cuối thế kỉ 20, Poopov phát minh ra phương pháp truyền sóng vô tuyến, qua đó xây dựng cơ sở ngành vô tuyến điện.
		
	}
	
	\item \mkstar{2}
	
	
	{
		Tìm hiểu và trình bày đóng góp quan trọng của Newton trong các lĩnh vực nghiên cứu khác.
		
	}
	
	\hideall
	{
		Nhờ hệ thống cơ học Newton, nhà khoa học Halley đã dự đoán được sự xuất hiện của một sao chổi. Đây được xem như bằng chứng thực nghiệm đầu tiên khẳng định sự đúng đắn của cơ học Newton và là một trong những thành tựu vĩ đại nhất trong lịch sử nhân loại.
		
		Newton đã phát minh công cụ tóa học có tên gọi là tính vi phân và tích phân để phục vụ cho những nghiên cứu về chuyển động của mình.
		
		Cơ học Newton đặt nền móng cho Vật lí cổ điển, đóng vai trò thúc đẩy những lí thuyết chính xác hơn và giữ được giá trị to lớn trong lĩnh vực khoa học ngày nay.
	}
	\item \mkstar{2}
	
	
	{
		Theo em, tốc độ truyền ánh sáng có phụ thuộc vào tốc độ của nguồn sáng hay không? Tại sao?
	}
	
	\hideall
	{
		Khi đo tốc độ ánh sáng, Michelson và Morley phát hiện ra nó luôn là hằng số, không phụ thuộc vào tốc độ của nguồn tức là không tuân theo công thức cộng vận tốc Galilei đã được tin tưởng suốt ba thế kỉ. 
	}
	\item \mkstar{2}
	
	
	{
		Trình bày hiểu biết của em về vật đen tuyệt đối. 
	}
	
	\hideall
	{
		Trong vật lý học, vật đen tuyệt đối, hay ngắn gọn là vật đen, là vật hấp thụ hoàn toàn tất cả các bức xạ điện từ chiếu đến nó, bất kể bước sóng nào. Điều này có nghĩa là sẽ không có hiện tượng phản xạ hay tán xạ trên vật đó, cũng như không có dòng bức xạ điện từ nào đi xuyên qua vật.
		
		
	}
	\item \mkstar{2}
	
	
	{
		Những vấn đề gì gây nên sự khủng hoảng của vật lý cuối thế kỉ XIX? Vì sao nói sự khủng hoảng đó là tiền đề của vật lý hiện đại?
	}
	
	\hideall
	{
		Maxwell đã xây dựng một lí thuyết thống nhất về các hiện tượng điện từ và các quá trình quang học. Lí thuyết này là cơ sở cho những thành tựu công nghệ rực rỡ mà con người đang hưởng thụ. Nhưng lí thuyết trường điện từ của Maxwell cũng còn những hạn chế nhất định, nhất là nó không giải thích được kết quả thực nghiệm khi nghiên cứu về vật đen của Max Planck và hiện tượng quang điện của Heinrich Hertz. Cơ học lượng tử ra đời giúp giải quyết vấn đề này.
	}
\end{enumerate}