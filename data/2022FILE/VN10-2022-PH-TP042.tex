
\setcounter{section}{0}

\begin{enumerate}[label=\bfseries Câu \arabic*:]
	\item \mkstar{2}
	
	
	{
		Hãy tìm hiểu những tác động của việc sử dụng năng lượng hạt nhân đến môi trường.
	}
	
	\hideall
	{
		\textbf{Ưu điểm của năng lượng hạt nhân}
		
		Tạo ra một số lượng lớn năng lượng: Phản ứng hạt nhân giải phóng nhiều hơn một triệu lần năng lượng so với thủy điện hoặc năng lượng gió. Vì vậy, một lượng điện năng lớn có thể được tạo ra. 
		
		Nguồn năng lượng xanh: Ưu điểm lớn nhất của nguồn năng lượng này là không tạo ra các khí thải nhà kính (như carbon dioxide, methane, ozone, chlorofluorocarbon) trong phản ứng hạt nhân. Khí thải nhà kính là một mối đe dọa lớn cho môi trường sống, chúng gây ra sự nóng lên toàn cầu và biến đổi khí hậu. Phản ứng hạt nhân không tạo ra các khí thải, nên có rất ít ảnh hưởng đến môi trường.
		
		Không làm ô nhiễm không khí: Việc đốt nhiên liệu như than đá tạo ra carbon dioxide và khói. Đó là một mối đe dọa đối với môi trường cũng như đời sống con người. Sản xuất năng lượng hạt nhân không thải ra khói. Vì thế, nó không gây  ô nhiễm không khí trực tiếp. Tuy nhiên, xử lý chất thải phóng xạ là một vấn đề lớn hiện nay.
		
		Nhiên liệu độc lập: Lò phản ứng hạt nhân sử dụng uranium làm nhiên liệu. Phản ứng phân hạch của một lượng nhỏ uranium có thể tạo ra một năng lượng lớn. Hiện nay, nguồn dự trữ uranium được tìm thấy trên Trái đất dự kiến sẽ đáp ứng được nhu cầu trong 100 năm nữa. Sử dụng năng lượng này có thể làm cho nhiều quốc gia có thể độc lập về năng lượng và không phụ thuộc vào việc khai thác những nhiên liệu như than đá. Nếu không có các lỗi của con người hay tai nạn và thiên tai, các lò phản ứng hạt nhân sẽ hoạt động rất hiệu quả trong một thời gian dài. Thêm vào đó, sau khi xây dựng, việc vận hành nhà máy đòi hỏi rất ít lao động.
		
		\textbf{Nhược điểm của năng lượng hạt nhân}
		
		Bức xạ: Sự giải phóng ngẫu nhiên các bức xạ có hại là một trong những hạn chế lớn nhất của năng lượng hạt nhân. Quá trình phân hạch giải phóng bức xạ, nhưng chúng được kiểm soát trong một lò phản ứng hạt nhân. Nếu các biện pháp an toàn không được đảm bảo, các bức xạ có thể tiếp xúc với môi trường sẽ dẫn đến những ảnh hưởng nghiêm trọng đến hệ sinh thái và con người.
		
		Không thể tái tạo: Mặc dù chúng tạo ra một lượng lớn năng lượng, các lò phản ứng hạt nhân vẫn phụ thuộc vào uranium. Đây là một nhiên liệu có thể bị cạn kiệt. Sự cạn kiệt của nó lại có thể gây ra một vấn đề nghiêm trọng. Các lò phản ứng sẽ phải ngừng hoạt động, chúng sẽ vẫn chiếm một diện tích lớn đất đai và làm ô nhiễm môi trường.
		
		Phát triển vũ khí hạt nhân: Năng lượng này có thể được sử dụng cho sản xuất và phổ biến vũ khí hạt nhân. Vũ khí hạt nhân sử dụng phản ứng phân hạch, nhiệt hạch, hoặc kết hợp cả hai phản ứng cho các mục đích phá hủy. Đó là một mối đe dọa lớn đối với thế giới vì chúng có thể gây ra một sự tàn phá quy mô lớn. Tác động của chúng có thể ảnh hưởng tới nhiều thế hệ (ví dụ, vụ đánh bom nguyên tử ở Hiroshima và Nagasaki, Nhật Bản).
		
		Chi phí xây dựng khổng lồ: Tuy một lượng lớn năng lượng có thể được sản xuất từ một nhà máy điện hạt nhân, nhưng nó đòi hỏi chi phí đầu tư lớn. Cần đến khoảng 10-15 năm để xây dựng xong một nhà máy điện hạt nhân. Việc xây dựng một nhà máy điện hạt nhân có thể không khả thi.
		
		Chất thải hạt nhân: Các chất thải được tạo ra sau phản ứng phân hạch chứa các nguyên tố không ổn định và phóng xạ cao. Nó rất nguy hiểm đối với môi trường cũng như sức khỏe con người và sẽ tồn tại trong một khoảng thời gian dài. Nó cần được xử  lý cẩn thận và phải cách biệt với môi trường sống. Độ phóng xạ của các nguyên tố này sẽ giảm trong một thời gian, sau đó phân hủy. Do đó, người ta phải được tích trữ và xử lý một cách cẩn thận. Việc tích trữ các nguyên tố phóng xạ trong một thời gian dài là rất khó khăn.
		
		Tai nạn nhà máy điện hạt nhân: Cho đến này, đã có hai vụ tai nạn nhà máy điện hạt nhân thảm khốc xảy ra: thảm họa Chernobyl xảy ra tại nhà máy điện hạt nhân Chernobyl (1986) tại Ukraine, và thảm họa hạt nhân Fukushima Daiichi (2011) tại Nhật Bản. Sau các sự cố, một lượng lớn các bức xạ đã bị phát tán vào môi trường, dẫn đến những thiệt hại về người, thiên nhiên và đất đai. Người ta không thể phủ nhận khả năng lặp lại những thảm họa này trong tương lai.
		
		Vận chuyển nhiên liệu và chất thải: Việc vận chuyển nhiên liệu uranium và các chất thải phóng xạ là rất khó khăn. Uranium phát ra một số bức xạ, do đó, nó cần phải được xử lý cẩn thận. Chất thải của quá trình sản xuất hạt nhân còn nguy hiểm hơn và cần được bảo vệ tốt hơn. Tất cả các phương tiện vận chuyển chúng đảm bảo các tiêu chuẩn an toàn quốc tế. Mặc dù chưa có tai nạn hoặc sự cố tràn nào được thống kê, nhưng quá trình vận chuyển vẫn còn là thách thức.
		
	}
	\item \mkstar{2}
	
	
	{
		Hãy tìm hiểu các tác động sinh học bởi sự tăng cường tia cực tím do suy giảm tầng ozone. Cần có những hành động thiết thực nào để hạn chế sự giảm sút của tầng ozone?
	}
	
	\hideall
	{
		\textbf{Hậu quả việc suy giảm tầng Ozone đem lại}
		
		Tác động đối với con người
		Sự suy giảm tầng ozon chính là nguyên nhân gây ra ung thư da, hình thành khối u ác tính. Bên cạnh đó nếu tiếp xúc với tia UV sẽ ảnh hưởng xấu, gây các bệnh về mắt
		
		Làm hủy hoại các sinh vật nhỏ: Thủng tầng oezon sẽ làm mất cân bằng hệ sinh thái động thực vật biển. Tia UV tăng lên thì sẽ có ảnh hưởng nghiêm trọng đến quá trình sinh trưởng của các loài tôm, cua, cá,… Làm giảm khả năng sinh sản của chúng.
		
		Đối với thực vật, chúng ta sẽ thấy sự thiệt hại của thảm thực vật qua các yếu tố như quá trình phát triển, thành phần dinh dưỡng,… Khả năng phát triển của chúng cũng bị suy giảm do hiện tượng này.
		
		Làm giảm chất lượng không khí: Tầng ozone suy giảm kéo theo lượng rất lớn bức xạ tử ngoại UV-B đến mặt đất, xác phản ứng hóa học từ đó cũng tăng lên và sẽ dẫn đến ô nhiễm khí quyển.
		
		\textbf{Các giải pháp làm giảm hiện tượng suy giảm tầng Ozone}
		
		- Hạn chế sử dụng năng lượng hạt nhân, sử dụng các loại khí gây thủng tầng Ozone trong các thiết bị, hoạt động sản xuất.
		
		- Hạn chế sử dụng thuốc trừ sâu hóa học.
		
		- Hạn chế sử dụng xe cá nhân
		Sử dụng các sản phẩm sạch thân thiện môi trường.
		
		- Xử lý nghiêm các khu công nghiệp, các nhà máy thải khí độc hại ra môi trường.
		
		- Giáo dục và tuyên truyền cho mọi người để ngăn chặn những hành động xấu làm thủng tầng ozone.
	}
	\item \mkstar{2}
	
	
	{
		"Lỗ thủng" ozone được hiểu như thế nào? Suy giảm tầng ozone là gì?
	}
	
	\hideall
	{
		\textbf{Sự suy giảm tầng ôzôn và “lỗ thủng ôzôn”}
		
		Vào giữa những năm 1970, các nhà khoa học nhận ra rằng tầng ôzôn bị đe dọa do sự tích tụ của các khí có chứa halogen (clo và brôm) trong khí quyển. Sau đó, vào giữa những năm 1980, các nhà khoa học đã phát hiện ra một “lỗ hổng” trong tầng ôzôn phía trên Nam Cực - khu vực bầu khí quyển của Trái đất bị suy giảm nghiêm trọng.
		Vậy, nguyên nhân nào dẫn đến sự mỏng đi của tầng ôzôn trên toàn cầu và “lỗ thủng ôzôn” ở Nam Cực?
		
		Hóa chất nhân tạo có chứa halogen được xác định là nguyên nhân chính gây mất tầng ôzôn. Các hóa chất này được gọi chung là các chất làm suy giảm tầng ôzôn (ODS). ODS đã được sử dụng trong hàng nghìn sản phẩm trong cuộc sống hàng ngày của mọi người trên khắp thế giới.
		
		Các ODS quan trọng nhất là chlorofluorocarbons (CFC), một chất đã từng được sử dụng rộng rãi trong máy điều hòa nhiệt độ, tủ lạnh, bình xịt và trong ống hít được bệnh nhân hen suyễn sử dụng. Các hóa chất khác, chẳng hạn như hydrochlorofluorocarbons (HCFCs), halogen và methyl bromide cũng làm suy giảm tầng ôzôn. Hầu hết máy tính, thiết bị điện tử và các bộ phận của thiết bị của chúng tôi đã được làm sạch bằng dung môi làm suy giảm tầng ôzôn. Bảng điều khiển ô tô, bọt cách nhiệt trong nhà và các tòa nhà văn phòng của chúng tôi, nồi hơi nước và thậm chí cả đế giày đều được làm bằng CFC hoặc HCFC. Các văn phòng, cơ sở máy tính, căn cứ quân sự, máy bay và tàu thủy được sử dụng rộng rãi để phòng cháy chữa cháy. Rất nhiều trái cây và rau quả chúng tôi ăn đã được xông hơi bằng methyl bromide để diệt sâu bệnh.
	}
	\item \mkstar{2}
	
	{
		Hãy tìm hiểu nguyên nhân gây mưa acid và các tác động tiêu cực đến môi trường.
	}
	
	\hideall{
		
		Mưa axit còn có tên gọi tiếng Anh là Acid Rain, dùng để chỉ các chất ô nhiễm công nghiệp có trong nước mưa và nước có độ pH dưới 5,6. Những hạt axit sẽ được lẫn vào trong nước mưa khiến cho độ pH giảm xuống. Mưa axit còn hòa tan một số kim loại nguy hiểm trong không khí khiến nước mưa thêm độc hơn.
		
		Hiện tượng mưa axit được tạo ra bởi lượng khí thải SO$_2$ và NO$_2$, do con người sản xuất trong quá trình phát triển của công nghiệp, hóa chất,... Con người khai thác nhiều than đá, dầu mỏ và các nhiên liệu tự nhiên khác không hợp lý nên dễ làm cho mưa axit xuất hiện.
		
		Nguyên nhân dẫn đến mưa axit có rất nhiều, nhưng chung quy lại chủ yếu là do con người và sự thay đổi của tự nhiên. Sự thay đổi của tự nhiên như sự phun trào của núi lửa, đám cháy,... Trong bầu khí quyển, lượng oxit của lưu huỳnh và nitơ tăng đáng kể làm hình thành mưa axit.
		
		Trong quá trình sống, con người phát triển kinh tế và công nghiệp sử dụng nhiều than đá, dầu mỏ để làm chất đốt. Từ đó, sản sinh ra các khí độc như SO$_2$, NO$_2$, H$_2$SO$_4$, HNO$_3$,... bởi trong than đá, dầu mỏ thường chứa một lượng lớn lưu huỳnh và không khí lại chứa rất nhiều khí nitơ.
		
		Bên cạnh đó, còn một số tác nhân khác như khí thải của các nhà máy công nghiệp, nhiệt điện, luyện kim, lọc dầu,.... do con người sản xuất ra cũng là nguyên nhân hình thành mưa axit.
		
		Tác hại của mưa axit:
		
		- Tác hại đầu tiên chúng ta cần quan tâm đó là mưa axit ảnh hưởng xấu đến sức khỏe con người. Nếu bạn sử dụng nước mưa để sinh hoạt như rửa mặt, tắm, giặt giũ,... có chứa chất axit dễ gây các bệnh về da như nấm, mẩn ngứa, nặng là viêm da. 
		
		- Mưa axit không chỉ ảnh hưởng đến sức khỏe con người mà còn ảnh hưởng đến bầu khí quyển, gây hậu quả không tốt lâu dài đến trái đất. Khi mưa axit kéo dài sẽ làm hạn chế tầm nhìn do trong bầu khí quyển hình thành các hạt sulfate, nitrate.
		
		- Mưa axit làm giảm khả năng sống, phát triển của các sinh vật sống dưới nước, vì mưa axit có độ pH thấp từ đó làm giảm độ pH có trong nước. Nếu lượng mưa axit có nhiều trong ao hồ, khiến những loài sinh vật bị suy yếu và chết dần.
		
		- Mưa axit không chỉ gây hại cho động vật mà còn ảnh hướng lớn đến thực vật, đặc biệt là các loại cây trồng. Khi xảy ra hiện tượng mưa axit, nước sẽ thấm vào đất và hòa tan các chất độc hại có trong đất. Rễ cây sẽ hấp thụ và ảnh hưởng đến năng suất cây trồng.
		
		- Thiệt hại vật chất lớn nhất từ mưa axit mang đến cho con người là làm xói mòn bề mặt các công trình kiến trúc. Minh chứng cụ thể như mưa axit đã làm hòa tan các loại đá như sa thạch, vôi, cẩm thạch.
		
		
	}
	
	\item \mkstar{2}
	
	
	{
		Hãy tìm hiểu sự biến đổi khí hậu hiện nay và tác động tiêu cực đến Việt Nam. Cần có những hành động thiết thực để hạn chế sự biến đổi khí hậu?
	}
	
	\hideall
	{
		\textbf{Tác động của BĐKH đến Việt Nam:}
		Trong những năm gần đây, nghiên cứu ảnh hưởng của biến đổi khí hậu cũng đã chỉ ra nhiều vấn đề cần phải lưu tâm, chẳng hạn năm 2017 là năm có số lượng các cơn bão ảnh hưởng tới nước ta nhiều bất thường (16 cơn bão), theo tính  toán của Ban chỉ đạo trung ương về phòng chống thiên tai và tổng cục thống kê thiệt hại khoảng 38,7 nghìn tỷ đồng tương đương 2,7 tỷ USD.
		
		Ảnh hưởng của BĐKH đến hoạt động sản xuất thấy rõ nhất là đối với ngành nông nghiệp.
		
		Biến đổi khí hậu còn ảnh hưởng tới hoạt động giao thông vận tải.
		
		Biến đổi khí hậu ảnh hưởng đến phát triển đô thị, các khu công nghiệp và nhà ở, mức độ ảnh hưởng tùy thuộc vào từng vùng, từng địa phương và từng vị trí theo địa hình phân bố. Nghiên cứu tổng thể cho thấy khu vực ven biển chịu tác động chính của bão, vùng miền núi chịu tác động của lũ quét, lốc xoáy, sạt lở, vùng trung du và đồng bằng chủ yếu là ngập lụt, lốc xoáy, mưa đá.
		
		Biến đổi khí hậu ảnh hưởng đến du lịch, thương mại, năng lượng…và nhiều hoạt động kinh tế khác trực tiếp hay gián tiếp. Những ảnh hưởng này trong những năm vừa qua đã biểu hiện khá rõ nét, mỗi ngành, lĩnh vực đều có thể cảm nhận và đánh giá được ảnh hưởng của BĐKH.
		
		Đối với công nghiệp, ảnh hưởng của BĐKH khí hậu sẽ tác động đến ngành công nghiệp chế biến, nhất là chế biến những sản phẩm nông nghiệp. Trong trường hợp nhiệt độ tăng sẽ làm tăng tiêu thụ năng lượng kéo theo nhiều hoạt động khác tăng theo như tăng công suất nhà máy phát điện, tăng sử dụng các thiết bị làm mát, ảnh hướng tới an ninh năng lượng quốc gia. 
		
		Đối với lãnh thổ, do đặc điểm địa lý, mỗi khu vực có vị trí Địa lý và địa hình khác nhau, mức độ chịu ảnh hưởng của BĐKH đối với mỗi vùng cũng khác nhau, dựa trên tiêu chí cơ bản về nước biển dâng, đặc trưng ảnh hưởng của biến đổi khí hậu có thể chia thành ba địa bàn lãnh thổ gồm đồng bằng, ven biển và miền núi.
		
		\textbf{Những giải pháp cần có đối với ảnh hưởng của Biến đổi khí hậu.}
		
		Nhằm hạn chế những ảnh hưởng tiêu cực và tận dụng những mặt tích cực của BĐKH dựa trên cơ sở đánh giá khách quan và những dự báo có tính dài hạn để từ đó có những giải pháp phù hợp.
		
		Thứ nhất, tăng cường phổ biến, truyền thông, nâng cao nhận thức về ảnh hưởng của BĐKH cho mọi tầng lớp nhân dân, các cấp, các ngành nhất là những diễn biến trong bối cảnh mới. Sau đại dịch Covid-19, cần có cách thức mới để người dân thấy được BĐKH cũng là nguyên nhân tác động tới sức khỏe và ảnh hưởng tới hệ sinh thái, con người gắn bó với hệ sinh thái và phải chống chịu, thích ứng.
		
		Thứ hai, BĐKH đã được khẳng định là một nhân tố ảnh hưởng tới mọi mặt đời sống, kinh tế-xã hội, Việt Nam là một trong những quốc gia chịu ảnh hưởng nặng nề nhất của BĐKH, do vậy về quan điểm chúng ta chấp nhận những ảnh hưởng này để có những giải pháp phù hợp, trong đó có những ảnh hưởng tiêu cực nhưng đồng thời cũng có những ảnh hưởng có tính tích cực, biến những thách thức của BĐKH thành những cơ hội để chủ động giảm thiểu và thích ứng dựa trên nguyên tắc giải quyết hài hòa, tính  hiệu quả, phù hợp với diễn thế của thiên nhiên được đặt lên hàng đầu.
		
		Thứ ba, xét trong bối cảnh mới, một trong những điểm nghẽn để giải quyết những vấn đề liên quan đến thể chế đó là “hoàn thiện thể chế kinh tế thị trường định hướng xã hội chủ nghĩa”. Giải pháp phù hợp trong ứng phó với BĐKH cần dựa vào tiếp cận thị trường (MBA), trước hết là vai trò của doanh nghiệp và người dân nhằm tạo ra cơ chế tạo nguồn lực, nhất là nguồn lực tài chính. Hiện tại quỹ ứng phó với BĐKH đã và đang vận hành ở nhiều quốc gia phát triển trên thế giới như một số nước ở chấu Âu, Nhật Bản, Hàn Quốc…, nếu doanh nghiệp và người dân tiếp cận quỹ này sẽ có được nguồn lực tài chính không phụ thuộc vào ngân sách Nhà nước. Thị trường Cac Bon là một cơ chế tài chính tốt trên thị trường, Việt Nam nên sớm hình thành và tham gia vào thị trường này.
		
		Thứ tư, cần phải căn cứ vào kịch bản đưa ra do BĐKH, nhất là kịch bản nước biển dâng thêm 1m từ nay đến cuối thế kỷ 21 để có những tính toán đầy đủ và phương án qui hoạch phù hợp đối với hoạt động sản xuất các ngành, lĩnh vực, phân bố dân cư và chuyển đổi phát triển kinh tế theo cơ cấu ngành và cơ cấu lãnh thổ phù hợp.
		
		Thứ năm, trong qui hoạch và xây dựng chính sách cần phải lồng ghép yếu tố BĐKH để có biện pháp chủ động ứng phó phù hợp với những ảnh hưởng của BĐKH đối với từng ngành, lĩnh vực và từng vùng phù hợp với thực tiễn đang và sẽ diễn ra.
		
		Thứ sáu, cần có những giải pháp cho từng ngành, lĩnh vực và từng vùng đối với ảnh hưởng của BĐKH xét trong bối cảnh mới. Trước hết chú trọng tới ngành nông nghiệp trong việc cơ cấu lại cây trồng, vật nuôi do những ảnh hưởng của BĐKH không chỉ chống chịu mà còn mang lại hiệu quả kinh tế cao nhất. Đối với từng vùng, ưu tiên hàng đầu là vùng ĐBSCL, triển khai và thực hiện tốt Nghị quyết 120 của Chính phủ để phát triển bền vững vùng ĐBSCL trong bối cảnh BĐKH. Những vùng khác phụ thuộc vào đặc trưng của mỗi vùng để có những giải pháp phù hợp, đối với vùng ven biển, ưu tiên hàng đầu là ứng phó với bão, áp thấp nhiệt đới, đối với vùng miền núi là lũ lụt, lũ quét, hạn hán, lốc xoáy và thoái hóa đất.
		
		Thứ bảy, tăng cường khả năng chống chịu trước ảnh hưởng của BĐKH, phát triển các mô hình kinh tế dựa vào hệ sinh thái (EbA),  dựa vào cộng đồng (CbA) và dựa vào tự nhiên (NbS), cho mỗi địa phương, lấy kiến thức bản địa kết hợp với khoa học công nghệ mới đầu tư phát triển, đặt sinh kế và sự an toàn của người dân lên hàng đầu. Từ những kết quả đạt được, nhân rộng mô hình cho từng địa phương, cho từng vùng.
		
		Thứ tám, biến đổi khí hậu là vấn đề có tính toàn cầu, chỉ một mình Việt Nam sẽ không giải quyết được, chính vì vậy cần có sự phối hợp với các tổ chức, các quốc gia trong khu vực và trên thế giới, hơn nữa chúng ta đã cam kết cùng nỗ lực toàn cầu để giảm thiểu phát thải khí nhà kính, chính vì vậy cần tăng cường hơn nữa hợp tác quốc tế, nắm bắt cơ hội để có những chuyển giao về khoa học công nghệ thế hệ mới, đầu tư tài chính và kinh nghiệm quốc tế trong ứng phó với BĐKH.
	}
	\item \mkstar{3}
	
	
	{
		Trong những ngày trời lạnh, sưởi bằng bếp than trong phòng kín có nguy cơ gây ngạt thở, thậm chí tử vong. Bạn hãy tìm hiểu nguyên nhân tại sao?
	}
	
	\hideall
	{
		Trong khói than chứa nhiều thành phần độc hại như: cacbon monoxit, CO$_2$, nitơ oxit và một số chất khác như lưu huỳnh oxit , muội than, hydrocacbon chưa cháy hết, fomandehit,… Các chất này khi tỏa ra trong không khí sẽ gây ảnh hưởng trực tiếp tới sức khỏe con người. Nó có thể là tác nhân khởi phát cơn khó thở đối với người mắc bệnh phổi tắc nghẽn mạn tính, người bệnh hen suyễn, cũng là tác nhân gây ngạt rất nguy hiểm với hệ hô hấp non nớt của trẻ em. Tiếp xúc thường xuyên với khói than làm tăng nguy cơ nhiễm trùng hô hấp như viêm phổi.
	}
	\item \mkstar{3}
	
	
	{
		Bạn hãy tìm hiểu thêm về các cách thu hồi dầu loang: sử dụng phao quây dầu, dùng vật liệu hấp thụ dầu và đốt tại chỗ. 
	}
	
	\hideall
	{
		1: Sử dụng booms (phao quây dầu)
		
		Phao quây dầu là một phương pháp rất phổ biến trong việc kiểm soát sự cố tràn dầu. Có nhiều loại phao quây dầu đã được thiết kế cho các khu vực khác nhau, nơi sự cố tràn dầu có thể xảy ra.
		
		2. Sử dụng Sorbents (chất hấp thụ dầu)
		
		Sorbens có nghĩa là các vật liệu hấp thu dầu được đặt trên bề mặt của khu vực bị ảnh hưởng tràn. Các chất hấp thụ này hút và hấp thụ dầu từ trên bề mặt của nước. Chúng nổi trên bề mặt nước, không hấp thụ nước nhưng hấp thụ dầu rất mạnh.
		
		3. Đốt tại chỗ
		
		Nói cách đơn giản, điều này có nghĩa là đốt dầu trên mặt biển, nơi xảy ra sự cố tràn dầu. Việc đốt cháy phải được thực hiên nhanh chóng trước khi sự cố tràn dầu có thể lan đến một khu vực rộng lớn hơn. Nhưng nhược điểm của việc đốt cháy tại chỗ là khí thải được giải phóng có chứa các chất độc hại có thể gây ra thiệt hại cho không khí đại dương và các sinh vật biển.
	}
	\item \mkstar{3}
	
	
	{
		Theo quy định của Chính phủ, phạt cảnh cáo hoặc phạt tiền đối với hành vi sử dụng diêm, bật lửa, điện thoại di động ở những nơi có quy định cấm ở trạm xăng. Vì sao không được dùng điện thoại di động ở trạm xăng?
	}
	
	\hideall
	{
		Khi đến các trạm xăng rất dễ dàng nhìn thấy biển cấm sử dụng điện thoại di động. Lý do có biển cấm đó bởi sử dụng điện thoại di động khi đang bơm xăng rất nguy hiểm, dễ gây ra cháy nổ.
		
		Sóng điện thoại: Nguyên do chính là hiện tượng bốc hơi ở xăng tạo nên ion tích điện xung quanh các cây xăng. Mỗi khi người dùng gọi điện thoại hoặc kết nối không dây như 3G, Wifi hay Bluetooth sẽ làm tăng gấp nhiều lần công suất phát sóng của điện thoại di động.
		
		Pin của điện thoại di động: Hai trường hợp có thể gây hậu quả đó là khi pin kém chất lượng hoặc dùng một thời gian quá lâu. Điều này làm cho điểm tiếp xúc pin và điện thoại bị mòn dần, tạo ra tia lửa điện khi điện thoại di động được dùng để nghe hoặc gọi.Nếu chẳng may đánh rơi điện thoại, tia lửa điện có khả năng xuất hiện từ chính trong cục pin. Thêm vào đó, sử dụng đèn flash trên điện thoại cũng gây cháy cao.
		
		Nhiệt độ thay đổi của điện thoại di động: Vì điện thoại tản nhiệt qua vỏ máy, khi nhiệt độ điện thoại tăng lên do nghe gọi hay chơi game, nếu linh kiện không đảm bảo sẽ gây cháy. Hiện tượng này xảy ra do nóng bất thường kết hợp với sự ma sát vải quần. Trong trường hợp điện thoại quá nóng có thể tạo tiếng nổ sẽ vô cùng nguy hiểm. Dù vậy, nhiệt độ điện thoại sẽ chẳng bao giờ cao như pô xe máy nên khó cháy nổ.
		
		
	}
	\item \mkstar{3}
	
	{
		Bạn hãy tìm hiểu thêm về quá trình sinh ra và phân rã ozone trong tự nhiên.
	}
	
	\hideall{
		Khi ozone được sinh ra, chúng nhanh chóng bị phân hủy trong thời gian ngắn bởi đây là một hợp chất có liên kết kém bền vững. Sự phân hủy của ozone không chỉ phụ thuộc vào nhiệt độ môi trường mà chúng còn bị ảnh hưởng bởi nồng độ pH, nồng độ các chất hòa tan, các chất hữu cơ tự nhiên. 
	}
	
	\item \mkstar{2}
	
	
	{
		Phân biệt giữa khí hậu và thời tiết.
	}
	
	\hideall
	{
		Giống nhau 
		
		Thời tiết và khí hậu đều diễn ra trong một vùng nhất định, đều có các yếu tố của các hiện tượng khí tượng: không khí, nhiệt độ, áp suất, lượng mưa,… người ta dựa vào các yếu tố này để phân loại thời tiết và khí hậu.
		
		Khác nhau
		
		- Sự khác nhau giữa khí hậu và thời tiết dựa vào thời gian và tính chất của nó
		
		+ Thời tiết là sự biểu hiện của các hiện tượng khí tượng xảy ra trong một thời gian ngắn tại một địa phương xác định và thời tiết luôn luôn thay đổi.
		
		(Ví dụ:  thời tiết trong 1 ngày tại TP Hồ Chí Minh, sáng trời nắng, chiều trời mưa)
		
		+ Khí hậu là sự lặp đi lặp lại của tình hình thời tiết ở một địa phương trong một thời gian dài và nó đã trở thành quy luật lặp đi lặp lại tạo ra đặc trưng về khí tượng cho một vùng miền.
		
		(Ví dụ: Việt Nam thuộc khí hậu nhiệt đới gió mùa, khó có thể thay đổi do phụ thuộc vị trí địa lí).
		
	}
	
\end{enumerate}