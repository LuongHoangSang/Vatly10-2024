\let\lesson\undefined
\newcommand{\lesson}{\phantomlesson{Ôn tập chương 9}}
\setcounter{section}{2}
\ANSMCQ
{	\begin{center}
		\begin{tabular}{|m{2.8em}|m{2.8em}|m{2.8em}|m{2.8em}|m{2.8em}|m{2.8em}|m{2.8em}|m{2.8em}|m{2.8em}|m{2.8em}|}
			\hline
			1.B  & 2.B  & 3.D  & 4.A  & 5.C  & 6.A  & 7.C  & 8.B  & 9.B  & 10.A  \\
			\hline
			11.C  & 12.B  & 13.B  & 14.B  & 15.D  & 16.A  & 17.B  & 18.D  & 19.A  & 20.D  \\
			\hline
		\end{tabular}
	\end{center}
}
\begin{enumerate}[label=\bfseries Câu \arabic*:, leftmargin=1.5cm]
	\item \mkstar{1}
	
	
	{
		Lực đàn hồi xuất hiện tỉ lệ với độ biến dạng khi 
		\begin{mcq}(2)
			\item một vật bị biến dạng dẻo.
			\item một vật biến dạng đàn hồi.
			\item một vật bị biến dạng.	
			\item ta ấn ngón tay vào một viên đất nặn.
		\end{mcq}
	}
	
	\hideall
	{	
		\textbf{Đáp án: B.}
		
		Lực đàn hồi xuất hiện tỉ lệ với độ biến dạng khi một vật biến dạng đàn hồi.
	}
	\item \mkstar{1}
	
	
	{
		Kết luận nào sau đây \textbf{không} đúng đối với lực đàn hồi?
		\begin{mcq}(2)
			\item Xuất hiện khi vật bị biến dạng.
			\item Luôn là lực kéo.
			\item Tỉ lệ với độ biến dạng.
			\item Ngược hướng với lực làm nó bị biến dạng.
		\end{mcq}
	}
	
	\hideall
	{	
		\textbf{Đáp án: B.}
		
		Lực đàn hồi có khi là lực kéo, có khi là lực nén.
	}
	\item \mkstar{1}
	
	
	{
		Điều nào sau đây là \textbf{sai} khi nói về phương và độ lớn của lực đàn hồi?
		\begin{mcq}
			\item Với cùng độ biến dạng như nhau, độ lớn của lực đàn hồi phụ thuộc vào kích thước và bản chất của vật đàn hồi. 
			\item Với các mặt tiếp xúc bị biến dạng, lực đàn hồi vuông góc với các mặt tiếp xúc. 
			\item Với các vật như lò xo, dây cao su, thanh dài, lực đàn hồi hướng dọc theo trục của vật. 
			\item Lực đàn hồi có độ lớn tỉ lệ nghịch với độ biến dạng của vật biến dạng. 
		\end{mcq}
	}
	
	\hideall
	{	
		\textbf{Đáp án: D.}
		
		Lực đàn hồi tỉ lệ thuận với độ biến dạng của vật biến dạng.
	}
	\item \mkstar{1}
	
	
	{
		Khẳng định nào sau đây là đúng khi ta nói về lực đàn hồi của lò xo và lực căng của dây?
		\begin{mcq}
			\item Đó là những lực chống lại sự biến dạng đàn hồi của lò xo và sự căng của dây.
			\item Đó là những lực gây ra sự biến dạng đàn hồi của lò xo và sự căng của dây.
			\item Chúng đều là những lực kéo.
			\item Chúng đều là những lực đẩy.
		\end{mcq}
	}
	
	\hideall
	{	
		\textbf{Đáp án: A.}
		
		Lực đàn hồi của lò xo và lực căng của dây: đó là những lực chống lại sự biến dạng đàn hồi của lò xo và sự căng của dây.
	}
	\item \mkstar{1}
	
	
	{
		Một vật tác dụng một lực vào một lò xo có đầu cố định và làm lò xo biến dạng. Điều nào dưới đây là \textbf{không} đúng?
		\begin{mcq}
			\item Độ đàn hồi của lò xo có độ lớn bằng lực tác dụng và chống lại sự biến dạng của lò xo.
			\item Lực đàn hồi cùng phương và ngược chiều với lực tác dụng. 
			\item Lực đàn hồi lớn hơn lực tác dụng và chống lại lực tác dụng.
			\item Khi vật ngừng tác dụng lên lò xo thì lực đàn hồi của lò xo cũng mất đi.
		\end{mcq}
	}
	
	\hideall
	{	
		\textbf{Đáp án: C.}
	}
	\item \mkstar{1}
	
	
	{
		Chọn phát biểu \textbf{sai} về lực đàn hồi của lò xo.
		\begin{mcq}
			\item Lực đàn hồi của lò xo có xu hướng chống lại nguyên nhân gây ra biến dạng.
			\item Lực đàn hồi của lò xo dài có phương là trục lò xo, chiều ngược với chiều biến dạng của lò xo.
			\item Lực đàn hồi của lò xo có độ lớn tuân theo định luật Húc.
			\item Lực đàn hồi của lò xo chỉ xuất hiện ở đầu lò xo đặt ngoại lực gây biến dạng.
		\end{mcq}
	}
	
	\hideall
	{	
		\textbf{Đáp án: A.}
		
		Lực đàn hồi xuất hiện ở hai đầu của lò xo và tác dụng vào các vật tiếp xúc (hay gắn) với lò xo làm nó biến dạng.
	}
	\item \mkstar{1}
	
	
	{
		Lực đàn hồi của lò xo có tác dụng làm cho lò xo
		\begin{mcq}(2)
			\item chuyển động.
			\item thu gia tốc. 
			\item có xu hướng lấy lại hình dạng và kích thước ban đầu.        
			\item vừa biến dạng vừa thu gia tốc.
		\end{mcq}
	}
	
	\hideall
	{	
		\textbf{Đáp án: C.}
	}
	\item \mkstar{2}
	
	
	{
		Một lò xo có chiều dài tự nhiên là $\SI{20}{cm}$. Khi lò xo có chiều dài $\SI{24}{cm}$ thì lực đàn hồi của nó bằng $\SI{5}{N}$. Hỏi khi lực đàn hồi của lò xo bằng $\SI{10}{N}$ thì chiều dài của nó bằng bao nhiêu?
		\begin{mcq}(4)
			\item $\SI{22}{cm}$.
			\item $\SI{28}{cm}$.
			\item $\SI{40}{cm}$.
			\item $\SI{48}{cm}$.
		\end{mcq}
	}
	
	\hideall
	{	
		\textbf{Đáp án: B.}
		
		Ta có:
		
		$$F = k \Delta l.$$
		
		Vậy: 
		$$F_1 = k \Delta l_1; F_2 = k \Delta l_2.$$
		
		Lập tỉ số:
		
		$$\dfrac{F_1}{F_2} = \dfrac{\Delta l_1}{\Delta l_2} \Rightarrow \Delta l_2 = \SI{0,08}{m} = \SI{8}{cm}.$$
		
		Chiều dài của lò xo là
		
		$$l' = l_0 + \Delta l_2 = \SI{28}{cm}.$$
	}
	\item \mkstar{2}
	
	
	{
		Một lò xo có chiều dài tự nhiên bằng $\SI{22}{cm}$. Lò xo được treo thẳng đứng, một đầu giữ cố định, còn đầu kia gắn một vật nặng. Khi ấy lò xo dài $\SI{27}{cm}$, cho biết độ cứng lò xo là $\SI{100}{N/m}$. Độ lớn lực đàn hồi bằng 
		\begin{mcq}(4)
			\item $\SI{500}{N}.$
			\item $\SI{5}{N}.$
			\item $\SI{20}{N}.$
			\item $\SI{50}{N}.$
		\end{mcq}
	}
	
	\hideall
	{	
		\textbf{Đáp án: B.}
		
		Độ biến dạng của lò xo:
		
		$$\Delta l = l-l_0 = \SI{5}{cm} = \SI{0,05}{m}.$$
		
		Độ lớn của lực đàn hồi:
		
		$$F_\text{đh} = k\Delta l = \SI{5}{N}.$$
		
	}
	\item \mkstar{2}
	
	
	{
		Phải treo một vật có khối lượng bằng bao nhiêu vào lò xo có độ cứng $\SI{100}{N/m}$ để lò xo dãn ra được $\SI{10}{cm}$? Lấy $g = \SI{10}{m/s}^2$. 
		\begin{mcq}(4)
			\item $\SI{1}{kg}$.
			\item $\SI{10}{kg}$.
			\item $\SI{100}{kg}$.
			\item $\SI{1000}{kg}$.
		\end{mcq}
	}
	
	\hideall
	{	
		\textbf{Đáp án: A.}
		
		Ta có: 
		
		$$F_\text{đh} = k\Delta l = \SI{10}{N}.$$
		
		Mà $F_\text{đh} = P.$
		
		Suy ra:
		
		$$m = \dfrac{P}{g} = \SI{1}{kg}.$$
		
		
	}
	
	\item \mkstar{2}
	
	
	{
		Dùng một lò xo để treo một vật có khối lượng $\SI{300}{g}$ thì thấy lò xo giãn một đoạn $\SI{2}{cm}$. Nếu treo thêm một vật có khối lượng $\SI{150}{g}$ thì độ giãn của lò xo là
		\begin{mcq}(4)
			\item $\SI{1}{cm}$.
			\item $\SI{2}{cm}$.
			\item $\SI{3}{cm}$.
			\item $\SI{4}{cm}$.
		\end{mcq}
	}
	
	\hideall
	{	
		\textbf{Đáp án: C.}
		
		Khi treo vật khối lượng $\SI{300}{g}$:
		
		$$m_1g = k\Delta l_1 \Rightarrow k = \dfrac{m_1 g}{\Delta l_1}.$$
		
		Khi treo thêm một vật ta có:
		
		$$(m_1+m_2)g = k\Delta l_2 \Rightarrow \Delta l_2 = \dfrac{(m_1+m_2)g}{k} = \SI{0,03}{m} = \SI{3}{cm}.$$
	}
	\item \mkstar{2}
	
	
	{Một lò xo có chiều dài tự nhiên bằng $\SI{20}{cm}$. Khi bị kéo lò xo dài $\SI{24}{cm}$ và lực đàn hồi của nó bằng $\SI{5}{N}$. Hỏi khi lực đàn hồi của lò xo bằng $\SI{15}{N}$ thì chiều dài của nó bằng bao nhiêu?
		
		\begin{mcq}(4)
			\item $\SI{28}{cm}$.
			\item $\SI{32}{cm}$.
			\item $\SI{45}{cm}$.
			\item $\SI{20}{cm}$.
		\end{mcq}
	}
	
	\hideall
	{	
		\textbf{Đáp án: B.}
		
		Ta có:
		
		$$F = k \Delta l.$$
		
		Vậy: 
		$$F_1 = k \Delta l_1; F_2 = k \Delta l_2.$$
		
		Lập tỉ số:
		
		$$\dfrac{F_1}{F_2} = \dfrac{\Delta l_1}{\Delta l_2} \Rightarrow \Delta l_2 = \SI{0,12}{m} = \SI{12}{cm}.$$
		
		Chiều dài của lò xo là
		
		$$l' = l_0 + \Delta l_2 = \SI{32}{cm}.$$
	}
	
	
	\item\mkstar{2}\\
	{Một lò xo có chiều dài tự nhiên $\SI{22}{\centi\meter}$. Lò xo được treo thẳng đứng, một đầu được giữ cố định, đầu còn lại gắn với vật nặng. Khi ấy lò xo dài $\SI{27}{\centi\meter}$, cho biết độ cứng lò xo là $\SI{100}{\newton/\meter}$. Độ lớn lực đàn hồi của lò xo bằng
		\begin{mcq}(4)
			\item $\SI{500}{\newton}$.
			\item $\SI{5}{\newton}$.
			\item $\SI{20}{\newton}$.
			\item $\SI{50}{\newton}$.
		\end{mcq}
	
}
\hideall{
\textbf{Đáp án B.}\\
Độ lớn lực đàn hồi của lò xo:
$$F_\text{đh}=k\left|\Delta\ell\right|=\left(\SI{100}{\newton/\meter}\right)\cdot\left(\SI{0.05}{\meter}\right)=\SI{5}{\newton}.$$
}

\item \mkstar{3}\\
{Một lò xo có chiều dài tự nhiên $\ell_0=\SI{27}{\centi\meter}$, được treo thẳng đứng. Khi treo vào một vật có trọng lượng $P_1=\SI{5}{\newton}$ thì lò xo dài $\ell_1=\SI{44}{\centi\meter}$. Khi treo vật khác có trọng lượng $P_2$ chưa biết, lò xo dài $\ell_2=\SI{35}{\centi\meter}$. Độ cứng của lò xo và trọng lượng $P_2$ là
	\begin{mcq}(2)
		\item $k=\SI{25.3}{\newton/\meter}$, $P_2=\SI{2.35}{\newton}$.
			\item $k=\SI{29.4}{\newton/\meter}$, $P_2=\SI{2.35}{\newton}$.
			\item $k=\SI{25.3}{\newton/\meter}$, $P_2=\SI{3.5}{\newton}$.
			\item $k=\SI{29.4}{\newton/\meter}$, $P_2=\SI{3.5}{\newton}$.
	\end{mcq}

}
\hideall{
\textbf{Đáp án B.}\\
Độ cứng lò xo:
$$k=\dfrac{P_1}{\ell_1-\ell_0}=\dfrac{\left(\SI{5}{\newton}\right)}{\SI{0.44}{\meter}-\SI{0.27}{\meter}}\approx\SI{29.4}{\newton/\meter}$$
Trọng lượng $P_2$:
$$P_2=k\left|\Delta\ell_2\right|=\left(\SI{29.4}{\newton/\meter}\right)\cdot\left|\SI{0.35}{\meter}-\SI{0.27}{\meter}\right|\approx\SI{2.35}{\newton}.$$
}


\item\mkstar{3}\\
{Một lò xo có độ cứng $k$, chiều dài tự nhiên $\ell_0$ được treo thẳng đứng, đầu trên cố định. Khi người ta treo quả cân có khối lượng $\SI{200}{\gram}$ vào đầu dưới của lò xo, lò xo có độ dài $\SI{32}{\centi\meter}$. Nếu treo thêm quả cân có khối lượng $\SI{500}{\gram}$ vào đầu dưới của lò xo thì lò xo có chiều dài $\SI{37}{\centi\meter}$. Lấy $g=\SI{10}{\meter/\second^2}$. Độ dài tự nhiên và độ cứng của lò xo là
	\begin{mcq}(2)
		\item $\ell_0=\SI{30}{\centi\meter}$, $k=\SI{1000}{\newton/\meter}$.
		\item $\ell_0=\SI{32}{\centi\meter}$, $k=\SI{300}{\newton/\meter}$.
\item $\ell_0=\SI{32}{\centi\meter}$, $k=\SI{2300}{\newton/\meter}$.
\item $\ell_0=\SI{30}{\centi\meter}$, $k=\SI{100}{\newton/\meter}$.
	\end{mcq}

}
\hideall{
\textbf{Đáp án D.}\\
Độ cứng của lò xo:
$$k=\dfrac{m_2g}{\ell_2-\ell_1}=\dfrac{\left(\SI{0.5}{\kilogram}\right)\cdot\left(\SI{10}{\meter/\second^2}\right)}{\SI{0.37}{\meter}-\SI{0.32}{\meter}}=\SI{100}{\newton/\meter}$$
Độ biến dạng của lò xo khi treo vật khối lượng $m_1=\SI{200}{\gram}$:
$$\Delta\ell_1=\dfrac{m_1g}{k}=\SI{0.02}{\meter}=\SI{2}{\centi\meter}$$
Độ dài tự nhiên của lò xo:
$$\ell_0=\ell_1-\Delta\ell_1=\SI{30}{\centi\meter}.$$
}

\item \mkstar{3}\\
{Người ta treo một đầu lò xo vào một điểm cố định, đầu dưới của lò xo là những chùm quả nặng, mỗi quả đều có khối lượng $m$. Khi chùm quả nặng có 2 quả, chiều dài lò xo là $\SI{15}{\centi\meter}$. Khi chùm quả nặng có 4 quả, chiều dài lò xo là $\SI{17}{\centi\meter}$. Cho $g=\SI{10}{\meter/\second^2}$. Số quả nặng cần treo để lò xo dài $\SI{21}{\centi\meter}$ là
	\begin{mcq}(4)
		\item 8 quả.
		\item 10 quả.
		\item 6 quả.
		\item 9 quả.
	\end{mcq}
	
}
\hideall{
	\textbf{Đáp án A.}\\
	Độ biến dạng của lò xo khi treo quả nặng:
	\begin{eqnarray*}
		&&\Delta\ell=\dfrac{mg}{k}\\
		&\Rightarrow& \dfrac{\ell_2-\ell_1}{\ell_3-\ell_1}=\dfrac{\left(m_2-m_1\right)g}{\left(m_3-m_1\right)g}\\
		&\Leftrightarrow& \dfrac{4m-2m}{N\cdot m-2m}=\dfrac{\SI{17}{\centi\meter}-\SI{15}{\centi\meter}}{\SI{21}{\centi\meter}-\SI{15}{\centi\meter}}\\
		&\Rightarrow& N=8.
	\end{eqnarray*}
}

\item \mkstar{3}\\
{ Một vật có khối lượng $M$ được gắn vào một dầu của lò xo có độ cứng $k$ được đặt trên mặt phẳng nghiêng góc $\theta$ so với phương ngang, bỏ qua ma sát giữa vật và mặt nghiêng, vật ở trạng thái đứng yên. Độ dãn của lò xo là
\begin{center}
	\includegraphics[width=0.2\linewidth]{../figs/VN10-2023-PH-TP0009-1}
\end{center}
\begin{mcq}(4)
	\item $x=\dfrac{2Mg\sin\theta}{k}$.
	\item $x=\dfrac{Mg\sin\theta}{k}$.
	\item $x=\dfrac{Mg}{k}$.
	\item $x=\sqrt{2Mg}$.
\end{mcq}
}
\hideall{
	\textbf{Đáp án B.}\\
Vật nhỏ cân bằng:
\begin{equation}
	\label{eq:0009-1}
	\vec P+\vec N+\vec F_\text{đh}=\vec 0
\end{equation}
Chiếu phương trình (\ref{eq:0009-1}) lên phương song song măt nghiêng:
\begin{eqnarray*}
	&&P\sin\theta=F_\text{đh}\\
	&\Leftrightarrow& mg\sin\theta=kx\\
	&\Rightarrow& x=\dfrac{Mg\sin\theta}{k}.
\end{eqnarray*}
}

\item \mkstar{3}\\
{Hai lò xo A và B được bố trí như hình vẽ. Độ cứng của lò xo A là $k_A=\SI{100}{\newton/\meter}$. Khi kéo đầu tự do của lò xo B ra, lò xo A dãn $\SI{5}{\centi\meter}$, lò xo B dãn $\SI{1}{\centi\meter}$. Độ cứng của lò xo B là
	\begin{center}
		\includegraphics[width=0.3\linewidth]{../figs/VN10-2023-PH-TP0009-2}
	\end{center}
	\begin{mcq}(4)
		\item $\SI{100}{\newton\meter}$.
		\item $\SI{25}{\newton/\meter}$.
		\item $\SI{350}{\newton/\meter}$.
		\item $\SI{500}{\newton/\meter}$.
	\end{mcq}

}
\hideall{
\textbf{Đáp án D.}\\
Vì hai lò xo được ghép nối tiếp, lực đàn hồi trên hai lò xo có độ lớn như nhau:
\begin{eqnarray*}
	&&F_\text{đh A}=F_\text{đh B}\\
	&\Leftrightarrow& \dfrac{k_B}{k_A}=\dfrac{\Delta\ell_A}{\Delta\ell_B}=5\\
	&\Rightarrow& k_B=5k_A=\SI{500}{\newton/\meter}.
\end{eqnarray*}


}

\item \mkstar{3}\\
{Hai lò xo giống nhau có cùng độ cứng $\SI{100}{\newton/\meter}$ được bố trí như hình vẽ. Vật $m$ có khối lượng $\SI{200}{\gram}$. Khi vật nặng cân bằng, độ dãn của mỗi lò xo là
	\begin{center}
		\includegraphics[width=0.1\linewidth]{../figs/VN10-2023-PH-TP0009-4}
	\end{center}
\begin{mcq}(4)
	\item $\SI{1}{\centi\meter}.$
	\item $\SI{2}{\centi\meter}$.
	\item $\SI{1.5}{\centi\meter}$.
	\item $\SI{3}{\centi\meter}$.
\end{mcq}
}
\hideall{
\textbf{Đáp án A.}\\
Vật nặng cân bằng:
$$F_\text{đh 1}+F_\text{đh 2}=mg$$
$$\Leftrightarrow \Delta\ell=\dfrac{mg}{2k}=\SI{0.01}{\meter}=\SI{1}{\centi\meter}.$$
}

\item \mkstar{3}\\
{Hình bên là đồ thị gồm hai đường thẳng xiên góc đi qua gốc toạ độ $O$, mô tả sự thay đổi giá trị của lực đàn hồi theo các độ dãn khác nhau của lò xo X và lò xo Y. Chọn kết luận đúng về độ cứng của hai lò xo.
	\begin{center}
		\includegraphics[width=0.3\linewidth]{../figs/VN10-2023-PH-TP0009-3}
	\end{center}
	\begin{mcq}(4)
		\item $k_X<k_Y$.
		\item $k_X\le k_Y$.
		\item $k_X=k_Y$.
		\item $k_X>k_Y$.
	\end{mcq}
}
\hideall{
\textbf{Đáp án D.}\\
Với cùng độ lớn lực đàn hồi, lò xo Y dãn nhiều hơn lò xo X, do đó:
$$k_X>k_Y.$$
}

\end{enumerate}