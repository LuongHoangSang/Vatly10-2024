\let\lesson\undefined
\newcommand{\lesson}{\phantomlesson{Bài 17.}}
\setcounter{section}{2}
\section{Trắc nghiệm}
\ANSMCQ
{	\begin{center}
		\begin{tabular}{|m{2.8em}|m{2.8em}|m{2.8em}|m{2.8em}|m{2.8em}|m{2.8em}|m{2.8em}|m{2.8em}|m{2.8em}|m{2.8em}|}
			\hline
			1.B  & 2.B  & 3.D  & 4.D  & 5.A  & 6.A  & 7.A  & 8.D  & 9.B  & 10.B  \\
			\hline
			11.B  & 12. B  & 13.B  & 14.C  & 15.A  & 16.B  & 17.A  & 18.B  & 19.B  & 20.C  \\
			\hline
			21.D  & 22.A  & 23.D  & 24.D  & 25.B  &   &   &   &   &  \\
			\hline
		\end{tabular}
	\end{center}
}
\begin{enumerate}[label=\bfseries Câu \arabic*:, leftmargin=1.5cm]
	\item \mkstar{2}
	
	{
		Động năng của một vật khối lượng $m$, chuyển động với vận tốc $v$ là
		\begin{mcq}(4)
			\item $W_\text{đ} = \dfrac{1}{2}mv$.
			\item $W_\text{đ} = \dfrac{1}{2}mv^2$.
			\item $W_\text{đ} = mv^2$.
			\item $W_\text{đ} = 2mv^2$.
		\end{mcq}
	}
	
	\hideall
	{	
		\textbf{Đáp án: B.}
		
		Động năng của một vật khối lượng $m$, chuyển động với vận tốc $v$ là $W_\text{đ} = \dfrac{1}{2}mv^2$.
	}
	\item \mkstar{2}
	
	{
		Chọn phát biểu \textbf{sai}. Động năng của một vật không đổi khi vật chuyển động
		\begin{mcq}(2)
			\item tròn đều.
			\item với gia tốc không đổi.
			\item với vận tốc không đổi.
			\item thẳng đều.
		\end{mcq}
	}
	
	\hideall
	{	
		\textbf{Đáp án: B.}
		
		Khi vật chuyển động có gia tốc thì vận tốc thay đổi, do đó động năng thay đổi.
	}
	\item \mkstar{2}
	
	
	{
		Động năng của vật tăng khi
		\begin{mcq}
			\item vận tốc của vật có giá trị âm.
			\item vận tốc của vật có giá trị dương.
			\item các lực tác dụng lên vật sinh công âm.
			\item các lực tác dụng lên vật sinh công dương.
		\end{mcq}
	}
	
	\hideall
	{	
		\textbf{Đáp án: D.}
		
		Các lực tác dụng lên vật sinh công dương thì $v$ tăng, do đó động năng tăng.
	}
	\item \mkstar{2}
	
	
	{
		Một vật có khối lượng $\SI{1.5}{kg}$ chuyển động với vận tốc $\SI{5}{m/s}$. Động năng của vật bằng
		\begin{mcq}(4)
			\item $\SI{7.5}{J}$.
			\item $\SI{3.75}{J}$.
			\item $\SI{37.5}{J}$.
			\item $\SI{18.75}{J}$.
		\end{mcq}
	}
	
	\hideall
	{	
		\textbf{Đáp án: D.}
		
		Động năng của vật: $W_\text{đ} = \dfrac{1}{2}mv^2 = \SI{18.75}{J}$.
	}
	\item \mkstar{2}
	
	
	{
		Một vận động viên có khối lượng $\SI{65}{kg}$ chạy đều hết quãng đường $\SI{200}{m}$ trong thời gian 40 giây. Động năng của vận động viên đó là
		\begin{mcq}(4)
			\item $\SI{812.5}{J}$. 
			\item $\SI{1625}{J}$.
			\item $\SI{325}{J}$.
			\item $\SI{180}{J}$.
		\end{mcq}
	}
	
	\hideall
	{	
		\textbf{Đáp án: A.}
		
		Vận tốc của vận động viên:
		$$v=\dfrac{s}{t} = \SI{5}{m/s}.$$
		
		Động năng của vận động viên:
		$$W_\text{đ} = \dfrac{1}{2} mv^2 = \SI{812.5}{J}.$$
	}
	\item \mkstar{2}
	
	
	{
		Khi vận tốc của vật tăng gấp đôi thì
		\begin{mcq}(2)
			\item động lượng của vật tăng gấp đôi.
			\item gia tốc của vật tăng gấp đôi.
			\item động năng của vật tăng gấp đôi.
			\item thế năng của vật tăng gấp đôi.
		\end{mcq}
	}
	
	\hideall
	{	
		\textbf{Đáp án: A.}
		
		Khi vận tốc của vật tăng gấp đôi thì động lượng của vật tăng gấp đôi, động năng của vật tăng gấp 4.
	}
	\item \mkstar{2}
	
	
	{
		Một vật có khối lượng $\SI{250}{g}$ đang di chuyển với tốc độ $\SI{10}{m/s}$. Động năng của vật này là
		\begin{mcq}(4)
			\item $\SI{12.5}{J}$.
			\item $\SI{25}{J}$.
			\item $\SI{12500}{J}$.
			\item $\SI{25000}{J}$.
		\end{mcq}
	}
	
	\hideall
	{	
		\textbf{Đáp án: A.}
		
		Động năng của vật:
		$$W_\text{đ} = \dfrac{1}{2} mv^2  =\SI{12.5}{J}.$$
	}
	\item \mkstar{2}
	
	
	{
		Độ biến thiên động năng của một vật chuyển động bằng
		\begin{mcq}
			\item công của lực ma sát tác dụng lên vật.
			\item công của lực thế (ví dụ: trọng lực) tác dụng lên vật.
			\item công của trọng lực tác dụng lên vật.
			\item công của các lực tác dụng lên vật.
		\end{mcq}
	}
	
	\hideall
	{	
		\textbf{Đáp án: D.}
		
		Độ biến thiên động năng của một vật chuyển động bằng công của các lực tác dụng lên vật.
	}
	\item \mkstar{2}
	
	
	{
		Dạng năng lượng tương tác giữa Trái Đất và vật là
		\begin{mcq}(2)
			\item động năng.
			\item thế năng trọng trường.
			\item thế năng đàn hồi.
			\item cơ năng.
		\end{mcq}
	}
	
	\hideall
	{	
		\textbf{Đáp án: B.}
		
		Dạng năng lượng tương tác giữa Trái Đất và vật là thế năng trọng trường.
	}
	\item \mkstar{2}
	
	
	{
		Thế năng trọng trường của một vật
		\begin{mcq}
			\item luôn dương vì độ cao của vật luôn dương. 
			\item có thể âm, dương hoặc bằng không.
			\item không thay đổi nếu vật chuyển động thẳng đều.
			\item không phụ thuộc vào vị trí của vật.
		\end{mcq}
	}
	
	\hideall
	{	
		\textbf{Đáp án: B.}
		
		Thế năng trọng trường của một vật có thể âm, dương hoặc bằng không.
	}
	\item \mkstar{2}
	
	
	{
		Đơn vị nào \textbf{không} phải đơn vị của động năng?
		\begin{mcq}(4)
			\item J. 
			\item N$\cdot$s.
			\item N$\cdot$m.
			\item N$\cdot$m$^2$/s$^2$.
		\end{mcq}
	}
	
	\hideall
	{	
		\textbf{Đáp án: B.}
		
	}
	
	\item \mkstar{2}
	
	
	{Động năng là đại lượng:
		\begin{mcq}
			\item Vô hướng, dương, âm hoặc bằng không.
			\item Vô hướng, có thể dương hoặc bằng không.
			\item Vectơ, luôn dương.
			\item Vectơ, có thể dương hoặc bằng không
		\end{mcq}
	}
	
	\hideall
	{	
		\textbf{Đáp án: B.}
	}
	\item \mkstar{2}
	
	
	{Một vật đang chuyển động với vận tốc $v$. Nếu hợp lực tác dụng vào vật triệt tiêu thì động năng của vật:
		\begin{mcq}(2)
			\item giảm theo thời gian.
			\item không thay đổi.
			\item tăng theo thời gian.
			\item triệt tiêu.
		\end{mcq}
	}
	
	\hideall
	{	
		\textbf{Đáp án: B.}
	}
	\item \mkstar{2}
	
	
	{Một ô tô khối lượng $\SI{1000}{kg}$ chuyển động với vận tốc $\SI{72}{km/h}$. Động năng của ô tô có giá trị:
		\begin{mcq} (4)
			\item $\xsi{10^5}{J}.$
			\item $\text{15,92}\xsi{\cdot 10^5}{J}.$
			\item $\xsi{2\cdot 10^5}{J}.$
			\item $\text{51,84}\xsi{\cdot 10^5}{J}.$
		\end{mcq}
	}
	
	\hideall
	{	
		\textbf{Đáp án: C}
		
		Đổi $\SI{72}{km/h} = \SI{20}{m/s}.$
		
		Động năng của ô tô
		
		$$W_\text{đ} = \dfrac{1}{2}mv^2 = \xsi{2\cdot 10^5}{J}.$$
	}
	\item \mkstar{2}
	
	
	{
		Một ô tô có khối lượng 2 tấn đang chuyển động với vận tốc $\SI{36}{km/h}$. Động năng của ô tô là
		\begin{mcq}(4)
			\item $\xsi{10\cdot 10^4}{J}.$
			\item $\xsi{10^3}{J}.$
			\item $\xsi{20\cdot 10^4}{J}.$
			\item $\text{2,6}\xsi{10^6}{J}.$
		\end{mcq}
	}
	
	\hideall
	{	
		\textbf{Đáp án: A.}
		
		Đổi $\SI{36}{km/h} = \SI{10}{m/s}$, 2 tấn = $\SI{2000}{kg}.$
		
		Động năng của ô tô là
		
		$$W_\text{đ} = \dfrac{1}{2}mv^2 = \xsi{10\cdot 10^4}{J}.$$
	}
	\item \mkstar{2}
	
	
	{Một vật có khối lượng $m = \SI{4}{kg}$ và động năng $\SI{18}{J}$. Khi đó vận tốc của vật là
		\begin{mcq}(4)
			\item $\SI{9}{m/s}.$
			\item $\SI{3}{m/s}.$
			\item $\SI{6}{m/s}.$
			\item $\SI{12}{m/s}.$
		\end{mcq}
	}
	
	\hideall
	{	
		\textbf{Đáp án: B.}
		
		Vận tốc của vật
		
		$$ v = \sqrt{\dfrac{2W_\text{đ}}{m}} = \SI{3}{m/s}.$$ 
	}
	\item \mkstar{2}
	
	
	{Một vật nằm yên có thể có:
		\begin{mcq}(4)
			\item Thế năng.
			\item Vận tốc.
			\item Động năng.
			\item Động lượng.
		\end{mcq}
	}
	
	\hideall
	{	
		\textbf{Đáp án: A.}
	}
	\item \mkstar{2}
	
	
	{Thế năng của một vật \textbf{không} phụ thuộc vào (xét vật rơi trong trọng trường)
		\begin{mcq}(2)
			\item Vị trí vật.
			\item Vận tốc vật.
			\item Khối lượng vật.
			\item Độ cao.
		\end{mcq}
	}
	
	\hideall
	{	
		\textbf{Đáp án: B.}
	}
	\item \mkstar{2}
	
	
	{Nếu khối lượng vật tăng gấp 2 lần, vận tốc vật giảm đi một nửa thì
		\begin{mcq}(2)
			\item động năng của vật không đổi.
			\item động năng giảm 2 lần.
			\item động năng tăng 2 lần.
			\item động năng bằng 0.
		\end{mcq}
	}
	
	\hideall
	{	
		\textbf{Đáp án: B.}
		
		Ta có:
		
		$$\dfrac{W_{\text{đ}_1}}{W_{\text{đ}_2}} = \dfrac{1}{2}.$$
		
		Động năng giảm 2 lần.
	}
	\item \mkstar{2}
	
	
	{Chỉ ra câu \textbf{sai} trong các phát biểu sau:
		\begin{mcq}
			\item Thế năng của một vật có tính tương đối. Thế năng tại mỗi vị trí có thể có giá trị khác nhau tùy theo cách chọn gốc thế năng.
			\item Động năng của một vật chỉ phụ thuộc vào khối lượng và vận tốc của vật. Thế năng chỉ phụ thuộc vị trí tương đối giữa các phần của hệ với điều kiện lực tương tác trong hệ là lực thế.
			\item Công của trọng lực luôn làm giảm thế năng nên công của trọng lực luôn dương.
			\item Thế năng của quả cầu dưới tác dụng của lực đàn hồi cũng là thế năng.
		\end{mcq}
	}
	
	\hideall
	{	
		\textbf{Đáp án: C.}
		
		A, B, D – đúng
		
		C – sai vì: Không phải lúc nào công của trọng lực cũng luôn dương.
	}
	\item \mkstar{2}
	
	
	{Thế năng của vật nặng $\SI{5}{kg}$ ở đáy 1 giếng sâu $\SI{10}{m}$ so với mặt đất tại nơi có gia tốc $g = \SI{10}{m/s}^2$ là bao nhiêu? Chọn gốc thế năng tại mặt đất.
		\begin{mcq}(4)
			\item $-\SI{100}{J}.$
			\item $\SI{100}{J}.$
			\item $\SI{500}{J}.$
			\item $-\SI{500}{J}.$
		\end{mcq}
	}
	
	\hideall
	{	
		\textbf{Đáp án: D.}
		
		Thế năng của vật
		
		$$W_\text{t} = mgz = -\SI{500}{J}.$$
		
		Mang dấu (-)  vì nó nằm bên dưới mặt đất.
	}
	\item \mkstar{2}
	
	
	{Một thang máy có khối lượng 1 tấn chuyển động từ tầng cao nhất cách mặt đất $\SI{100}{m}$ xuống tầng thứ 10 cách mặt đất $\SI{40}{m}$. Lấy $g = \SI{10}{m/s}^2$. Nếu chọn gốc thế năng tại tầng 10, thì thế năng của thang máy ở tầng cao nhất là
		\begin{mcq}(4)
			\item $\SI{588}{kJ}$.
			\item $\SI{392}{kJ}$.
			\item $\SI{980}{kJ}$.
			\item $\SI{598}{kJ}$.
		\end{mcq}
	}
	
	\hideall
	{	
		\textbf{Đáp án: A.}
		
		Chọn gốc thế năng tại tầng 10.
		
		Độ cao của vật khi ở tầng cao nhất so với mốc thế năng bằng $z = 100 - 40 = \SI{60}{m}$ 
		
		$$W_\text{t} = mgz = \SI{588000}{J} = \SI{588}{kJ}.$$
	}
	\item \mkstar{2}
	
	
	{Trong các vật sau, vật nào \textbf{không} có động năng?
		\begin{mcq}
			\item Hòn bi lăn trên sàn nhà.
			\item Viên đạn đang bay.
			\item Máy bay đang bay.
			\item Hòn bi nằm yên trên mặt sàn.
		\end{mcq}
	}
	
	\hideall
	{	
		\textbf{Đáp án: D.}
		
		Động năng là năng lượng vật có được do chuyển động.
	}
	\item \mkstar{2}
	
	
	{Một vật được ném thẳng đứng từ dưới lên cao. Trong quá trình chuyển động của vật thì
		\begin{mcq}
			\item Thế năng của vật giảm, trọng lực sinh công dương.
			\item Thế năng của vật giảm, trọng lực sinh công âm.
			\item Thế năng của vật tăng, trọng lực sinh công dương.
			\item Thế năng của vật tăng, trọng lực sinh công âm. 
		\end{mcq}
	}
	
	\hideall
	{	
		\textbf{Đáp án: D.}
		
		Khi một vật được ném lên, độ cao của vật tăng dần nên thế năng tăng.
		
		Trong quá trình chuyển động của vật từ dưới lên, trọng lực luôn hướng ngược chiều chuyển động nên nó là lực cản, do đó trọng lực sinh công âm.
	}
	\item \mkstar{2}
	
	
	{Thế năng trọng trường là đại lượng: 
		\begin{mcq}
			\item Vô hướng, có thể dương hoặc bằng không. 
			\item Vô hướng, có thể âm, dương hoặc bằng không. 
			\item Véctơ cùng hướng với véc tơ trọng lực. 
			\item Véctơ có độ lớn luôn dương hoặc bằng không. 
		\end{mcq}
	}
	
	\hideall
	{	
		\textbf{Đáp án: B.}
		
		Ta có, thế năng hấp dẫn là đại lượng vô hướng, có thể âm, dương hoặc bằng 0.
	}
\end{enumerate}

\section{Tự luận}
\begin{enumerate}[label=\bfseries Câu \arabic*:, leftmargin=1.5cm]
	\item \mkstar{2}
	
	
	{
		Thả một quả bóng từ độ cao $h$ xuống sàn nhà. Động năng của quả bóng được chuyển hóa thành những dạng năng lượng nào ngay khi quả bóng chạm vào sàn nhà?
	}
	
	\hideall
	{	
		Thả một quả bóng từ độ cao h xuống sàn nhà. Ngay khi quả bóng chạm vào sàn nhà, chủ yếu động năng của quả bóng chuyển hóa thành thế năng làm quả bóng nảy lên, có một phần nhỏ động năng của quả bóng được chuyển hóa thành nhiệt năng (làm nóng quả bóng và mặt sàn) và năng lượng âm thanh (phát ra tiếng bụp ngay khi bóng chạm đất).
		
		
	}
		
		\item \mkstar{2}
		
		
		{
			Khi đang bay, năng lượng của thiên thạch tồn tại dưới dạng nào? Tại sao năng lượng của thiên thạch lại rất lớn so với năng lượng của các vật thường gặp? Khi va vào Trái Đất, năng lượng của thiên thạch được chuyển hóa thành những dạng năng lượng nào?
		}
		
		\hideall
		{	
			- Khi đang bay, năng lượng của thiên thạch tồn tại chủ yếu dưới dạng động năng và thế năng trọng trường, ngoài ra còn có quang năng, nhiệt năng.
			
			- Năng lượng của thiên thạch rất lớn so với năng lượng của các vật thường gặp vì:
			
			+ Thiên thạch có khối lượng lớn.
			
			+ Thiên thạch di chuyển với tốc độ lớn $\Rightarrow$ có động năng lớn.
			
			+ Khoảng cách từ thiên thạch tới Trái Đất rất lớn $\Rightarrow$ có thế năng trọng trường rất lớn.
			
			- Khi va vào Trái Đất, năng lượng của thiên thạch được chuyển hóa thành động năng của các mảnh vỡ, tạo thành các hố lõm trên bề mặt Trái Đất.
		}
		
	\item \mkstar{2}
	
	
	{
		Dùng lực để nâng vật có khối lượng $\SI{180}{g}$ thẳng đứng lên cao một đoạn $\SI{50}{cm}$. Tìm công của trọng lực trong quá trình trên. Lấy $g=\SI{10}{m/s^2}$.
	}
	
	\hideall
	{	
		Công của trọng lực:
		$$A_P = mgh\cos(180^\circ) = -mgh = \SI{-0.9}{J}.$$
	}
	
	\item \mkstar{2}
	
	
	{
		Một vận động viên quần vợt thực hiện cú giao bóng kỉ lục, quả bóng đạt tới tốc độ $\SI{196}{km/h}$. Biết khối lượng quả bóng là $\SI{60}{g}.$ Tính động năng của quả bóng.
	}
	
	\hideall
	{	
		Động năng của quả bóng được tính theo công thức:
		
		$$W_\text{đ} =\dfrac{1}{2}mv^2 \approx \SI{89}{J}.$$
	}
	
	\item \mkstar{2}
	
	
	{
		Máy đóng cọc có đầu búa nặng 0,5 tấn, được nâng lên độ cao $\SI{10}{m}$ so với mặt đất. Tính thế năng của đầu búa. Lấy $g = \SI{9,8}{m/s}^2$.
	}
	
	\hideall
	{	
		Thế năng của đầu búa
		
		$$W_\text{t} = mgh = \SI{49000}{J} = \SI{49}{kJ}.$$
	}
	
	
	
	\item \mkstar{2}
	
	
	{
		Một mũi tên nặng $\SI{48}{g}$ đang chuyển động với tốc độ $\SI{10}{m/s}$. Tìm động năng của mũi tên.
	}
	
	\hideall
	{	
		Động năng của mũi tên
		
		$$W_\text{đ} = \dfrac{1}{2}mv^2 =\SI{2,4}{J}.$$
	}
	\item \mkstar{2}
	
	
	{
		Phan - Xi - Păng là ngọn núi cao nhất trong ba nước Việt Nam, Lào, Campuchia, được mệnh danh là "nóc nhà Đông Dương". Lên đến đỉnh núi cao $\SI{3147}{m}$ này là ước mơ của nhiều bạn trẻ. Tính thế năng của người leo núi có khối lượng $\SI{70}{kg}$ khi leo lên đến đỉnh núi Phan - Xi - Păng. Lấy $g = \SI{9,8}{m/s}^2.$
	}
	
	\hideall
	{	
		Chọn mốc tính thế năng tại chân đỉnh núi.
		
		Thế năng của người ở đỉnh núi:
		
		$$W_\text{t} = mgh = \SI{2158842}{J}.$$
	}
	\item \mkstar{2}
	
	
	{
		So sánh động năng của ô tô $\SI{1000}{kg}$ chuyển động với tốc độ $\SI{4}{m/s}$ và động năng của xe máy nặng $\SI{100}{kg} $ chuyển động với tốc độ $\SI{15}{m/s}$.
	}
	
	\hideall
	{	
		Động năng của ô tô:
		
		$$W_{\text{đ}_1} = \dfrac{1}{2}m_1v_1^2 = \SI{8000}{J}.$$
		
		Động năng của xe máy:
		
		$$W_{\text{đ}_2} = \dfrac{1}{2}m_2v_2^2 = \SI{11250}{J}.$$
		
		Động năng của xe máy lớn hơn động năng của ô tô.
	}
	
	\item \mkstar{2}
	
	
	{
		Một ôtô có khối lượng $\SI{1100}{kg}$ đang chạy với vận tốc $\SI{24}{m/s}$. Độ biến thiên động năng của ôtô bằng bao nhiêu khi vận tốc sau khi hãm là $\SI{10}{m/s}$?
		
		
	}
	
	\hideall
	{	
		Độ biến thiên động năng của ô tô:
		$$\Delta W_\text{đ} = \dfrac{1}{2}mv_2^2 - \dfrac{1}{2}mv_1^2 = - \SI{261800}{J}.$$
	}
	
	\item \mkstar{3}
	
	
	{
		Một vật nhỏ $m$ được truyền vận tốc đầu $v_0=\SI{5}{m/s}$ tại A để vật trượt trên mặt phẳng ngang $\text{AB} = \SI{3}{m}$. Hệ số ma sát giữa vật và mặt ngang là $\mu=\SI{0.2}{}$. Lấy $g=\SI{10}{m/s^2}$. Tính vận tốc $v$ của vật tại B.
	}
	
	\hideall
	{	
		Áp dụng định lý động năng:
		$$A_\text{ms} = \dfrac{1}{2}mv^2 - \dfrac{1}{2}mv_0^2 \Rightarrow -F_\text{ms}s = \dfrac{1}{2}mv^2 - \dfrac{1}{2}mv_0^2 \Rightarrow -\mu g s = \dfrac{1}{2}v^2 - \dfrac{1}{2}v_0^2 \Rightarrow v = \SI{3.6}{m/s}.$$
	}

\item \mkstar{3}


{
	Xe ô tô khối lượng 2 tấn bắt đầu khởi hành trên đường thẳng nằm ngang, đi được $\SI{50}{m}$ thì đạt được vận tốc $\SI{54}{km/h}$. Lực kéo của động cơ bằng $\SI{10000}{N}$. Lấy $g=\SI{10}{m/s^2}$.
	\begin{enumerate}[label=\alph*)]
		\item Dùng định lý động năng, tính công của lực ma sát.
		\item Tính độ lớn lực ma sát tác dụng lên xe trong đoạn đường trên.
	\end{enumerate}
}

\hideall
{	
	\begin{enumerate}[label=\alph*)]
		\item Dùng định lý động năng, tính công của lực ma sát.
		
		Áp dụng định lý động năng:
		$$A_F + A_\text{ms} = \dfrac{1}{2}mv^2 - 0 \Rightarrow A_\text{ms} = \dfrac{1}{2}mv^2 - A= \dfrac{1}{2} mv^2 - Fs= \SI{-275000}{J}.$$
		\item Tính độ lớn lực ma sát tác dụng lên xe trong đoạn đường trên.
		
		Ta có $A_\text{ms} = -F_\text{ms} s \Rightarrow F_\text{ms} = \SI{5500}{N}$.
	\end{enumerate}
}

\item \mkstar{3}


{
	Một viên đạn có khối lượng $\SI{14}{g}$ bay theo phương ngang với vận tốc $\SI{400}{m/s}$ xuyên qua tấm gỗ dày $\SI{5}{cm}$, sau khi xuyên qua gỗ, đạn có vận tốc $\SI{120}{m/s}$. Tính lực cản trung bình của tấm gỗ tác dụng lên viên đạn.
}

\hideall
{	
	Độ biến thiên động năng của viên đạn khi xuyên qua tấm gỗ là:
	$$\Delta W_\text{đ} = \dfrac{1}{2}mv^2_2 - \dfrac{1}{2}mv_1^2 = - \SI{1019.2}{\joule}.$$
	
	Theo định lí biến thiên động năng:
	
	$$A_\text{c} = \Delta W_\text{đ} = F_\text{c}S \Rightarrow F_\text{c} = -\SI{20384}{\newton}.$$
	
}

	\item \mkstar{3}


{
	Xe ô tô khối lượng 1 tấn bắt đầu khởi hành trên đường thẳng nằm ngang, đi được $\SI{50}{m}$ thì đạt được vận tốc $\SI{36}{km/h}$. Cho hệ số ma sát giữa bánh xe và mặt đường là $\SI{0.05}{}$. Lấy $g=\SI{10}{m/s^2}$. Dùng định lý động năng tính công của lực kéo của động cơ, từ đó suy ra độ lớn lực kéo của động cơ.
}

\hideall
{	
	Áp dụng định lý động năng:
	$$A_F + A_\text{ms} = \dfrac{1}{2}mv^2 - 0 \Rightarrow A_F = \dfrac{1}{2}mv^2 - A_\text{ms} = \dfrac{1}{2}mv^2 - (-F_\text{ms}s) = \SI{75000}{J}.$$
	
	Lực kéo của động cơ:
	$$A_F=Fs \Rightarrow F = \SI{1500}{N}.$$
}

\item \mkstar{3}


{
	Một vật có khối lượng $\SI{10}{kg}$ đang chuyển động với tốc độ $\SI{5}{m/s}$ trên mặt bàn nằm ngang. Do có ma sát, vật chuyển động chậm dần đều và đi được $\SI{1}{m}$ thì dừng lại. Tính hệ số ma sát giữa vật và mặt bàn. Lấy gia tốc trọng trường $g = \SI{9,8}{m/s}^2.$
}

\hideall
{	
	Động năng của vật khi đang chuyển động:
	
	$$W_\text{đ} = \dfrac{1}{2}mv^2 = \SI{125}{J}.$$
	
	Vật dừng lại nên động năng lúc sau bằng 0.
	
	Vật chuyển động chậm dần đều do ma sát nên:
	
	$$0- W_\text{đ} = A_{\text{F}_\text{ms}} \Rightarrow A_{\text{F}_\text{ms}} = - \SI{125}{J}.$$
	
	Lực ma sát ngược chiều chuyển động
	
	$$A_{\text{F}_\text{ms}} = Fs\cos 180^\circ \Rightarrow F_\text{ms} = \SI{125}{N}.$$
	
	Hệ số ma sát
	
	$$\mu = \dfrac{F_\text{ms}}{mg} \approx \SI{1,28}{m/s}^2.$$
}
	
\end{enumerate}