% ==============================================
%		delcaration of default data  
% ==============================================
\graphicspath{{../figs/}{../extra/}{../icon/}} % các thư mục chứa hình ảnh 

\definecolor{obcolor}{HTML}	
%		{ffce00}	% Mathematics
		{aa98ff}	% Physics
%		{99bcff}	% Chemistry
%		{c4e538}	% Biology
%		{f37676}	% English
% --- testing
\newcommand{\PhysicsA}{
	\newcommand{\sublogo}{Physics_A.png}
	\newcommand{\subname}{Vật lý}	
	\newcommand{\substyle}{Chuyên đề}
}
\newcommand{\PhysicsL}{
	\newcommand{\sublogo}{Physics-L}
	\newcommand{\subname}{Vật lý}	
	\newcommand{\substyle}{Lý thuyết}
}
\newcommand{\Physics}{
	\newcommand{\sublogo}{Physics_A.png}
	\newcommand{\subname}{Vật lý}	
	\newcommand{\substyle}{}
}
\newcommand{\ChemistryA}{
	\newcommand{\sublogo}{Chemistry_A.png}
	\newcommand{\subname}{Hóa học}	
	\newcommand{\substyle}{Dạng bài}	
}
\newcommand{\ChemistryL}{
	\newcommand{\sublogo}{Chemistry_L.png}
	\newcommand{\subname}{Hóa học}	
	\newcommand{\substyle}{Lý thuyết}
}
\newcommand{\MathematicsA}{
	\newcommand{\sublogo}{Mathematics_A.png}
	\newcommand{\subname}{Toán học}	
	\newcommand{\substyle}{Dạng bài}
}
\newcommand{\MathematicsL}{
	\newcommand{\sublogo}{Mathematics_L.png}
	\newcommand{\subname}{Toán học}	
	\newcommand{\substyle}{Lý thuyết}
}
\newcommand{\BiologyA}{
	\newcommand{\sublogo}{Biology_A.png}
	\newcommand{\subname}{Sinh học }
	\newcommand{\substyle}{Dạng bài}		
}
\newcommand{\BiologyL}{
	\newcommand{\sublogo}{Biology_L.png}
	\newcommand{\subname}{Sinh học}	
	\newcommand{\substyle}{Lý thuyết}
}
% ========== color =============

\colorlet{pagecol}{white}	

\colorlet{headerbcol}{obcolor}
\colorlet{headerdcol}{black}
\colorlet{headertcol}{black}

\colorlet{footercol}{obcolor}
\colorlet{footertextcol}{black}

\colorlet{secdcol}{black}
\colorlet{secbcol}{obcolor}
\colorlet{sectcol}{black}
\colorlet{secnbcol}{white}
\colorlet{secntcol}{black}

\colorlet{ssecdcol}{black}
\colorlet{ssectcol}{black}
\colorlet{ssecnbcol}{obcolor}
\colorlet{ssecntcol}{black}

\colorlet{dangtbcol}{obcolor!15}

\colorlet{vidufcol}{obcolor}
\colorlet{vidutbcol}{obcolor}
\colorlet{viduttcol}{black}
\colorlet{vidumbcol}{pagecol}
\colorlet{vidumtcol}{black}
\colorlet{vidulbcol}{pagecol}
\colorlet{vidultcol}{black}

\colorlet{baitapfcol}{obcolor}
\colorlet{baitaptbcol}{obcolor}
\colorlet{baitapttcol}{black}
\colorlet{baitapmbcol}{obcolor!15}
\colorlet{baitapmtcol}{black}
\colorlet{baitaplbcol}{pagecol}
\colorlet{baitapltcol}{black}

\colorlet{stardrawcol}{white}
\colorlet{starmkdcol}{white}
\colorlet{starempcol}{vidutbcol}

\definecolor{manatbcol}{HTML}{05C88C}
\colorlet{manatfcol}{manatbcol}
\colorlet{manafcol}{manatbcol!25}
\colorlet{manambcol}{manafcol}

\definecolor{luuytbcol}{HTML}{546de5}
\colorlet{luuytfcol}{luuytbcol}
\colorlet{luuyfcol}{luuytbcol!25}
\colorlet{luuymbcol}{luuyfcol}

\definecolor{pphaptbcol}{HTML}{f45b5b}
\colorlet{pphaptfcol}{pphaptbcol}
\colorlet{pphapfcol}{pphaptbcol!25}
\colorlet{pphapmbcol}{pphapfcol}

\pagecolor{pagecol}

% --------- stars color for white background ----
\newcommand{\whiteBGstarBegin}{
	\colorlet{stardrawcol}{obcolor}
	\colorlet{starmkdcol}{obcolor}
	\colorlet{starempcol}{white}
}
\newcommand{\whiteBGstarEnd}{
	\colorlet{stardrawcol}{white}
	\colorlet{starmkdcol}{white}
	\colorlet{starempcol}{obcolor}
}
% ==============================================
% 		format the page with abitrary page length
% ==============================================
\edef\svparindent{\the\parindent}
\standaloneenv{page}
\makeatletter\newenvironment{fixedpagewidth}[1]
{\begin{page}\begin{minipage}{\dimexpr#1-\sa@border@left-\sa@border@right\relax}%
			\parindent\svparindent\relax}
		{\end{minipage}\end{page}}
\makeatother
\newlength{\mypagelength}
\setlength{\mypagelength}{485pt}

% ==============================================
% 		format for sections
% ==============================================
% --- Định cấu trúc 
\setcounter{secnumdepth}{4} % đánh số đến cấp thứ 4 của chỉ mục (subsubsection sẽ được đánh số)
% --- lesson
\newcounter{mychap}[part]
\newcommand{\mychap}[1]{
	\begin{flushleft}
		\refstepcounter{mychap}
		\Large\sffamily\bfseries Chương \themychap\ \\ 
		\LARGE #1
		\vspace{1em}
	\end{flushleft}
}
% --- section


\definecolor{cor1}{HTML}{57C3CF}
\definecolor{cor2}{HTML}{4EA8B1}
\newcommand\SecTitle[1]{%
	\begin{tikzpicture}
		\node (A) [rounded rectangle,minimum width=\textwidth, minimum height=25pt,color=sectcol,fill=secbcol,line width=1.5pt,draw=secdcol, text width=\textwidth-100pt,align=left] {\hspace{20pt}\parbox{\textwidth}{\large\textbf{\textsf{#1}}}};
		\tikzmath{coordinate \p;
			\p = (A.south)-(A.north); 
			\distAB = veclen(\px,\py)-1.6; 
		}
		\node (B) [rounded rectangle,rounded rectangle east arc=0pt,minimum width=0.1\textwidth, minimum height=\distAB pt,color=sectcol,fill=secnbcol,line width=1.5pt,draw=secdcol, text width=0.1\textwidth,align=left,anchor=west] at (A.west) {\hspace{20pt}\parbox{0.4\textwidth}{\large\textbf{\textsf{\thesection}}}};
	\end{tikzpicture}
}
\titleformat{\section}
{\normalfont\renewcommand\thesection{\Roman{section}}}
{}
{0em}
{\SecTitle{#1}}
\titlespacing{\section}{0pt}{20pt}{0pt}

\renewcommand{\thesubsection}{\arabic{subsection}}
\DeclareRobustCommand*\circledsubsec[1]{%
	\tikz[
	baseline=(char.base)
	]{%
		\node[
		shape=circle,
		draw,
		line width=1.5pt,
		inner sep=3pt,
		fill=ssecnbcol,
		%lightgray
		] (char) {\color{black}\bfseries\fontfamily{phv}\selectfont\thesubsection};%
	}%
}

\newcommand\SubSecTitle[1]{%
	\begin{enumerate}[leftmargin=24pt, nosep,label=\circledsubsec{\arabic*},ref=\arabic*]
		%\refstepcounter{subsection}
		%\setcounter{enumi}{subsection}
		\item \normalsize\textbf{\textsf{#1}}
	\end{enumerate}
}
\titleformat{\subsection}
{}
{}
{0em}
{\SubSecTitle{#1}}
\titlespacing{\subsection}{0cm}{10pt}{-5pt}


\newcommand\SubSubSecTitle[1]{%
	\begin{itemize}[leftmargin=13pt, nosep]
		\item [\tikz{\draw[draw=obcolor,fill=obcolor,line width=1.5pt] circle (3pt);}] \normalsize\textbf{\textsf{#1}}
	\end{itemize}
}
\titleformat{\subsubsection}
{}
{}
{0em}
{\SubSubSecTitle{#1}}
\titlespacing{\subsubsection}{0cm}{-5pt}{-5pt}

% =============================================


% --- định nghĩa môi trường mới - Dạng 
\newcounter{dang} % định nghĩa chỉ số cho Dạng 
[section] % chỉ số sẽ reset mỗi section 
\newenvironment % định nghĩa môi trường mới 
{dang} % tên môi trường 
[1] % số thành phần phải có 
{
	\refstepcounter{dang} % chỉ số tương ứng 
	\leavevmode
	\begin{center}
		\leavevmode \vspace{-0.6cm}
		\begin{tcolorbox}
			[
			bicolor
			,sidebyside
			,width=0.93\textwidth
			,lefthand width=2.7cm
			,arc=0.5cm
			%	,rounded corners
			,colback=dangtbcol
			,colbacklower=white
			,segmentation engine=path
			,segmentation style=
			{
				line width=1.5pt
				,solid
			}
			%						,borderline={0.3mm}{0.3mm}{black}
			]
			\large
			{
				\bf Mục tiêu \thedang:
			}
			\tcblower
			\centering\large
			{
				\color{white}\large\bfseries #1
			}
		\end{tcolorbox}
	\end{center}
	
} 

{
	
	\par
	\medskip	
}

% --- Định nghĩa lệnh tạo box Phương pháp giải 
\newcommand{\ppgiai}[1] %lệnh hộp (tóm tắt lý thuyết)
{	\begin{center}
		\vspace*{-0.3cm}
		\leavevmode 
		\begin{tcolorbox}
			[
			%standard jigsaw
			,enhanced
			,opacityback=0
			,opacityfill=1
			,attach boxed title to top center={yshift=-10pt,yshifttext=-10pt}
			,boxed title style=
			{
				size=small
				%height=26pt
				,boxrule=0pt
				,colframe=pphaptfcol
				,colback=pphaptbcol
				,arc=9pt
				,center title
				,overlay={
					\node[anchor=west] 
					at ([xshift=8pt] $ (frame.north west)$ )
					{\includegraphics[height=0.9cm]{icon-20}};}
			}
			,coltitle=black
			,title=\strut \hspace{40pt}\textbf{\textsf{Phương pháp giải}\hspace{10pt}}
			%						,opacityback=0
			,opacityframe=1
			,breakable
			,pad at break*=2mm
			,colback=pphapmbcol,
			,colframe=pphapfcol
			,width=\linewidth
			,before upper={\parindent15pt}
			
			%						,watermark color=blue!3!white
			%						,watermark text=\arabic{tcbbreakpart}
			]
			{
				#1
			}
		\end{tcolorbox}
	\end{center}
}

% --- Định nghĩa lệnh tạo box Manatips
\newcommand{\manatip}[1] %lệnh hộp (tóm tắt lý thuyết)
{	\begin{center}
		\vspace*{-0.3cm}
		\leavevmode 
		\begin{tcolorbox}
			[
			%standard jigsaw
			,enhanced
			,opacityback=0
			,opacityfill=1
			,attach boxed title to top center={yshift=-10pt,yshifttext=-10pt}
			,boxed title style=
			{
				size=small
				%height=26pt
				,boxrule=0pt
				,colframe=manatfcol
				,colback=manatbcol
				,arc=9pt
				,center title
				,overlay={
					\node[anchor=west] 
					at ([xshift=8pt] $ (frame.north west)$ )
					{\includegraphics[height=0.9cm]{manatip-logo}};}
			}
			,coltitle=black
			,title=\strut \hspace{40pt}\textbf{\textsf{Manatip}\hspace{10pt}}
			%						,opacityback=0
			,opacityframe=1
			,breakable
			,pad at break*=2mm
			,colback=manambcol,
			,colframe=manafcol
			,width=\linewidth
			,before upper={\parindent15pt}
			
			%						,watermark color=blue!3!white
			%						,watermark text=\arabic{tcbbreakpart}
			]
			{
				#1
			}
		\end{tcolorbox}
	\end{center}
}
% --- Định nghĩa lệnh tạo box Lưu ý khi dàn trang

% --- Định nghĩa lệnh tạo box Lưu ý 
\newcommand{\luuy}[1] %lệnh hộp (tóm tắt lý thuyết)
{	
	\vspace*{-0.3cm}
	\begin{center}
		\leavevmode 
		\begin{tcolorbox}
			[
			%standard jigsaw
			,enhanced
			,opacityback=0
			,opacityfill=1
			,attach boxed title to top center={yshift=-10pt,yshifttext=-10pt}
			,boxed title style=
			{
				size=small
				%height=26pt
				,boxrule=0pt
				,colframe=luuytfcol
				,colback=luuytbcol
				,arc=9pt
				,center title
				,overlay={
					\node[anchor=west] 
					at ([xshift=8pt] $ (frame.north west)$ )
					{\includegraphics[height=0.9cm]{luuy-logo}};}
			}
			,coltitle=black
			,title=\strut \hspace{40pt}\textbf{\textsf{Lưu ý}\hspace{10pt}}
			%						,opacityback=0
			,opacityframe=1
			,breakable
			,pad at break*=2mm
			,colback=luuymbcol,
			,colframe=luuyfcol
			,width=\linewidth
			,before upper={\parindent15pt}
			
			%						,watermark color=blue!3!white
			%						,watermark text=\arabic{tcbbreakpart}
			]
			{
				#1
			}
		\end{tcolorbox}
	\end{center}
}


% --- Tạo môi trường các đáp án trắc nghiệm 
\makeatletter
\@ifpackagelater{tasks}{2019/10/04}
{
	\NewTasksEnvironment[style=enumerate,label=\Alph*.,label-format={\bfseries},label-width=2ex,label-offset=1ex,item-indent=1.8cm]{mcq}[\item](1)
	% Code which runs if the package date is 2019/10/04 or later
}
{
	\NewTasks[style=enumerate,counter-format={\bfseries tsk[A].},label-width=2ex,label-offset=1.5ex,item-indent=1.8cm]{mcq}[\item](1)
	% Code which runs if the package date is older than 2019/10/04
}
\makeatother
%	\NewEnviron{mcq}[1][]
%		{
	% Misc. stuff to preceed the tasks env here
	%			\def\tempbegin
	%				{%\vspace{1cm}
		%					\begin{twopartasks}
			%				}%
		%					\expandafter\tempbegin\BODY
		%					\end{twopartasks}
	% Misc. stuff to follow
	%		}

% -- insert stars
\newcommand\score[2]{%
	\pgfmathsetmacro\pgfxa{#1 + 1}%
	\tikzstyle{scorestars}=[star, star points=5, star point ratio=2, draw=stardrawcol, inner sep=2pt, anchor=outer point 3]%
	\begin{tikzpicture}[baseline]
		\foreach \i in {1, ..., #2} {
			\pgfmathparse{\i<=#1 ? "starmkdcol" : "starempcol"}
			\edef\starcolor{\pgfmathresult}
			\draw (\i*3.5ex, 0ex) node[name=star\i, scorestars, fill=\starcolor]  {};
		}
	\end{tikzpicture}%
}
\newcommand{\mkstar}[1]{\protect\score{#1}{4}}


% --- Định nghĩa môi trường ví dụ 
\newcommand{\vidu}[3] % -- không đánh số, có lời giải 
{
	\begin{tcolorbox}[
		title=\textbf{\textsf{Ví dụ }}\hfill \mkstar{#1}
		,width=0.9\linewidth
		,grow to right by=0.1\textwidth
		%		,text width=0.9/textwidth
		,breakable
		,colbacktitle=vidutbcol
		,coltitle=viduttcol
		,colframe=vidufcol
		,colback=vidumbcol
		,boxrule=1.5pt
		]
		{#2}
		\tcblower 
		{#3}
	\end{tcolorbox}
	\vspace{5pt}
}

\newcommand{\viduon}[2] % không đánh số, không lời giải
{
	\begin{tcolorbox}[
		title=\textbf{\textsf{Ví dụ}}\hfill \mkstar{#1}
		,width=0.9\linewidth
		,grow to right by=0.1\textwidth
		%		,text width=0.9/textwidth
		,breakable
		,colbacktitle=vidutbcol
		,coltitle=viduttcol
		,colframe=vidufcol
		,colback=vidumbcol
		,boxrule=1.5pt
		]
		{#2}
	\end{tcolorbox}
	\vspace{5pt}
}

\newcounter{viduii}[dang] % chỉ số của ví dụ, reset khi bắt đầu dạng mới 
\newcommand{\viduii}[3] % có đánh số, có lời giải 
{
	\refstepcounter{viduii}
	\begin{tcolorbox}[
		title=\textbf{\textsf{Ví dụ \theviduii~}}\hfill \mkstar{#1}
		,width=0.9\linewidth
		,grow to right by=0.1\textwidth
		%		,text width=0.9/textwidth
		,breakable
		,colbacktitle=vidutbcol
		,coltitle=viduttcol
		,colframe=vidufcol
		,colback=vidumbcol
		,boxrule=1.5pt
		]
		{#2}
		\tcblower
		{#3}
	\end{tcolorbox}
	\vspace{5pt}
}


\newcommand{\viduin}[2] % có đánh số, không lời giải 
{
	
	\refstepcounter{viduii}
	\begin{tcolorbox}[
		title=\textbf{\textsf{Ví dụ \theviduii~}}\hfill \mkstar{#1}
		,width=0.9\linewidth
		,grow to right by=0.1\textwidth
		%		,text width=0.9/textwidth
		,breakable
		,colbacktitle=vidutbcol
		,coltitle=viduttcol
		,colframe=vidufcol
		,colback=vidumbcol
		,boxrule=1.5pt
		]
		{#2}
	\end{tcolorbox}
	\vspace{5pt}
}

% --- môi trường bài tập có đánh số 

\newcounter{baitapii}[section] % chỉ số của bài tập, reset khi bắt đầu dạng mới 
\newcommand{\baitapii}[3] % có đánh số, có lời giải 
{
	\refstepcounter{baitapii}
	\vspace*{5pt}
	\begin{tcolorbox}[
		title=\textbf{\textsf{Bài tập \thebaitapii~}}\hfill \mkstar{#1}
		,width=0.9\linewidth
		,grow to right by=0.1\textwidth
		,breakable
		,colbacktitle=baitaptbcol
		,coltitle=baitapttcol
		,colframe=baitapfcol
		,colback=baitapmbcol
		,boxrule=1.5pt
		]
		{#2}
		\tcblower
		{#3}
	\end{tcolorbox}
	\vspace{5pt}
}	
% --- các ký tự tạo thêm 
% --- ký hiệu song song 
\newcommand{\parallelsum} % tên lệnh tạo ký hiệu song song 
{
	{\mathbin{\!/\mkern-5mu/\!}}
}
\newcommand{\dpara}{\parallelsum}
% --- đồng nhất kí hiệu độ (đơn vị góc) thành ^\circ
\renewcommand{\ang}[1]{#1^\circ}
% --- ký hiệu suất điện động và công suất như sgk
% ký hiệu từ font calligra 
%	\DeclareFontFamily{U}{calligra}{}
%	\DeclareFontShape{U}{calligra}{m}{n}{<->callig15}{}
%	\newcommand{\calE} % lệnh tạo ký hiệu sđđ 
%		{
	%			{\!\!\text{\usefont{U}{calligra}{m}{n}\textbf{E}}\,\,}
	%		}
%	\newcommand{\calP} % lệnh tạo ký hiệu công suất 
%		{
	%			{\!\!\text{\usefont{U}{calligra}{m}{n}P}\,\,}
	%		}
% ký hiệu từ font Boondox 
\DeclareFontFamily{U}{BOONDOX-cal}{\skewchar\font=45 }
\DeclareFontShape{U}{BOONDOX-cal}{m}{n}{
	<-> s*[1.05] BOONDOX-r-cal}{}
\DeclareFontShape{U}{BOONDOX-cal}{b}{n}{
	<-> s*[1.05] BOONDOX-b-cal}{}
\DeclareMathAlphabet{\bdx}{U}{BOONDOX-cal}{m}{n}
\SetMathAlphabet{\bdx}{bold}{U}{BOONDOX-cal}{b}{n}
\DeclareMathAlphabet{\bbdx}{U}{BOONDOX-cal}{b}{n}
\newcommand{\calE}{\bdx{E}}
\newcommand{\calP}{\bdx{P}}
\newcommand{\ud}{\;\text{d}}
% lệnh siunit 
\newcommand{\xsi}[2]{\SI[parse-numbers=false]{#1}{#2}}

% --- định nghĩa môi trường định luật 
\newtheorem{thrPh}{Định luật}
\newtheorem*{thrPh*}{Định luật}
%--- hỗ trợ bảng 
\renewcommand{\theadfont}
{
	\normalfont\bfseries
} % làm ô trong bảng canh giữa + in đậm 
\newcommand{\nfhead}[1] % tên lệnh 
{
	\renewcommand{\theadfont}
	{
		\normalfont
	}
	\thead{#1}
	\renewcommand{\theadfont}
	{
		\normalfont\bfseries
	} 
} % làm ô trong bảng canh giữa 
% lưu ý, khi sử dụng \thead và \nfhead thì phải xuống dòng thủ công.

% --- tạo dòng trống 
\newcommand{\Pointilles}[1]{%
	\par\nobreak
	\noindent\rule{0pt}{\baselineskip}% Provides a larger gap between the preceding paragraph and the dots
	%\doublespacing
	\multido{}{#1}{\noindent\makebox[\linewidth]{\dotfill}\endgraf}% ... dotted lines ...
	%\onehalfspacing
	\bigskip% Gap between dots and next paragraph
}

\newcommand{\Linesfill}[1]{%
	\par\nobreak
	\noindent\rule{0pt}{\baselineskip}% Provides a larger gap between the preceding paragraph and the dots
	%\doublespacing
	\multido{}{#1}{\noindent\rule{\linewidth}{0.2pt}\endgraf}% ... dotted lines ...
	%\onehalfspacing
	\bigskip% Gap between dots and next paragraph
}	

\newcommand{\Blfill}[1]{%
	\par\nobreak
	\noindent\rule{0pt}{\baselineskip}% Provides a larger gap between the preceding paragraph and the dots
	%\doublespacing
	\multido{}{#1}{\noindent\rule{\linewidth}{0pt}\endgraf}% ... dotted lines ...
	%\onehalfspacing
	\bigskip% Gap between dots and next paragraph
}	

\newcommand{\phantomline}[2][b]{
	\ifx b#1 \Blfill{#2} \else
	\ifx d#1	\Pointilles{#2} \else
	\ifx l#1 \Linesfill{#2}
	\fi\fi\fi 
}

% --- các lệnh che các đoạn văn bản 
\newlength{\saveparindent}
\AtBeginDocument{\setlength{\saveparindent}{\parindent}}

\newsavebox{\mytext}

\newcommand{\dotshide}[1]{%
	\savebox{\mytext}{%
		\parbox[t]{\columnwidth}{
			\setlength{\parindent}{\saveparindent}
			#1\par\xdef\savedprevdepth{\the\prevdepth}
		}%
	}%
	\noindent
	\pgfmathparse{int(round(\the\dp\mytext/\the\baselineskip))}
	\Pointilles{\pgfmathresult}
	\par
	% restore \prevdepth to compute correctly the interline glue
	\prevdepth\savedprevdepth
}

\newcommand{\lineshide}[1]{%
	\savebox{\mytext}{%
		\parbox[t]{\columnwidth}{
			\setlength{\parindent}{\saveparindent}
			#1\par\xdef\savedprevdepth{\the\prevdepth}
		}%
	}%
	\noindent
	\pgfmathparse{int(round(\the\dp\mytext/\the\baselineskip))}
	\Linesfill{\pgfmathresult}
	\par
	% restore \prevdepth to compute correctly the interline glue
	\prevdepth\savedprevdepth
}

\newcommand{\blankhide}[1]{%
	\savebox{\mytext}{%
		\parbox[t]{\columnwidth}{
			\setlength{\parindent}{\saveparindent}
			#1\par\xdef\savedprevdepth{\the\prevdepth}
		}%
	}%
	\noindent
	\pgfmathparse{int(round(\the\dp\mytext/\the\baselineskip))}
	\Blfill{\pgfmathresult}
	\par
	% restore \prevdepth to compute correctly the interline glue
	\prevdepth\savedprevdepth
}

\newcommand{\hide}[2][b]{
	\ifx b#1 #2
	\fi
}

\newcommand{\bltext}[1]{#1}

% --- new commands workshop 
\DeclareSIUnit\minute{\textrm{phút}}
% 
\newcommand{\bai}[1]{\part{#1}}
\newcommand{\LO}[1]{\chapter{#1}}

% --- equation number
\renewcommand{\theequation}{\arabic{equation}}
% --- reduce space above and below equation 
%\setlength{\abovedisplayskip}{3pt}
%\setlength{\belowdisplayskip}{3pt}

% --- testing 
\newcommand{\hides}[1]{#1}
\setlength{\abovedisplayskip}{0pt}


\usepackage{booktabs} % Required for nicer horizontal rules in tables


\newcommand{\SGBegin}{
	\begin{fixedpagewidth}{\mypagelength}
		\hspace*{-10pt}
		\begin{tikzpicture}
			%\draw[draw=pagecol,fill=footercol,anchor=west] (0,0) rectangle (10pt,-20pt);
			\node[inner sep=0pt] (oblogo) at (0,0)
			{\includegraphics[height=20pt]{\sublogo}};
			\node [rectangle,rounded corners=8pt,	minimum height=16pt, color=headertcol,draw=headerdcol,fill=headerbcol,line width=1.3pt, text width=2cm,minimum width=2cm] (obname) at ($(oblogo.east)+(45pt,0)$) {\parbox{\textwidth}{\centering\small\textbf{\textsf{\subname}}}};%
			\node [rectangle,rounded corners=8pt,	minimum height=10pt, color=headertcol,draw=headerdcol,line width=1pt, text width=3cm,minimum width=3cm,draw opacity=0] (obname) at ($(obname.east)+(30pt,0)$) {\parbox{\textwidth}{\centering\small\textbf{\textsf{\substyle}}}};%	
		\end{tikzpicture}
		\vspace{1cm}
	}
	
	\newcommand{\SGEnd}{
		\begin{adjustwidth}{-42pt}{}
			\vspace{2cm}
			%		\fancyhfoffset[L]{-0.1\paperwidth}
			%		\fancyhfoffset[R]{-0.1\paperwidth}
			\begin{tikzpicture}
				\draw[draw=pagecol,fill=footercol] (0,0) rectangle (595pt,-47pt);
				\node[inner sep=0pt,anchor=east] (oblogo) at (80pt,-25pt) {\includegraphics[height=25pt]{\sublogo}};
				\node [rectangle,minimum height=10pt, color=footertextcol,draw=footercol,fill=footercol,line width=1pt, text width=15cm,minimum width=2cm,anchor=west] (obname) at ($(oblogo.east)+(15pt,0pt)$)   {\parbox{0.6\textwidth}{\small\textbf{\textsf{\chapname}}}};%
				\node[inner sep=0pt,anchor=east] (comlogo) at (443pt,-25pt) {\includegraphics[height=0.5cm]{Logo.png}};
			\end{tikzpicture}
			%	\end{tcolorbox}
		%		\vspace{20pt}
	\end{adjustwidth}
\end{fixedpagewidth}
}
% --- các lệnh bật / tắt dáp án
\newcommand{\AnswersOff}{
\def\anskey{0}
}
\newcommand{\AnswersOn}{
\def\anskey{1}
}	
\newcommand{\hideall}[1]{#1}
\renewcommand{\hideall}[1]{
\ifthenelse{\equal{\anskey}{0}}{}{#1}	
}
% ---- testing
\newcommand{\phantomlesson}[1]{#1}
\newcommand{\chapter}[2][]{
	\newcommand{\chapname}{#1}
	\begin{flushleft}
		\begin{minipage}[t]{\linewidth}
			\includegraphics[height=1cm]{hdht-logo.png}
			\hspace{0pt}	
			\sffamily\bfseries\large \lesson
			\begin{flushleft}
				\LARGE\bfseries #2
			\end{flushleft}
		\end{minipage}
	\end{flushleft}
	\vspace{1cm}
	\normalfont\normalsize
}

	% --- tạo hộp Tóm tắt lý thuyết  
\newcommand{\hops}[1] %lệnh hộp (tóm tắt lý thuyết)
{	
	\begin{flushright}
		\leavevmode 
		\begin{tcolorbox}
			[
			standard jigsaw
			,opacityback=0
			,opacityframe=1
			,breakable
			,pad at break*=2mm
			%						,colback=white!20!white,
			,colframe=black!70!white
			,width=\textwidth
			,before upper={\parindent15pt}
			%						,watermark color=blue!3!white
			%						,watermark text=\arabic{tcbbreakpart}
			]
			{
				#1
			}
		\end{tcolorbox}
	\end{flushright}
}
