\setcounter{section}{0}

\begin{enumerate}[label=\bfseries Câu \arabic*:]
	\item \mkstar{1}
	
	
	{
		Chọn đáp án đúng nhất. Trường hợp nào sau đây có công cơ học?
		\begin{mcq}
			\item Khi có lực tác dụng vào vật. 
			\item Khi có lực tác dụng vào vật và vật chuyển động theo phương vuông góc với lực. 
			\item Khi có lực tác dụng vào vật và vật đứng yên. 
			\item Khi có lực tác dụng vào vật và vật chuyển động. 
		\end{mcq}
	}
	
	\hideall
	{	
			\textbf{Đáp án: D.}
	}
		\item \mkstar{1}
	
	
	{
		Công thức tính công cơ học khi lực $F$ làm vật dịch chuyển một quãng đường $s$ theo hướng của lực là
		\begin{mcq}(4)
			\item $A=\dfrac{F}{s}$. 
			\item $A=Fs$.
			\item $A=\dfrac{s}{F}$. 
			\item $A=F-s$. 
		\end{mcq}
	}
	
	\hideall
	{	
		\textbf{Đáp án: B.}
		
		Công thức tính công cơ học khi lực $F$ làm vật dịch chuyển một quãng đường $s$ theo hướng của lực là $A=Fs$.
	}
		\item \mkstar{1}
	
	
	{
		Trong các phát biểu sau, phát biểu nào đúng với định luật về công?
		\begin{mcq}
			\item Các máy cơ đơn giản đều cho ta lợi về công. 
			\item Không một máy cơ đơn giản nào cho ta lợi về công. 
			\item Không một máy cơ đơn giản nào cho ta lợi về lực. 
			\item Không một máy cơ đơn giản nào cho ta lợi về đường đi. 
		\end{mcq}
	}
	
	\hideall
	{	
		\textbf{Đáp án: B.}
		
		Không một máy cơ đơn giản nào cho ta lợi về công. Được lợi bao nhiêu lần về lực thì thiệt bấy nhiêu lần về đường đi và ngược lại.
	}
		\item \mkstar{1}
	
	
	{
		Trong trường hợp nào dưới đây \textbf{không} có công cơ học?
		\begin{mcq}
			\item Một người đang kéo một vật chuyển động. 
			\item Hòn bi đang chuyển động thẳng đều trên mặt sàn nằm ngang coi như tuyệt đối nhẵn. 
			\item Một lực sĩ đang nâng quả tạ từ thấp lên cao. 
			\item Máy xúc đất đang làm việc. 
		\end{mcq}
	}
	
	\hideall
	{	
		\textbf{Đáp án: B.}
		
		Hòn bi đang chuyển động thẳng đều trên mặt sàn nằm ngang coi như tuyệt đối nhẵn không có công cơ học.
	}
		\item \mkstar{1}
	
	
	{
		Trong các phát biểu sau, phát biểu nào \textbf{sai}?
		\begin{mcq}
			\item Ròng rọc cố định chỉ có tác dụng đổi hướng của lực và cho ta lợi về công. 
			\item Ròng rọc động cho ta lợi hai lần về lực, thiệt hai lần về đường đi.
			\item Mặt phẳng nghiêng cho ta lợi về lực, thiệt về đường đi. 
			\item Đòn bẩy cho ta lợi về lực, thiệt về đường đi.
		\end{mcq}
	}
	
	\hideall
	{	
		\textbf{Đáp án: A.}
		
		Ròng rọc cố định chỉ có tác dụng đổi hướng của lực và không cho ta lợi về công. 
	}
		\item \mkstar{1}
	
	
	{
		Công suất là
		\begin{mcq}(2)
			\item công thực hiện được trong một giây. 
			\item công thực hiện được trong một ngày. 
			\item công thực hiện được trong một giờ.
			\item công thực hiện được trong một đơn vị thời gian. 
		\end{mcq}
	}
	
	\hideall
	{	
		\textbf{Đáp án: D.}
	}
		\item \mkstar{1}
	
	
	{
		Biểu thức tính công suất là
		\begin{mcq}(4)
			\item $\calP = \dfrac{A}{t}$.
			\item $\calP = At$.
			\item $\calP= \dfrac{t}{A}$.
			\item $\calP = \dfrac{A}{t^2}$.
		\end{mcq}
	}
	
	\hideall
	{	
		\textbf{Đáp án: A.}
	}
		\item \mkstar{1}
	
	
	{
		Đơn vị của công suất là
		\begin{mcq}(4)
			\item W.
			\item kW.
			\item J/s.
			\item Tất cả đều đúng.
		\end{mcq}
	}
	
	\hideall
	{	
		\textbf{Đáp án: D.}	
	}
		\item \mkstar{1}
	
	
	{
		Vật có cơ năng khi
		
		\begin{mcq}(2)
			\item vật có khả năng sinh công.	
			\item vật có khối lượng lớn.	
			\item vật có tính ì lớn.	
			\item vật đứng yên.	
		\end{mcq}
	}
	
	\hideall
	{	
		\textbf{Đáp án: A.}
		
		Vật có cơ năng khi vật có khả năng sinh công.	
	}
		\item \mkstar{1}
	
	
	{
	Thế năng hấp dẫn phụ thuộc vào những yếu tố nào?
	
	\begin{mcq}(2)
		\item Khối lượng.
		\item Trọng lượng riêng.	
		\item Khối lượng và độ cao.	
		\item Khối lượng và vận tốc.	
	\end{mcq}
	}
	
	\hideall
	{	
		\textbf{Đáp án: C.}
	}
		\item \mkstar{1}
	
	
	{
		Phát biểu nào sau đây đầy đủ nhất khi nói về sự chuyển hóa cơ năng?
		
		\begin{mcq}
			\item Động năng có thể chuyển hóa thành thế năng.	
			\item Thế năng có thể chuyển hóa thành động năng.	
			\item Động năng và thế năng có thể chuyển hóa lẫn nhau, cơ năng không được bảo toàn.	
			\item Động năng và thế năng có thể chuyển hóa lẫn nhau, cơ năng được bảo toàn.
		\end{mcq}
	}
	
	\hideall
	{	
		\textbf{Đáp án: D.}
	}
		\item \mkstar{1}
	
	
	{
		Một người dùng một cần cẩu để nâng một thùng hàng có khối lượng 2500 kg lên độ cao 12 m từ mặt đất. Tính công thực hiện được trong trường hợp này.
		
		\begin{mcq}(4)
			\item $\SI{300}{kJ}$.
			\item $\SI{250}{kJ}$.	
			\item $\SI{2.08}{kJ}$.	
			\item $\SI{300}{J}$.	
		\end{mcq}
	}
	
	\hideall
	{	
		\textbf{Đáp án: A.}
		
		Công thực hiện được là
		$$A=10mh=\SI{300}{kJ}$$
	}
		\item \mkstar{1}
	
	
	{
		Một đầu máy xe lửa kéo các toa xe bằng lực $F=\SI{7500}{N}$. Công của lực kéo là bao nhiêu khi các toa xe chuyển động được quãng đường $s=\SI{8}{km}$?
		
		\begin{mcq}(4)
			\item $\SI{60000}{kJ}$.	
			\item $\SI{60}{kJ}$.	
			\item $\SI{60000}{J}$.
			\item Kết quả khác.	
		\end{mcq}
	}
	
	\hideall
	{	
		\textbf{Đáp án: A.}
		
		Công thực hiện được là
		$$A=mgh=\SI{60000}{kJ}$$
	}
		\item \mkstar{1}
	
	
	{
		Con ngựa kéo xe chuyển động đều với vận tốc $\SI{9}{km/h}$. Lực kéo là $200\ \text N$. Công suất của ngựa nhận giá trị nào sau đây?
		
		\begin{mcq}(4)
			\item $\SI{1500}{W}$
			\item $\SI{500}{W}$
			\item $\SI{1000}{W}$	
			\item $\SI{250}{W}$
		\end{mcq}
	}
	
	\hideall
	{	
		\textbf{Đáp án: B.}
		
		Công suất là
		$$\calP = \dfrac{A}{t} = Fv=\SI{500}{W}$$
	}
		\item \mkstar{1}
	
	
	{
		Một máy cơ trong 1 giờ sản ra một công là $\SI{330}{kJ}$. Công suất của máy cơ này là
		
		\begin{mcq}(4)
			\item $\SI{92.5}{W}$.	
			\item $\SI{91.7}{W}$.	
			\item $\SI{90.2}{W}$.	
			\item $\SI{97.5}{W}$.	
		\end{mcq}
	}
	
	\hideall
	{	
		\textbf{Đáp án: B.}
		
		Công suất là
		$$\calP = \dfrac{A}{t} = \SI{91.7}{W}$$
	}
		\item \mkstar{1}
	
	
	{
		Người ta dùng một mặt phẳng nghiêng để kéo một vật. Nếu không có ma sát thì công cần thiết là 125 J. Thực tế có ma sát nên công cần thiết là 175 J. Hiệu suất của mặt phẳng nghiêng trên là bao nhiêu?
		
		\begin{mcq}(4)
			\item $\SI{81.33}{\%}$.	
			\item $\SI{83.33}{\%}$.	
			\item $\SI{71.43}{\%}$.	
			\item $\SI{77.33}{\%}$.	
		\end{mcq}
	}
	
	\hideall
	{	
		\textbf{Đáp án: C.}
		
		Hiệu suất:
		$$H=\dfrac{A_\text{ích}}{A_\text{toàn phần}} = \dfrac{A_F}{A_F + A_\text{ms}} =\SI{71.43}{\%} $$
	}
	\item \mkstar{1}
	
	
	{
		Một người đi xe đạp đều từ chân dốc đến đỉnh dốc cao 5 m. Dốc dài 40 m, biết lực ma sát cản trở chuyển động của xe có độ lớn là 20 N. Cả người và xe có khối lượng là $\SI{37.5}{kg}$. Công tổng cộng do người đó sinh ra là bao nhiêu?
		
		\begin{mcq}(4)
			\item 3800 J.	
			\item 4200 J.	
			\item 4000 J.	
			\item 2675 J.	
		\end{mcq}
	}
	
	\hideall
	{	
		\textbf{Đáp án: D.}
		
		Công do người đó sinh ra:
		$$A=A_P + A_\text{ms} = 2675\ \text J$$	
	}
	\item \mkstar{1}
	
	
	{
		Một cái máy bơm dùng để bơm nước vào ao. Một giờ nó bơm được $\SI{1000}{m^3}$ nước lên cao 2 m. Biết trọng lượng riêng của nước là $\SI{10000}{N/m^3}$. Công suất của máy bơm là
		
		\begin{mcq}(4)
			\item $\SI{5}{kW}$.	
			\item $\SI{5200.2}{W}$.	
			\item $\SI{5555.6}{W}$.	
			\item $\SI{5650}{W}$.	
		\end{mcq}
	}
	
	\hideall
	{	
		\textbf{Đáp án: C.}
		
		Công suất máy bơm:
		$$\calP = \dfrac{A}{t} = \dfrac{Ph}{t} = \dfrac{dVh}{t} = \SI{5555.6}{W}$$
	}
	\item \mkstar{2}


{
	Trong quá trình chuyển động, nếu vật chỉ chịu tác dụng của trọng lực thì cơ năng của vật đó được tính bởi hệ thức
	\begin{mcq}(2)
		\item $W=2mgz+mv^2$.
		\item $W=mgz+mv^2$.
		\item $W=2mgz+\dfrac{1}{2}mv^2$.
		\item $W=mgz+\dfrac{1}{2}mv^2$.
	\end{mcq}
}

\hideall
{	
	\textbf{Đáp án: D.}
	
	Cơ năng của vật bằng tổng động năng và thế năng:
	$$W=mgz+\dfrac{1}{2}mv^2.$$
}
\item \mkstar{2}


{
	Một vật có khối lượng $\SI{2}{kg}$ rơi tự do không vận tốc đầu từ độ cao $\SI{5}{m}$ xuống mặt đất. Lấy $g=\SI{10}{m/s^2}$. Cơ năng của vật bằng
	\begin{mcq}(4)
		\item $\SI{10}{J}$.
		\item $\SI{100}{J}$.
		\item $\SI{50}{J}$.
		\item $\SI{5}{J}$.
	\end{mcq}
}

\hideall
{	
	\textbf{Đáp án: B.}
	
	Cơ năng của vật bằng tổng động năng và thế năng (vì $v=0$ nên động năng bằng 0):
	$$W=W_\text t = mgh = \SI{100}{J}.$$
}
\item \mkstar{2}


{
	Từ điểm M có độ cao $\SI{0.8}{m}$ so với mặt đất, người ta ném lên một vật có khối lượng $\SI{0.5}{kg}$ với vận tốc $\SI{2}{m/s}$. Lấy $g=\SI{10}{m/s^2}$. Cơ năng của vật bằng
	\begin{mcq}(4)
		\item $\SI{5}{J}$. 
		\item $\SI{8}{J}$.
		\item $\SI{4}{J}$.
		\item $\SI{1}{J}$.
	\end{mcq}
}

\hideall
{	
	\textbf{Đáp án: A.}
	
	Cơ năng của vật bằng tổng động năng và thế năng:
	$$W=mgh + \dfrac{1}{2}mv^2 = \SI{5}{J}.$$
}
	\item \mkstar{2}


{
	Một vật có khối lượng $m=\SI{1}{kg}$ được thả rơi tự do từ độ cao $\SI{20}{m}$ so với mặt đất. Chọn gốc thế năng tại mặt đất. Bỏ qua mọi ma sát. Ngay khi vật chạm đất thì
	\begin{mcq}(2)
		\item động năng cực đại, thế năng cực tiểu.
		\item động năng bằng thế năng.
		\item động năng cực tiểu, thế năng cực đại.
		\item động năng bằng một nửa thế năng.
	\end{mcq}
}

\hideall
{	
	\textbf{Đáp án: A.}
	
	Khi vật chạm đất thì động năng cực đại vì $v$ cực đại, thế năng cực tiểu vì $h$ cực tiểu.
}
	\item \mkstar{2}


{
	Một vật có khối lượng $\SI{100}{g}$ được ném thẳng đứng lên cao với vận tốc $\SI{10}{m/s}$ từ độ cao $\SI{5}{m}$ so với mặt đất. Chọn gốc thế năng tại mặt đất. Lấy $g=\SI{10}{m/s^2}$. Cơ năng của vật khi chuyển động là
	\begin{mcq}(4)
		\item $\SI{15}{J}$.
		\item $\SI{11.25}{J}$.
		\item $\SI{10.5}{J}$.
		\item $\SI{10}{J}$.
	\end{mcq}
}

\hideall
{	
	\textbf{Đáp án: D.}
	
	Cơ năng của vật bằng tổng động năng và thế năng:
	$$W=mgh + \dfrac{1}{2}mv^2 = \SI{10}{J}.$$
}	
		\item \mkstar{2}
	
	
	{
		Một vật khối lượng $m=\SI{2}{kg}$ được ném theo phương thẳng đứng hướng xuống từ độ cao $\SI{15}{m}$ so với mặt đất với tốc độ $\SI{10}{m/s}$. Bỏ qua mọi lực cản. Chọn gốc thế năng tại mặt đất. Lấy $g=\SI{10}{m/s^2}$. Tốc độ của vật khi vật vừa chạm đất là
		\begin{mcq}(4)
			\item $\SI{10}{m/s}$.
			\item $\SI{15}{m/s}$.
			\item $\SI{20}{m/s}$.
			\item $\SI{400}{m/s}$.
		\end{mcq}
	}
	
	\hideall
	{	
		\textbf{Đáp án: C.}
		
		Bảo toàn cơ năng lúc vừa ném và lúc vừa chạm đất:
		$$W_1 = W_2 \Rightarrow W_\text{đ 1} + W_\text{t 1} = W_\text{đ 2} + W_\text{t 2} \Rightarrow \dfrac{1}{2}mv_1^2 + mgz_1 = \dfrac{1}{2}mv_2^2 + mgz_2.$$
		$$\Rightarrow \dfrac{1}{2}v_1^2 + gz_1 = \dfrac{1}{2}v_2^2 + gz_2 \Rightarrow v_2 = \SI{20}{m/s}.$$
	}
	\item \mkstar{2}
	
	
	{
		Một vật trượt trên mặt phẳng nghiêng có ma sát, sau khi lên tới điểm cao nhất nó trượt xuống vị trí ban đầu. Trong quá trình chuyển động trên
		\begin{mcq}
			\item công của lực ma sát tác dụng vào vật bằng 0.
			\item tổng công của trọng lực và lực ma sát tác dụng vào vật bằng 0.
			\item công của trọng lực tác dụng vào vật bằng 0.
			\item hiệu giữa công của trọng lực và lực ma sát tác dụng vào vật bằng 0.
		\end{mcq}
	}
	
	\hideall
	{	
		\textbf{Đáp án: C.}
		
		Một vật trượt trên mặt phẳng nghiêng có ma sát, sau khi lên tới điểm cao nhất nó trượt xuống vị trí ban đầu. Trong quá trình chuyển động trên công của trọng lực tác dụng vào vật bằng 0 vì công của trọng lực khi vật đi lên với khi vật đi xuống trái dấu, cùng độ lớn.
	}
\end{enumerate}
\hideall
{
	\begin{center}
		\textbf{BẢNG ĐÁP ÁN}
	\end{center}
	\begin{center}
		\begin{tabular}{|m{2.8em}|m{2.8em}|m{2.8em}|m{2.8em}|m{2.8em}|m{2.8em}|m{2.8em}|m{2.8em}|m{2.8em}|m{2.8em}|}
			\hline
			1.D  & 2.B  & 3.B  & 4.B  & 5.A  & 6.D  & 7.A  & 8.D  & 9.A  & 10.C  \\
			\hline
			11.D  & 12. A  & 13.A  & 14.B  & 15.B  & 16.C  & 17.D  & 18.C  & 19.D  & 20.B  \\
			\hline
			21.A  & 22.A  & 23.D  & 24.C & 25.C  &   &   &   &   &  \\
			\hline
		\end{tabular}
	\end{center}
}