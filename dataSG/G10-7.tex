\newcommand{\chapter}[2][]{
	\newcommand{\chapname}{#2}
	\begin{flushleft}
		\begin{minipage}[t]{\linewidth}
			\includegraphics[height=1cm]{hdht-logo.png}
			\hspace{0pt}	
			\sffamily\bfseries\large Bài  9.
			\begin{flushleft}
				\huge\bfseries #1
			\end{flushleft}
		\end{minipage}
	\end{flushleft}
	\vspace{1cm}
	\normalfont\normalsize
}
\chapter[Chuyển động thẳng biến đổi đều]{Chuyển động thẳng biến đổi đều}
\section{Lý thuyết}

	\begin{itemize}
	\item Chọn hệ quy chiếu gồm:
	\begin{itemize}
		\item Chiều dương (thường là chiều chuyển động của một vật);
		\item Gốc tọa độ (thường là vị trí xuất phát của một vật);
		\item Mốc thời gian (thường là thời điểm bắt đầu chuyển động của một vật).
	\end{itemize}
	\item Phương trình chuyển động của vật có dạng tổng quát như sau:
	\begin{equation*}
		x=x_0+v(t-t_0)+\dfrac{1}{2}a(t-t_0)^2\textrm{ với }(t\geq t_0),
	\end{equation*}
	trong đó:
	\begin{itemize}
		\item[+] $x_0$: tọa độ ban đầu của vật tại thời điểm $t_0$;
		\item[+] $x$: tọa độ của vật tại thời điểm $t$;
		\item[+] $v$: vận tốc của vật ($v>0$ nếu vật chuyển động cùng chiều dương, $v<0$ nếu vật chuyển động ngược chiều dương);
		\item[+] $v$: gia tốc của vật ($a\cdot v\geq 0$ nếu vật chuyển động nhanh dần đều, $a\cdot v\leq 0$ nếu vật chuyển động chậm dần đều).
	\end{itemize}
	\item Hai vật gặp nhau khi chúng có cùng tọa độ:
	\begin{equation*}
		x_1=x_2.
	\end{equation*}
	\item Khoảng cách giữa hai vật tại thời điểm $t$ bất kì là:
	\begin{equation*}
		d=\left|x_1-x_2\right|.
	\end{equation*}
\end{itemize}	
\section{Mục tiêu bài học - Ví dụ minh họa}
\begin{dang}{Thực hiện xác định quãng đường, vận tốc, gia tốc, thời gian của chuyển động thẳng biến đổi đều thông qua phương trình.}
	\viduii{2}{Cho phương trình chuyển động của một chất điểm dọc theo trục $Ox$ có dạng $x = 10 + 4t - \text{0,5}t^2$ . Vận tốc của chuyển động sau $\SI{2}{s}$ là bao nhiêu?
	}
	{	\begin{center}
			\textbf{Hướng dẫn giải}
		\end{center}
		
		Phương trình chuyển động:
		$$x = 10 + 4t - \text{0,5}t^2.$$
		
		Suy ra: $v_0 = \SI{4}{m/s}; a =-\SI{1}{m/s}^2$.
		
		Phương trình vận tốc:
		$$v =v_0 +at =4-t.$$
		
		Với $t =\SI{2}{s}$ suy ra $v= \SI{2}{m/s}$.
		
	}
	\viduii{2}{Phương trình cơ bản của 1 vật chuyển động: $x = 6t^2 - 18t + 12 \ (\text{cm})$. Hãy xác định:
		
		a/ Vận tốc của vật, gia tốc của chuyển động và cho biết tính chất của chuyển động.
		
		b/ Vận tốc của vật ở thời điểm $t = \SI{2}{s}$.
	}
	{	\begin{center}
			\textbf{Hướng dẫn giải}
		\end{center}
		
		a/ Phương trình chuyển động
		
		$$x = 6t^2 - 18t + 12 \ \text{cm}$$
		Suy ra: $v_0 = -\SI{18}{cm/s}; a =\SI{12}{cm/s}^2$.
		
		Vật chuyển động chậm dần đều do $av<0$.
		
		b/ Phương trình vận tốc:
		$$v =v_0 +at =-18+12t.$$
		
		Ở thời điểm $t=\SI{2}{s} \Rightarrow v =\SI{6}{cm/s}$.
		
	}
	
	\viduii{3}{Một vật chuyển động thẳng có phương trình: $x = 4t^2 + 20t\ (\text{m})$. Tính quãng đường vật đi được từ thời điểm $t_1 = \SI{2}{s}$ đến thời điểm $t_2 = \SI{5}{s}$.
	}
	{	\begin{center}
			\textbf{Hướng dẫn giải}
		\end{center}
		
		Vị trí của vật đi được $\SI{2}{s}$ là:
		
		$$s_1 = 4t_1^2 + 20t_1 =\SI{56}{m}.$$
		
		Vị trí của vật đi được $\SI{5}{s}$ là:
		
		$$s_2 = 4t_2^2 + 20t_2 =\SI{200}{m}.$$
		
		Quãng đường vật đi từ $\SI{2}{s}$  đến $\SI{5}{s}$ là:
		$$\Delta x = x_2 - x_1 = \SI{144}{m}.$$
		
	}
	\viduii{3}{Vật chuyển động thẳng có phương trình: $x = 2t^2 - 4t + 10\ (\text{m})$. Vật sẽ dừng lại tại vị trí
		\begin{mcq}(4)
			\item $\SI{6}{m}.$
			\item $\SI{4}{m}.$
			\item $\SI{10}{m}.$
			\item $\SI{8}{m}.$
		\end{mcq}
	}
	{	\begin{center}
			\textbf{Hướng dẫn giải}
		\end{center}
		
		Vật sẽ dừng lại khi $v = 0$.
		
		Từ phương trình chuyển động ta suy ra phương trình vận tốc: $$v = - 4 + 4t.$$
		
		Vật sẽ dừng lại thì $v=0$ suy ra $t =\SI{1}{s}.$
		
		Thay $t =\SI{1}{s}$ vào phương trình chuyển động ta được $x = \SI{8}{m}.$
		
		\textbf{Đáp án: D}.
	}
\end{dang}
\begin{dang}{Xây dựng phương trình chuyển động thẳng biến đổi đều}
	\viduii{2}{Một vật chuyển động thẳng chậm dần đều với tốc độ ban đầu $\SI{3}{\meter/\second}$ và gia tốc có độ lớn $\SI{2}{\meter/\second^2}$. Biết thời điểm ban đầu vật ở gốc tọa độ và chuyển động ngược chiều dương của trục tọa độ. Viết phương trình chuyển động của vật.
	}
	{	\begin{center}
			\textbf{Hướng dẫn giải}
		\end{center}
		
		Chọn gốc thời gian là khi vật bắt đầu chuyển động.
		
		Vì vật chuyển động chậm dần đều ngược chiều dương nên:
		\begin{equation*}
			\left\{
			\begin{array}{rcl}
				a\cdot v &<&0\\
				v &<& 0
			\end{array}
			\right.
			\quad
			\Rightarrow 
			\left\{
			\begin{array}{rcl}
				a &>& 0\\
				v &<& 0.
			\end{array}
			\right.
		\end{equation*}
		
		Kết hợp với các dữ kiện của đề bài, ta suy ra:
		\begin{equation*}
			\left\{
			\begin{array}{rcr}
				a&=&\SI{2}{\meter/\second^2}\\
				v&=&\SI{-3}{\meter/\second} .
			\end{array}
			\right.
		\end{equation*}
		
		Phương trình chuyển động của vật có dạng:
		$x=-3t+t^2$ (m, s).
	}
	\viduii{3}{Một đoạn dốc thẳng dài $\SI{62,5}{m}$, Nam đi xe đạp và khởi hành từ chân dốc đi lên với $v_0 =\SI{18}{km/h}$ chuyển động chậm dần đều với gia tốc có độ lớn $\SI{0,2}{m/s}^2$.
		
		a/ Viết phương trình chuyển động của Nam.
		
		b/ Nam đi hết đoạn dốc trong bao lâu?
	}
	{	\begin{center}
			\textbf{Hướng dẫn giải}
		\end{center}
		
		Đổi $\SI{18}{km/h} = \SI{5}{m/s}$.
		
		Chọn gốc toạ độ tại chân dốc, chiều dương từ chân đến đỉnh dốc, gốc thời gian là khi Nam bắt đầu lên dốc.
		
		a/ Khi nam lên dốc, Nam đi theo chiều dương nên $v>0$.
		
		Chuyển động chậm dần đều: 
		$$av<0 \Rightarrow a<0.$$
		
		Phương trình chuyển động:
		$$x =x_0 +v_0t+\dfrac{1}{2}at^2 = 5t - \text{0,1}t^2.$$
		
		b/ Thời gian đi hết đoạn dốc
		$$\text{62,5} =5t - \text{0,1}t^2 \Rightarrow t = \SI{25}{s}.$$
	}
\end{dang}

\begin{dang}{Thực hiện xác định vị trí, thời điểm hai vật chuyển động thẳng biến đổi đều gặp nhau}
	\viduii{3}{Một xe ô tô bắt đầu chuyển động thẳng nhanh dần đều với gia tốc $\SI{0,5}{m/s}^2$ đúng lúc một xe máy chuyển động thẳng đều với vận tốc $\SI{36}{km/h}$ vượt qua nó.Xác định thời điểm và vị trí hai xe gặp nhau và vận tốc xe ô tô khi đó?
		
		b/ Xác định thời điểm để hai xe cách nhau một quãng đường là $\SI{100}{m}$.
	}
	{	\begin{center}
			\textbf{Hướng dẫn giải}
		\end{center}
		
		a/ Chọn chiều dương là chiều chuyển động của ô tô, gốc tọa độ tại vị trí xuất phát, gốc thời gian là lúc xe ô tô khởi hành.
		
		+ Đối với xe ô tô:
		$$x_{01} = 0; v_{01} = 0; a_{01} = \SI{0,5}{m/s}^2.$$	
		
		Phương trình chuyển động:
		$$x_1 = \text{0,25}t^2.$$
		
		+ Đối với xe máy:
		$$x_{02} = 0; v_{02} =\SI{36}{km/h}=\SI{10}{m/s}; a_{01} = \SI{0}{m/s}^2.$$	
		
		Phương trình chuyển động:
		$$x_2 =10t.$$
		
		+ Hai xe gặp nhau:
		$$x_1=x_2 \Rightarrow \text{0,25}t^2=10t.$$
		Suy ra $t =\SI{40}{s}.$
		
		+ Vị trí 2 xe gặp nhau $x = \SI{400}{m}$.
		
		+ Vậy hai xe gặp nhau sau $\SI{40}{s}$ và cách gốc là $\SI{400}{m}$
		
		+ Vận tốc ô tô:
		$$v_1 = v_{01}+ a_1t = \SI{20}{m/s}.$$  
		
	}
	\viduii{3}{
		Trong một chuyến từ thiện của trung tâm A thì mọi người dừng lại bên đường uống nước. Sau đó ô tô bắt đầu chuyển động nhanh dần đều với gia tốc $\SI{0,5}{m/s}^2$ thì có một xe khách vượt qua xe với vận tốc $\SI{18}{km/h}$ và gia tốc $\SI{0,3}{m/s}^2$. Hỏi khi ô tô đuổi kịp xe khách thì vận tốc của ô tô và sau quãng đường bao nhiêu?
	}
	{	\begin{center}
			\textbf{Hướng dẫn giải}
		\end{center}
		
		Chọn chiều dương là chiều chuyển động của ô tô, gốc tọa độ tại vị trí uống nước, gốc thời gian là lúc xe ô tô khởi hành.
		
		+ Đối với xe ô tô:
		$$x_{01} = 0; v_{01} = 0; a_{01} = \SI{0,5}{m/s}^2.$$	
		
		Phương trình chuyển động:
		$$x_1 = \text{0,25}t^2.$$
		
		+ Đối với xe khách:
		$$x_{02} = 0; v_{02} =\SI{18}{km/h}=\SI{5}{m/s}; a_{01} = \SI{0,3}{m/s}^2.$$	
		
		Phương trình chuyển động:
		$$x_2 =5t+\text{0,15}t^2.$$
		
		+ Hai xe gặp nhau:
		$$x_1=x_2 \Rightarrow \text{0,25}t^2=5t+\text{0,15}t^2.$$
		Suy ra $t =\SI{50}{s}.$
		
		+ Vận tốc ô tô:
		$$v_1 = v_{01}+ a_1t = \SI{25}{m/s}.$$  
		
		+ Quãng đường đi:
		$$S =\text{0,25}t^2 = \SI{625}{m}.$$	
	}
\end{dang}
\begin{dang}{Thực hiện xác định vận tốc, khoảng cách giữa hai vật chuyển động thẳng biến đổi đều}
	\viduii{3}{Phương trình cơ bản của một vật chuyển động: $x=6t^2-18t+12$ (cm, s). Hãy xác định:	
		\begin{enumerate}[label=\alph*)]
			\item Vận tốc ban đầu của vật, gia tốc của chuyển động và cho biết tính chất của chuyển động.
			\item Vận tốc của vật ở thời điểm $t=\SI{2}{\second}$.
		\end{enumerate}
	}
	{	\begin{center}
			\textbf{Hướng dẫn giải}
		\end{center}
		
		\begin{enumerate}[label=\alph*)]
			\item
			Phương trình chuyển động của vật: $x=6t^2-18t+12$ (cm, s).
			
			Từ phương trình chuyển động ta suy ra được vận tốc ban đầu và gia tốc của vật là:
			\begin{equation*}
				\left\{\begin{array}{ll}{v_0=\SI{-18}{cm/\second}}&\\{a=\SI{12}{cm/\second^2}.}&\end{array}\right.
			\end{equation*}
			
			Vì $a\cdot v_0 < 0$ nên vật chuyển động chậm dần đều.
			\item 
			Vận tốc của vật ở thời điểm $t=\SI{2}{\second}$ là:
			\begin{equation*}
				v=v_0+at=\SI{6}{cm/\second}.
			\end{equation*}
		\end{enumerate}
	}
	\viduii{3}{Một xe ô tô bắt đầu chuyển động thẳng nhanh dần đều với gia tốc $\SI{0,5}{m/s}^2$ đúng lúc một xe máy chuyển động thẳng đều với vận tốc $\SI{36}{km/h}$ vượt qua nó. Xác định thời điểm để hai xe cách nhau một quãng đường là $\SI{100}{m}$.
	}
	{	\begin{center}
			\textbf{Hướng dẫn giải}
		\end{center}
		
		Chọn chiều dương là chiều chuyển động của ô tô, gốc tọa độ tại vị trí xuất phát, gốc thời gian là lúc xe ô tô khởi hành.
		
		+ Đối với xe ô tô:
		$$x_{01} = 0; v_{01} = 0; a_{01} = \SI{0,5}{m/s}^2.$$	
		
		Phương trình chuyển động:
		$$x_1 = \text{0,25}t^2.$$
		
		+ Đối với xe máy:
		$$x_{02} = 0; v_{02} =\SI{36}{km/h}=\SI{10}{m/s}; a_{01} = \SI{0}{m/s}^2.$$	
		
		Phương trình chuyển động:
		$$x_2 =10t.$$
		
		Để 2 xe cách nhau $\SI{40}{m}$ thì 
		$$|x_1-x_2| = 100.$$
		$$\Rightarrow \left\{\begin{array}{ll}{x_1-x_2 =100}&\\{x_2-x_1 =100.}&\end{array}\right.$$
		
		$$\Rightarrow \left\{\begin{array}{ll}{\text{0,25}t^2 -10t=100 \Rightarrow t=\SI{48,28}{s}}&\\{10t-\text{0,25}t^2=100\Rightarrow t =\SI{20}{s}.}&\end{array}\right.$$
		
		
	}
\end{dang}
\begin{dang}{Thực hiện xác định vị trí, thời điểm gặp nhau giữa vật chuyển động thẳng đều và vật chuyển động thẳng biến đổi đều}
	\viduii{3}{Một ôtô đang chuyển động thẳng đều với vận tốc $\SI{45}{km/h}$ bỗng tăng ga chuyển động nhanh dần đều.
		
		a.  Tính gia tốc của xe biết rằng sau $\SI{30}{s}$ ô tô đạt vận tốc $\SI{72}{km/h}$.
		
		b.  Trong quá trình tăng tốc nói trên, vào thời điểm nào kể từ lúc tăng tốc, vận tốc của xe là $\SI{64,8}{km/h}$?
		
	}
	{	\begin{center}
			\textbf{Hướng dẫn giải}
		\end{center}
		
		Đổi $\SI{45}{km/h} = \SI{12,5}{m/s};\ \SI{72}{km/h} = \SI{20}{m/s}$.
		
		Chọn chiều dương là chiều chuyển động.
		
		a. Gia tốc của xe là
		$$a = \dfrac{v-v_0}{t} = \SI{0,25}{m/s}^2.$$
		
		b. Đổi $\SI{64,8}{km/h} = \SI{18}{m/s}$.
		
		Thời gian của xe từ lúc tăng tốc tới vận tốc của xe là $\SI{64,8}{km/h}$ là
		
		$$t= \dfrac{v'-v_0}{a} = \SI{22}{s}.$$
		
	}
	\viduii{3}{Cùng một lúc, từ hai địa điểm A và B cách nhau $\SI{50}{m}$ có hai vật chuyển động ngược chiều để gặp nhau. Vật thứ nhất xuất phát từ A chuyển động đều với vận tốc $\SI{5}{m/s}$, vật thứ hai xuất phát từ B chuyển động nhanh dần đều không vận tốc đầu với gia tốc $\SI{2}{m/s}^2$. Chọn trục $Ox$ trùng đường thẳng AB, gốc tọa độ tại A, chiều dương từ A đến B, gốc thời gian là lúc xuất phát
		
		a.  Viết phương trình chuyển động của mỗi vật.
		
		b.  Xác định thời điểm và vị trí hai xặp nhau.
		
		c.   Xác định thời điểm mà tại đó hai vật có vận tốc bằng nhau.
		
		
	}
	{	\begin{center}
			\textbf{Hướng dẫn giải}
		\end{center}
		
		Chọn chiều dương là chiều chuyển động của xe A. 
		
		a. Phương trình chuyển động của 2 xe lần lượt là:
		
		$$x_\text{A} = x_0+vt = 5t.$$
		
		$$x_\text{B} = x_0 + vt +\dfrac{1}{2}at^2 = 50 - t^2.$$
		
		b. Hai xe gặp nhau thì:
		
		$$x_\text{A} = x_\text{B}.$$
		$$\Leftrightarrow 5t = 50 - t^2.$$
		
		Suy ra $t =\SI{5}{s}.$
		
		Vậy để 2 xe gặp nhau sau $\SI{5}{s}$ và cách A $\SI{25}{m}.$
		
		c. Phương trình vận tốc của xe B là:
		
		$$v_\text{B} = v_0 +at = 2t.$$
		
		Hai xe có cùng vận tốc
		$$2t = 5 \Rightarrow t =\SI{2,5}{s}.$$
		
		Vậy hai xe có vận tốc bằng nhau khi $x_\text{A} = \SI{12,5}{m}$ chuyển động. 
		
	}
	\viduii{3}{Hai vật cùng xuất phát một lúc tại A, chuyển động cùng chiều. Vật thứ nhất chuyển động đều với vận tốc $v_1 = \SI{20}{m/s}$, vật thứ hai chuyển động thẳng nhanh dần đều với vận tốc ban đầu bằng không và gia tốc $\SI{0,4}{m/s}^2$. Chọn chiều dương là chiều chuyển động, gốc tọa độ O tại A, gốc thời gian là lúc xuất phát.
		
		a.  Xác định thời điểm và vị trí hai xe gặp nhau.
		
		b.  Viết phương trình vận tốc của vật thứ hai. Xác định khoảng cách giữa hai vật tại thời điểm chúng có vận tốc bằng nhau.
		
	}
	{	\begin{center}
			\textbf{Hướng dẫn giải}
		\end{center}
		
		a. Phương trình chuyển động:
		$$x_1 = 20t.$$
		$$x_2 = \text{0,2}t^2.$$
		
		Khi hai vật gặp nhau thì:
		
		$$x_1 =x_2 \Leftrightarrow 20t = \text{0,2}t^2.$$
		
		Suy ra: $t =0$ và $t =\SI{100}{s}.$
		
		Vị trí gặp: $x_1 =x_2 =\SI{2000}{m}.$
		
		b. Phương trình vận tốc của vật thứ hai
		$$v_2 =\text{0,4}t\ \text{m/s}.$$
		
		Thời điểm lúc hai vật có vận tốc bằng nhau: 
		
		$$v_2 =\text{0,4}t = 20 \Rightarrow t = \SI{50}{s}.$$
		
		Tọa độ các vật lúc đó: 
		
		$$x_1 = 20t = \SI{1000}{m}; x_2 = \text{0,2}t^2 = \SI{500}{m}.$$
		
		Khoảng cách giữa hai vật: 
		
		$$\Delta x = x_1 - x_2 =\SI{500}{m}.$$
	}
	
\end{dang}
