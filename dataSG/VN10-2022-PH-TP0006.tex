\setcounter{section}{0}

\begin{enumerate}[label=\bfseries Câu \arabic*:]
	\item \mkstar{1}
	
	
	{
		Lực đàn hồi xuất hiện tỉ lệ với độ biến dạng khi 
		\begin{mcq}(2)
			\item một vật bị biến dạng dẻo.
			\item một vật biến dạng đàn hồi.
			\item một vật bị biến dạng.	
			\item ta ấn ngón tay vào một viên đất nặn.
		\end{mcq}
	}
	
	\hideall
	{	
		\textbf{Đáp án: B.}
		
		Lực đàn hồi xuất hiện tỉ lệ với độ biến dạng khi 
	}
		\item \mkstar{1}
	
	
	{
		Kết luận nào sau đây \textbf{không} đúng đối với lực đàn hồi?
		\begin{mcq}(2)
			\item Xuất hiện khi vật bị biến dạng.
			\item Luôn là lực kéo.
			\item Tỉ lệ với độ biến dạng.
			\item ngược hướng với lực làm nó bị biến dạng.
		\end{mcq}
	}
	
	\hideall
	{	
		\textbf{Đáp án: B.}
		
		Lực đàn hồi có khi là lực kéo, có khi là lực nén.
	}
		\item \mkstar{1}
	
	
	{
		Điều nào sau đây là \textbf{sai} khi nói về phương và độ lớn của lực đàn hồi?
		\begin{mcq}
			\item Với cùng độ biến dạng như nhau, độ lớn của lực đàn hồi phụ thuộc vào kích thước và bản chất của vật đàn hồi. 
			\item Với các mặt tiếp xúc bị biến dạng, lực đàn hồi vuông góc với các mặt tiếp xúc. 
			\item Với các vật như lò xo, dây cao su, thanh dài, lực đàn hồi hướng dọc theo trục của vật. 
			\item Lực đàn hồi có độ lớn tỉ lệ nghịch với độ biến dạng của vật biến dạng. 
		\end{mcq}
	}
	
	\hideall
	{	
		\textbf{Đáp án: D.}
		
		 Lực đàn hồi tỉ lệ thuận với độ biến dạng của vật biến dạng.
	}
		\item \mkstar{1}
	
	
	{
		Khẳng định nào sau đây là đúng khi ta nói về lực đàn hồi của lò xo và lực căng của dây?
		\begin{mcq}
			\item Đó là những lực chống lại sự biến dạng đàn hồi của lò xo và sự căng của dây.
			\item Đó là những lực gây ra sự biến dạng đàn hồi của lò xo và sự căng của dây.
			\item Chúng đều là những lực kéo.
			\item Chúng đều là những lực đẩy.
		\end{mcq}
	}
	
	\hideall
	{	
		\textbf{Đáp án: A.}
		
		Lực đàn hồi của lò xo và lực căng của dây: đó là những lực chống lại sự biến dạng đàn hồi của lò xo và sự căng của dây.
	}
		\item \mkstar{1}
	
	
	{
		Một vật tác dụng một lực vào một lò xo có đầu cố định và làm lò xo biến dạng. Điều nào dưới đây là \textbf{không} đúng?
		\begin{mcq}
			\item Độ đàn hồi của lò xo có độ lớn bằng lực tác dụng và chống lại sự biến dạng của lò xo.
			\item Lực đàn hồi cùng phương và ngược chiều với lực tác dụng. 
			\item Lực đàn hồi lớn hơn lực tác dụng và chống lại lực tác dụng.
			\item Khi vật ngừng tác dụng lên lò xo thì lực đàn hồi của lò xo cũng mất đi.
		\end{mcq}
	}
	
	\hideall
	{	
		\textbf{Đáp án: C.}
	}
		\item \mkstar{1}
	
	
	{
		Chọn phát biểu \textbf{sai} về lực đàn hồi của lò xo.
		\begin{mcq}
			\item Lực đàn hồi của lò xo có xu hướng chống lại nguyên nhân gây ra biến dạng.
			\item Lực đàn hồi của lò xo dài có phương là trục lò xo, chiều ngược với chiều biến dạng của lò xo.
			\item Lực đàn hồi của lò xo có độ lớn tuân theo định luật Húc.
			\item Lực đàn hồi của lò xo chỉ xuất hiện ở đầu lò xo đặt ngoại lực gây biến dạng.
		\end{mcq}
	}
	
	\hideall
	{	
		\textbf{Đáp án: A.}
		
		Lực đàn hồi xuất hiện ở hai đầu của lò xo và tác dụng vào các vật tiếp xúc (hay gắn) với lò xo làm nó biến dạng.
	}
	\item \mkstar{1}
	
	
	{
		Lực đàn hồi của lò xo có tác dụng làm cho lò xo
		\begin{mcq}(2)
			\item chuyển động.
			\item thu gia tốc. 
			\item có xu hướng lấy lại hình dạng và kích thước ban đầu.        
			\item vừa biến dạng vừa thu gia tốc.
		\end{mcq}
	}
	
	\hideall
	{	
		\textbf{Đáp án: C.}
	}
	\item \mkstar{2}
	
	
	{
		Một lò xo có chiều dài tự nhiên là $\SI{20}{cm}$. Khi lò xo có chiều dài $\SI{24}{cm}$ thì lực đàn hồi của nó bằng $\SI{5}{N}$. Hỏi khi lực đàn hồi của lò xo bằng $\SI{10}{N}$ thì chiều dài của nó bằng bao nhiêu?
		\begin{mcq}(4)
			\item $\SI{22}{cm}$.
			\item $\SI{28}{cm}$.
			\item $\SI{40}{cm}$.
			\item $\SI{48}{cm}$.
		\end{mcq}
	}
	
	\hideall
	{	
		\textbf{Đáp án: B.}
		
		Ta có:
		
		$$F = k \Delta l.$$
		
		Vậy: 
		$$F_1 = k \Delta l_1; F_2 = k \Delta l_2.$$
		
		Lập tỉ số:
		
		$$\dfrac{F_1}{F_2} = \dfrac{\Delta l_1}{\Delta l_2} \Rightarrow \Delta l_2 = \SI{0,08}{m} = \SI{8}{cm}.$$
		
		Chiều dài của lò xo là
		
		$$l' = l_0 + \Delta l_2 = \SI{28}{cm}.$$
	}
	\item \mkstar{2}
	
	
	{
		Một lò xo có chiều dài tự nhiên bằng $\SI{22}{cm}$. Lò xo được treo thẳng đứng, một đầu giữ cố định, còn đầu kia gắn một vật nặng. Khi ấy lò xo dài $\SI{27}{cm}$, cho biết độ cứng lò xo là $\SI{100}{N/m}$. Độ lớn lực đàn hồi bằng 
		\begin{mcq}(4)
			\item $\SI{500}{N}.$
			\item $\SI{5}{N}.$
			\item $\SI{20}{N}.$
			\item $\SI{50}{N}.$
		\end{mcq}
	}
	
	\hideall
	{	
		\textbf{Đáp án: B.}
		
		Độ biến dạng của lò xo:
		
		$$\Delta l = l-l_0 = \SI{5}{cm} = \SI{0,05}{m}.$$
		
		Độ lớn của lực đàn hồi:
		
		$$F_\text{đh} = k\Delta l = \SI{5}{N}.$$
		
	}
	\item \mkstar{2}
	
	
	{
		Phải treo một vật có khối lượng bằng bao nhiêu vào lò xo có độ cứng $\SI{100}{N/m}$ để lò xo dãn ra được $\SI{10}{cm}$? Lấy $g = \SI{10}{m/s}^2$. 
		\begin{mcq}(4)
			\item $\SI{1}{kg}$.
			\item $\SI{10}{kg}$.
			\item $\SI{100}{kg}$.
			\item $\SI{1000}{kg}$.
		\end{mcq}
	}
	
	\hideall
	{	
		\textbf{Đáp án: A.}
		
		Ta có: 
		
		$$F_\text{đh} = k\Delta l = \SI{10}{N}.$$
		
		Mà $F_\text{đh} = P.$
		
		Suy ra:
		
		$$m = \dfrac{P}{g} = \SI{1}{kg}.$$
		
		
	}
		\item \mkstar{2}
	
	
	{
		Phải treo một vật có trọng lượng bằng bao nhiêu vào một lò xo có độ cứng $k = \SI{100}{N/m}$ để nó dãn ra được $\SI{10}{cm}$? Lấy $g = \SI{10}{m/s}^2$.
		\begin{mcq}(4)
			\item $\SI{1000}{N}.$
			\item $\SI{100}{N}.$
			\item $\SI{10}{N}.$
			\item $\SI{1}{N}.$
		\end{mcq}
	}
	
	\hideall
	{	
		\textbf{Đáp án: C.}
		
		Ta có: 
		
		$$F_\text{đh} = k\Delta l = \SI{10}{N}.$$
		
		Mà $F_\text{đh} = P.$
		
		Nên $P = \SI{10}{N}.$
	}
	\item \mkstar{2}
	
	
	{
		Dùng một lò xo để treo một vật có khối lượng $\SI{300}{g}$ thì thấy lò xo giãn một đoạn $\SI{2}{cm}$. Nếu treo thêm một vật có khối lượng $\SI{150}{g}$ thì độ giãn của lò xo là
		\begin{mcq}(4)
			\item $\SI{1}{cm}$.
			\item $\SI{2}{cm}$.
			\item $\SI{3}{cm}$.
			\item $\SI{4}{cm}$.
		\end{mcq}
	}
	
	\hideall
	{	
		\textbf{Đáp án: C.}
		
		Khi treo vật khối lượng $\SI{300}{g}$:
		
		$$m_1g = k\Delta l_1 \Rightarrow k = \dfrac{m_1 g}{\Delta l_1}.$$
		
		Khi treo thêm một vật ta có:
		
		$$(m_1+m_2)g = k\Delta l_2 \Rightarrow \Delta l_2 = \dfrac{(m_1+m_2)g}{k} = \SI{0,03}{m} = \SI{3}{cm}.$$
	}
	\item \mkstar{2}
	
	
	{Một lò xo có chiều dài tự nhiên bằng $\SI{20}{cm}$. Khi bị kéo lò xo dài $\SI{24}{cm}$ và lực đàn hồi của nó bằng $\SI{5}{N}$. Hỏi khi lực đàn hồi của lò xo bằng $\SI{15}{N}$ thì chiều dài của nó bằng bao nhiêu?
		
		\begin{mcq}(4)
			\item $\SI{28}{cm}$.
			\item $\SI{32}{cm}$.
			\item $\SI{45}{cm}$.
			\item $\SI{20}{cm}$.
		\end{mcq}
	}
	
	\hideall
	{	
		\textbf{Đáp án: B.}
		
			Ta có:
		
		$$F = k \Delta l.$$
		
		Vậy: 
		$$F_1 = k \Delta l_1; F_2 = k \Delta l_2.$$
		
		Lập tỉ số:
		
		$$\dfrac{F_1}{F_2} = \dfrac{\Delta l_1}{\Delta l_2} \Rightarrow \Delta l_2 = \SI{0,12}{m} = \SI{12}{cm}.$$
		
		Chiều dài của lò xo là
		
		$$l' = l_0 + \Delta l_2 = \SI{32}{cm}.$$
	}
	\item \mkstar{1}
	
	
	{
		Khi đoàn tàu đang chuyển động trên đường nằm ngang thì áp lực có độ lớn bằng lực nào?
		\begin{mcq}(2)
			\item Lực kéo do đầu tàu tác dụng lên toa tàu.
			\item Trọng lực của tàu.
			\item Lực ma sát giữa tàu và đường ray.
			\item Cả ba lực trên.
		\end{mcq}
	}
	
	\hideall
	{	
		\textbf{Đáp án: D.}
	}
	\item \mkstar{1}
	
	
	{Đơn vị của áp lực là
		
		\begin{mcq}(4)
			\item N/m$^2.$            
			\item Pa.
			\item N.
			\item N/cm$^2$
		\end{mcq}
	}
	
	\hideall
	{	
		\textbf{Đáp án: C.}
	}
		\item \mkstar{1}
	
	
	{
		Chỉ ra kết luận \textbf{sai} trong các kết luận sau.
		\begin{mcq}
			\item Áp lực là lực ép có phương vuông góc với mặt bị ép.
			\item Đơn vị của áp suất là N/m$^2.$ 
			\item Áp suất là độ lớn của áp lực trên một diện tích bị ép.
			\item Đơn vị của áp lực là đơn vị của lực.
		\end{mcq}
	}
	
	\hideall
	{	
		\textbf{Đáp án: C.}
	}
	\item \mkstar{1}
	
	
	{
		Khi nhúng một khối lập phương vào nước, mặt nào của khối lập phương chịu áp lực lớn nhất của nước?
		\begin{mcq}(2)
			\item Áp lực như nhau ở cả 6 mặt.
			\item Mặt trên.
			\item Mặt dưới.
			\item Các mặt bên.
		\end{mcq}
	}
	
	\hideall
	{	
		\textbf{Đáp án: C.}
	}
	\item \mkstar{1}
	
	
	{
		Muốn tăng áp suất thì
		\begin{mcq}
			\item Giảm diện tích mặt bị ép và giảm áp lực theo cùng tỉ lệ.
			\item Giảm diện tích mặt bị ép và tăng áp lực.
			\item Tăng diện tích mặt bị ép và tăng áp lực theo cùng tỉ lệ.
			\item Tăng diện tích mặt bị ép và giảm áp lực.
		\end{mcq}
	}
	
	\hideall
	{	
		\textbf{Đáp án: B.}
	}
	\item \mkstar{1}
	
	
	{
		Muốn giảm áp suất thì
		\begin{mcq}
			\item Giảm diện tích mặt bị ép và giảm áp lực theo cùng tỉ lệ.
			\item Tăng diện tích mặt bị ép và tăng áp lực theo cùng tỉ lệ.
			\item Tăng diện tích mặt bị ép và giữ nguyên áp lực.
			\item Giảm diện tích mặt bị ép và giữ nguyên áp lực.
		\end{mcq}
	}
	
	\hideall
	{	
		\textbf{Đáp án: C.}
	}
	\item \mkstar{1}
	
	
	{
		Khi nằm trên đệm mút ta thấy êm hơn khi nằm trên phản gỗ. Tại sao vậy?
		\begin{mcq}
			\item Vì đệm mút mềm hơn phản gỗ nên áp suất tác dụng lên người giảm.
			\item Vì đệm mút dầy hơn phản gỗ nên áp suất tác dụng lên người giảm.
			\item Vì đệm mút dễ biến dạng để tăng diện tích tiếp xúc vì vậy giảm áp suất tác dụng lên thân người.
			\item Vì lực tác dụng của phản gỗ vào thân người lớn hơn.
		\end{mcq}
	}
	
	\hideall
	{	
		\textbf{Đáp án: C.}
	}
		\item \mkstar{2}
	
	
	{Một người tác dụng lên mặt sàn một áp suất $\text{1,7}\xsi{\cdot 10^4}{N/m^2}$. Diện tích của bàn chân tiếp xúc với mặt sàn là $\SI{0,03}{m^2}$. Trọng lượng của người đó là
		
		\begin{mcq}(4)
			\item $\SI{51}{N}$.
			\item $\SI{510}{N}$.
			\item $\SI{5100}{N}$.
			\item $\SI{51000}{N}$.
		\end{mcq}
	}
	
	\hideall
	{	
		\textbf{Đáp án: B.}
		
		Trọng lượng của người bằng áp lực của người đó tác dụng lên mặt sàn:
		
		$$ P = F = pS = \SI{510}{N}.$$
		
		
	}
	\item \mkstar{2}
	
	
	{
		Biết thầy Giang có khối lượng $\SI{60}{kg}$, diện tích một bàn chân là $\SI{30}{cm}^2$. Tính áp suất thầy Giang tác dụng lên sàn khi đứng cả hai chân. Lấy $g= \SI{10}{m/s^2}.$
		\begin{mcq}(4)
			\item $\SI{1}{Pa}$.
			\item $\SI{2}{Pa}$.
			\item $\SI{10}{Pa}$.
			\item $\SI{100000}{Pa}$.
		\end{mcq}
	}
	
	\hideall
	{	
		\textbf{Đáp án: D.}
		
		Diện tích của hai bàn chân của thầy:
		
		$$S'=2S = \SI{0,006}{m^2}.$$
		
		Áp suất của thầy tác dụng lên sàn:
		
		$$p = \dfrac{F}{S} = \dfrac{P}{S} = \SI{100000}{Pa}.$$
	}
	\item \mkstar{2}
	
	
	{
		Một máy đánh ruộng (2 bánh) có khối lượng 1 tấn, để máy chạy được trên nền đất ruộng thì áp suất máy tác dụng lên đất là $\SI{10000}{Pa}$. Lấy $g= \SI{10}{m/s^2}.$ Diện tích 1 bánh của máy đánh phải tiếp xúc với ruộng là
		\begin{mcq}(4)
			\item $\SI{1}{m^2}$.
			\item $\SI{0,5}{m^2}$.
			\item $\SI{10000}{m^2}$.
			\item $\SI{10}{m^2}$.
		\end{mcq}
	}
	
	\hideall
	{	
		\textbf{Đáp án: B.}
		
		Trọng lực của máy cày:
		
		$$P = mg = \xsi{10^4}{N}.$$
		
		Diện tích tiếp xúc của máy
		
		$$S = \dfrac{P}{p} = \SI{1}{m^2}.$$
		
		Diện tích tiếp xúc 1 bánh là
		
		$$S' = \dfrac{S}{2} = \SI{0,5}{m^2}.$$
	}
	\item \mkstar{2}
	
	
	{
		Một bình hình trụ cao $\SI{1,8}{m}$ đựng đầy rượu. Biết khối lượng riêng của rượu là $\SI{800}{kg/m}^3$. Áp suất của rượu tác dụng lên điểm M cách đáy bình $\SI{20}{cm}$ là
		\begin{mcq}(4)
			\item $\SI{1440}{Pa}$.
			\item $\SI{1280}{Pa}$.
			\item $\SI{12800}{Pa}$.
			\item $\SI{1600}{Pa}$.
		\end{mcq}
	}
	
	\hideall
	{	
		\textbf{Đáp án: C.}
		
		Khoảng cách từ điểm M đến mặt thoáng $h = \SI{1,6}{m}.$
		
		Áp suất của rượu tác dụng lên M
		
		$$ p_\text{M} = \rho g h =\SI{12800}{Pa}.$$
	}
	\item \mkstar{1}
	
	
	{
		Kết luận nào sau đây đúng khi nói về áp suất chất lỏng?
		\begin{mcq}
			\item Áp suất mà chất lỏng tác dụng lên một điểm phụ thuộc khối lượng lớp chất lỏng phía trên. 
			\item Áp suất mà chất lỏng tác dụng lên một điểm phụ thuộc trọng lượng lớp chất lỏng phía trên.
			\item Áp suất mà chất lỏng tác dụng lên một điểm phụ thuộc thể tích lớp chất lỏng phía trên.
			\item Áp suất mà chất lỏng tác dụng lên một điểm phụ thuộc độ cao lớp chất lỏng phía trên.
		\end{mcq}
	}
	
	\hideall
	{	
		\textbf{Đáp án: D.}
	}
	
\end{enumerate}
	\hideall
	{
		\begin{center}
			\textbf{BẢNG ĐÁP ÁN}
		\end{center}
		\begin{center}
			\begin{tabular}{|m{2.8em}|m{2.8em}|m{2.8em}|m{2.8em}|m{2.8em}|m{2.8em}|m{2.8em}|m{2.8em}|m{2.8em}|m{2.8em}|}
				\hline
				1.B  & 2.B  & 3.D  & 4.A  & 5.C  & 6.A  & 7.C  & 8.B  & 9.B  & 10.A  \\
				\hline
				11.C  & 12. C  & 13.B  & 14.D  & 15.C  & 16.C  & 17.C  & 18.B  & 19.C  & 20.C  \\
				\hline
				21.B  & 22.D  & 23.B  & 24.C  & 25.D  &   &   &   &   &  \\
				\hline
			\end{tabular}
		\end{center}
	}