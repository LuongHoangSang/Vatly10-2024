\setcounter{section}{0}
\section{Trắc nghiệm}
\begin{enumerate}[label=\bfseries Câu \arabic*:]
	\item \mkstar{1}
	
	
	{Chuyển động cơ học là
		\begin{mcq}
			\item sự thay đổi khoảng cách của vật so với vật khác. 
			\item sự thay đổi phương chiều của vật. 
			\item sự thay đổi vị trí của vật so với vật khác. 
			\item sự thay đổi hình dạng của vật so với vật khác. 
		\end{mcq}
	}
	
	\hideall
	{		\textbf{Đáp án: C.}
		
		Chuyển động cơ học là sự thay đổi vị trí của vật so với vật khác. 
		
	}
	\item \mkstar{1}
	
	
	{Chuyển động và đứng yên có tính tương đối vì
		\begin{mcq}
			\item một vật đứng yên so với vật này sẽ đứng yên so với vật khác. 
			\item một vật đứng yên so với vật này nhưng lại chuyển động so với vật khác.
			\item một vật chuyển động hay đứng yên phụ thuộc vào quỹ đạo chuyển động. 
			\item một vật chuyển động so với vật này sẽ chuyển động so với vật khác. 
		\end{mcq}
	}
	
	\hideall
	{		\textbf{Đáp án: B.}
		
		Chuyển động và đứng yên có tính tương đối vì một vật đứng yên so với vật này nhưng lại chuyển động so với vật khác. 
		
	}
	\item \mkstar{2}
	
	
	{Một chiếc xe buýt chuyển động trên đường, nếu ta nói chiếc xe buýt đang đứng yên thì đã chọn vật làm mốc là gì?
		\begin{mcq}(2)
			\item Người đứng bên đường. 
			\item Tài xế. 
			\item Cột mốc cắm trên đường. 
			\item Mặt đường. 
		\end{mcq}
	}
	
	\hideall
	{\textbf{Đáp án: B.}
		
		Đối với tài xế trong xe, chiếc xe buýt đứng yên.
		
	}
	\item \mkstar{1}
	
	
	{Dạng chuyển động của viên đạn được bắn ra từ khẩu súng AK là
		\begin{mcq}
			\item chuyển động thẳng. 
			\item chuyển động cong. 
			\item chuyển động tròn. 
			\item vừa chuyển động cong vừa chuyển động tròn. 
		\end{mcq}
	}
	
	\hideall
	{\textbf{Đáp án: B.}
		
		Dạng chuyển động của viên đạn được bắn ra từ khẩu súng AK là chuyển động cong.
		
		
	}
	\item \mkstar{1}
	
	
	{
		Dạng chuyển động của quả bom được ném ra từ máy bay ném bom là
		\begin{mcq}
			\item chuyển động thẳng. 
			\item chuyển động cong. 
			\item chuyển động tròn. 
			\item vừa chuyển động cong vừa chuyển động tròn. 
		\end{mcq}
	}
	
	\hideall
	{\textbf{Đáp án: B.}
		
		Dạng chuyển động của quả bom được ném ra từ máy bay ném bom là chuyển động cong. 
		
	}
	
	\item \mkstar{1}
	
	
	{Dạng chuyển động của quả dừa rơi từ trên cây xuống là
		\begin{mcq}
			\item chuyển động thẳng. 
			\item chuyển động cong. 
			\item chuyển động tròn. 
			\item vừa chuyển động cong vừa chuyển động tròn. 
		\end{mcq}
		
	}
	
	\hideall
	{\textbf{Đáp án: A.}
		
		Dạng chuyển động của quả dừa rơi từ trên cây xuống là chuyển động thẳng.		
		
	}
	\item \mkstar{2}
	
	
	{Hai chiếc tàu hỏa chạy cùng chiều, cùng vận tốc trên hai đường ray song song nhau. Người ngồi trên chiếc tàu thứ nhất sẽ
		\begin{mcq}(2)
			\item chuyển động so với tàu thứ hai. 
			\item đứng yên so với tàu thứ hai. 
			\item chuyển động so với tàu thứ nhất.
			\item đứng yên so với đường ray. 
		\end{mcq}
		
	}
	
	\hideall
	{\textbf{Đáp án: B.}
		
		Người ngồi trên chiếc tàu thứ nhất sẽ đứng yên so với tàu thứ hai.
		
	}
	
	\item \mkstar{2}
	
	
	{Hai ô tô chuyển động cùng chiều, cùng vận tốc đi ngang qua một ngôi nhà. Phát biểu nào dưới đây là đúng?
		\begin{mcq}
			\item Các ô tô chuyển động đối với nhau.
			\item Các ô tô đứng yên đối với ngôi nhà.
			\item Các ô tô đứng yên đối với nhau.
			\item Ngôi nhà đứng yên đối với các ô tô.
		\end{mcq}
	}
	
	\hideall
	{	\textbf{Đáp án: C.}
		
		Các ô tô đứng yên đối với nhau.
		
	}
	\item \mkstar{2}
	
	
	{Trên toa xe lửa đang chạy thẳng đều, một chiếc va li đặt trên giá để hàng. Ta nói chiếc va li
		\begin{mcq}(2)
			\item chuyển động so với thành tàu.
			\item chuyển động so với đầu máy.
			\item chuyển động so với người lái tàu.
			\item chuyển động so với đường ray.
		\end{mcq}
	}
	
	\hideall
	{\textbf{Đáp án: D.}	
		
		Ta nói chiếc va li chuyển động so với đường ray.
		
	}
	\item \mkstar{2}
	
	
	{Chuyển động của đầu van xe đạp so với trục xe khi chuyển động thẳng trên đường là
		
		\begin{mcq}
			\item chuyển động tròn.	
			\item chuyển động thẳng.	
			\item chuyển động cong.	
			\item sự kết hợp giữa chuyển động thẳng với chuyển động tròn.	
		\end{mcq}
		
	}
	
	\hideall
	{\textbf{Đáp án: C.}
		
		Chuyển động của đầu van xe đạp so với trục xe khi chuyển động thẳng trên đường là chuyển động cong.
		
	}
	\item \mkstar{2}
	
	
	{Chuyển động của đầu van xe đạp so với mặt đường khi xe chuyển động thẳng trên đường là
		
		\begin{mcq}
			\item chuyển động tròn.	
			\item chuyển động thẳng.	
			\item chuyển động cong.	
			\item sự kết hợp giữa chuyển động thẳng với chuyển động tròn.	
		\end{mcq}
		
	}
	
	\hideall
	{\textbf{Đáp án: D.}
		
		Chuyển động của đầu van xe đạp so với mặt đường khi xe chuyển động thẳng trên đường là sự kết hợp giữa chuyển động thẳng với chuyển động tròn.	
	}
	\item \mkstar{2}
	
	
	{Nhà Lan cách trường 2 km. Lan đạp xe từ nhà tới trường mất 10 phút. Vận tốc đạp xe của Lan là
		
		\begin{mcq}(4)
			\item $\SI{0.2}{km/h}$.	
			\item $\SI{200}{m/s}$.	
			\item $\SI{3.33}{m/s}$.	
			\item $\SI{2}{km/h}$.	
		\end{mcq}
		
	}
	
	\hideall
	{\textbf{Đáp án: C.}
		
		Vận tốc đạp xe của Lan là
		$$v=\dfrac{s}{t} = \SI{3.33}{m/s}$$
	}
	\item \mkstar{2}
	
	
	{Mai đi bộ tới trường với vận tốc $\SI{4}{km/h}$ mất thời gian là 15 phút. Khoảng cách từ nhà Mai tới trường là
		
		\begin{mcq}(4)
			\item $\SI{1000}{m}$.
			\item $\SI{6}{km}$.	
			\item $\SI{3.75}{km}$.	
			\item $\SI{3600}{m}$.	
		\end{mcq}
		
	}
	
	\hideall
	{\textbf{Đáp án: A.}
		
		Khoảng cách từ nhà Mai tới trường là
		$$s=vt=\SI{1000}{m}$$
	}
	\item \mkstar{2}
	
	
	{Đường từ nhà Nam tới công viên dài $\SI{7.2}{km}$. Nếu đi với vận tốc không đổi $\SI{1}{m/s}$ thì thời gian để Nam đi từ nhà tới công viên là
		
		\begin{mcq}(4)
			\item $\SI{0.5}{h}$.	
			\item $\SI{1}{h}$.	
			\item $\SI{1.5}{h}$.
			\item $\SI{2}{h}$.	
		\end{mcq}
		
	}
	
	\hideall
	{\textbf{Đáp án: D.}
		
		Thời gian để Nam đi từ nhà tới công viên là
		$$t=\dfrac{s}{v} = \SI{2}{h}$$
	}
	\item \mkstar{2}
	
	
	{Đường đi từ nhà đến trường dài $\SI{4.8}{km}$. Nếu đi xe đạp với vận tốc trung bình $\SI{4}{m/s}$ thì thời gian để Nam đến trường mất bao lâu?
		
		\begin{mcq}(4)
			\item $\SI{1.2}{h}$
			\item $\SI{120}{s}$
			\item $\xsi{1/3}{h}$	
			\item $\SI{0.3}{h}$
		\end{mcq}
		
	}
	
	\hideall
	{\textbf{Đáp án: C.}
		
		Thời gian để Nam đến trường là
		$$t=\dfrac{s}{v} = \xsi{1/3}{h}$$
	}
	\item \mkstar{2}
	
	
	{Vận tốc của ô tô là $\SI{36}{km/h}$, của xe máy là $\SI{34000}{m/h}$, của tàu hỏa là $\SI{14}{m/s}$. Sắp xếp độ lớn vận tốc của các phương tiện trên từ bé đến lớn là
		
		\begin{mcq}(2)
			\item tàu hỏa - ô tô - xe máy.	
			\item ô tô - tàu hỏa - xe máy.	
			\item ô tô - xe máy - tàu hỏa.	
			\item xe máy - ô tô - tàu hỏa.	
		\end{mcq}
		
	}
	
	\hideall
	{\textbf{Đáp án: D.}
		
		Sắp xếp độ lớn vận tốc của các phương tiện trên từ bé đến lớn là xe máy - ô tô - tàu hỏa.
	}
	\item \mkstar{3}
	
	
	{Hùng đứng gần một vách núi và hét lên một tiếng, sau 2 giây kể từ khi hét thì Hùng nghe thấy tiếng vọng lại. Hỏi khoảng cách từ Hùng đến vách núi là bao xa? Biết vận tốc của âm thanh trong không khí là $\SI{330}{m/s}$.
		
		\begin{mcq}(4)
			\item 660 m.	
			\item 330 m.	
			\item 115 m.	
			\item 55 m.	
		\end{mcq}
		
	}
	
	\hideall
	{\textbf{Đáp án: B.}
		
		Quãng đường âm thanh đi được là $2s$, áp dụng công thức:
		$$2s=vt \Rightarrow s = 330\ \text m$$
	}
	\item \mkstar{3}
	
	
	{Lúc 5h sáng, Tân chạy thể dục từ nhà ra công viên. Biết từ nhà Tân đến công viên cách nhau $\SI{2.5}{km}$. Tân chạy với vận tốc $\SI{5}{km/h}$. Hỏi Tân về đến nhà lúc mấy giờ? Cho rằng khi đến công viên thì Tân lập tức chạy về nhà với cùng vận tốc như cũ.
		
		\begin{mcq}(4)
			\item 5 giờ 30 phút.	
			\item 6 giờ.	
			\item 1 giờ.	
			\item 0,5 giờ.	
		\end{mcq}
		
	}
	
	\hideall
	{\textbf{Đáp án: B.}
		
		Thời gian để Tân chạy từ nhà đến công viên rồi chạy về:
		$$t= \dfrac{2s}{v} = 1\ \text{giờ}$$
		
		Mà Tân xuất phát lúc 5 giờ, sau 1 giờ thì chạy thì Tân về nhà lúc 6 giờ.	
	}
	\item \mkstar{3}
	
	
	{Lúc 5h sáng, Tiến chạy thể dục từ nhà ra công viên. Biết từ nhà Tiến đến công viên cách nhau $\SI{2.5}{km}$. Tiến chạy với vận tốc $\SI{5}{km/h}$. Hỏi Tiến mất bao lâu để về đến nhà? Cho rằng khi đến công viên thì Tiến lập tức chạy về nhà với cùng vận tốc như cũ.
		
		\begin{mcq}(4)
			\item 5 giờ 30 phút.	
			\item 6 giờ.	
			\item 1 giờ.	
			\item 0,5 giờ.	
		\end{mcq}
		
	}
	
	\hideall
	{\textbf{Đáp án: C.}
		
		Thời gian để Tiến chạy từ nhà đến công viên rồi chạy về:
		$$t= \dfrac{2s}{v} = 1\ \text{giờ}$$
	}
	\item \mkstar{4}
	
	
	{Hai ô tô chuyển động thẳng đều, khởi hành đồng thời từ 2 địa điểm cách nhau 20 km. Nếu đi ngược chiều nhau thì sau 15 phút chúng gặp nhau. Nếu đi cùng chiều thì sau 30 phút chúng đuổi kịp nhau. Vận tốc của hai xe đó là
		
		\begin{mcq}(2)
			\item $\SI{20}{km/h}$ và $\SI{30}{km/h}$.	
			\item $\SI{30}{km/h}$ và $\SI{40}{km/h}$.	
			\item $\SI{40}{km/h}$ và $\SI{20}{km/h}$.
			\item $\SI{20}{km/h}$ và $\SI{60}{km/h}$.
		\end{mcq}
		
	}
	
	\hideall
	{\textbf{Đáp án: D.}
		
		Khi đi ngược chiều, hai xe gặp nhau khi
		$$s = s_1 + s_2 \Rightarrow \SI{20}{km} = v_1 t_1 + v_2 t_2$$
		
		Khi đi cùng chiều, hai xe gặp nhau khi
		$$s=|s_1 - s_2| \Rightarrow \SI{20}{km} = |v_1 t_1' - v_2t_2'|$$
		
		Giải hệ trên, ta được $\SI{20}{km/h}$ và $\SI{60}{km/h}$.
	}
	
\end{enumerate}


\hideall
{
	\begin{center}
		\textbf{BẢNG ĐÁP ÁN}
	\end{center}
	\begin{center}
		\begin{tabular}{|m{2.8em}|m{2.8em}|m{2.8em}|m{2.8em}|m{2.8em}|m{2.8em}|m{2.8em}|m{2.8em}|m{2.8em}|m{2.8em}|}
			\hline
			1.C  & 2.B  & 3.B  & 4.B  & 5.B  & 6.A  & 7.B  & 8.C  & 9.D  & 10.C  \\
			\hline
			11.D  & 12.C  & 13.A  & 14.D  & 15.C  & 16.D  & 17.B  & 18.B  & 19.C  & 20.D  \\
			\hline
			
		\end{tabular}
	\end{center}
}
\section{Tự luận}
\begin{enumerate}[label=\bfseries Câu \arabic*:]
	\item \mkstar{1}
	
	{
		Làm thế nào để biết một ô tô trên đường, một chiếc thuyền trôi trên sông, một đám mây trên trời, ... đang chuyển động hay đứng yên?
	}
	
	\hideall{
		Để biết một ô tô trên đường, một chiếc thuyền trôi trên sông, một đám mây trên trời, ... đang chuyển động hay đứng yên, ta cần so sánh với những vật mốc cố định như mặt đường, cột mốc, bờ sông, ...
	}
	
	\item \mkstar{1}
	
	{Hãy tìm ví dụ về chuyển động cơ học, trong đó chỉ rõ vật được chọn làm mốc.}
	
	\hideall{
		HS tự tìm lấy 1 ví dụ.
	}
	\item \mkstar{2}
	
	
	{Mặt Trời mọc đằng đông, lặn đằng tây. Như vậy có phải Mặt Trời chuyển động còn Trái Đất đứng yên không?
	}
	
	\hideall
	{Có thể coi như là Mặt Trời chuyển động quanh Trái Đất nếu chọn Trái Đất làm vật mốc. Tuy nhiên thực tế thì Trái Đất mới là quay xung quanh Mặt Trời.
	}
	\item \mkstar{2}
	
	
	{Vận tốc của một ô tô là $\SI{36}{km/h}$, của người đi xe đạp là $\SI{10.8}{km/h}$, của một tàu hỏa là $\SI{10}{m/s}$. Các con số trên cho biết điều gì?
	}
	
	\hideall
	{Các con số trên chỉ vận tốc.
		\begin{itemize}
			\item Vận tốc của một ô tô là $\SI{36}{km/h}$: Ô tô đi trong thời gian 1 giờ được quãng đường dài 36 km;
			\item Vận tốc của người đi xe đạp là $\SI{10.8}{km/h}$: Người đi xe đạp trong 1 giờ được quãng đường dài $\SI{10.8}{km}$;
			\item Vận tốc của một tàu hỏa là $\SI{10}{m/s}$: Tàu hỏa đi trong 1 giây được quãng đường dài $\SI{10}{m}$.
		\end{itemize}
	}
	\item \mkstar{2}
	
	
	{Một đoàn tàu trong hời gian $\SI{1.5}{h}$ đi được quãng đường dài $\SI{81}{km}$. Tính vận tốc của tàu theo đơn vị $\SI{}{km/h}$ và $\SI{}{m/s}$.
	}
	
	\hideall
	{
		Tính theo đơn vị $\SI{}{km/h}$: $v=\SI{54}{km/h}$.
		
		Tính theo đơn vị $\SI{}{m/s}$: $v=\SI{15}{m/s}$.
	}
\end{enumerate}