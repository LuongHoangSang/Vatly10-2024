\let\lesson\undefined
\newcommand{\lesson}{\phantomlesson{Bài 7: Gia tốc. Chuyển động thẳng biến đổi đều}}
\chapter[Phương trình toạ độ của vật chuyển động thẳng biến đổi đều]{Phương trình toạ độ của vật chuyển động thẳng biến đổi đều}
\setcounter{section}{0}
\section{Lý thuyết}
\subsection{Phương trình toạ độ của chất điểm chuyển động thẳng biến đổi đều}
Phương trình chuyển động của vật là phương trình mô tả sự thay đổi tọa độ của vật theo thời gian. \\
Để lập phương trình toạ độ của vật chuyển động thẳng biến đổi đều, ta thực hiện các bước như sau:
\begin{itemize}
	\item Chọn hệ quy chiếu gồm:
	\begin{itemize}
		\item Chiều dương (thường là chiều chuyển động của một vật);
		\item Gốc tọa độ (thường là vị trí xuất phát của một vật);
		\item Mốc thời gian (thường là thời điểm bắt đầu chuyển động của một vật).
	\end{itemize}
	\item Xét một điểm chuyển động thẳng biến đổi đều trên đường thẳng O$x$. Ở thời điểm ban đầu ($t_0$), chất điểm ở vị trí A có tọa độ $x_{0}$  với vận tốc ban đầu $v_0$ và gia tốc $a$. Mốc thời gian được chọn lúc bắt đầu chuyển động. Ở thời điểm $t$, chất điểm ở vị trí B có tọa độ $x$ như hình vẽ.  	
	\begin{center}
		\begin{tikzpicture}
			\coordinate (laxis) at (-0.5,0);
			\coordinate (O) at (0,0);
			\coordinate (A) at (2,0);
			\coordinate (va) at ($(A)+(1,0)$);
			\coordinate (raxis) at (8,0);
			\coordinate (ldaxis) at (-0.5,-1);
			\coordinate (Od) at (0,-1);
			\coordinate (Ad) at (2,0);
			\coordinate (B) at (5,-1);
			\coordinate (vb) at ($(B)+(2,0)$);
			\coordinate (rdaxis) at (8,-1);
			\draw[->] (laxis) -- (raxis);
			\draw[->] (ldaxis) -- (rdaxis);
			
			
			\draw[->,ultra thick,blue] (A) -- (va);
			\draw[->,ultra thick,green!60!black] (B) -- (vb);
			\node[above=2mm] at (A) {A};
			\node[below left=1mm and 0.5mm] at (A) {$x_0$};
			\node[above=2mm] at (B) {B};
			\node[below left=1mm and 0.5mm] at (B) {$x$};
			\node[above=2mm] at (O) {O};
			
			%		\node[above=2mm] at (Od) {O};
			%		\node[below=2mm] at (C) {C};
			\node[right] at (raxis) {$x$};
			\node[above=1mm] at (va) {$\vec{v}_{0}$};
			\node[above=1mm] at (vb) {$\vec{v}$};
			
			\node[left=3cm,anchor=west] at (laxis) {thời điểm $t_0$};
			\node[left=3cm,anchor=west] at (ldaxis) {thời điểm $t> t_0$};
			\foreach \i in {O,Od,B,A}
			{
				\filldraw (\i) circle (2pt);
			}
			
			%		\coordinate (odd) at ($(O)-(0,2)$);
			\coordinate (add) at ($(A)-(0,2)$);
			\coordinate (bdd) at ($(B)-(0,1)$);
			%		\draw[<->,thick] (odd) -- (add);
			\draw[<->] (add) -- (bdd) node[midway,fill=pagecol] {$d$};
			\draw[dashed] (O)--(Od);
			\draw[dashed] (A)--(add);
			\draw[dashed] (B)--(bdd);
			
		\end{tikzpicture}
	\end{center}
	Phương trình chuyển động của vật có dạng tổng quát như sau:
	\begin{equation*}
		x=x_0+d=x_0+v_0\cdot(t-t_0)+\dfrac{1}{2}a\cdot(t-t_0)^2\qquad\textrm{ với }(t\geq t_0),
	\end{equation*}
	Thông thường, để thuận tiện trong tính toán, ta chọn thời điểm $t_0=0$, khi đó phương trình chuyển động của chất điểm trở thành 
	\begin{equation*}
		x=x_0+v_0t+\dfrac{1}{2}at^{2}.
	\end{equation*}
\end{itemize}	
\subsection{Điều kiện để hai vật gặp nhau}
Hai vật gặp nhau khi chúng có cùng tọa độ:
\begin{equation*}
	x_1=x_2.
\end{equation*}
\subsection{Khoảng cách giữa hai vật trong quá trình chuyển động}
Khoảng cách giữa hai vật tại thời điểm $t$ bất kì là:
\begin{equation*}
	\Delta x=\left|x_1-x_2\right|.
\end{equation*}
\section{Mục tiêu bài học - Ví dụ minh họa}
\begin{dang}{Xây dựng phương trình\\ chuyển động thẳng biến đổi đều}
	\viduii{2}{Một vật chuyển động thẳng chậm dần đều với tốc độ ban đầu $\SI{3}{\meter/\second}$ và gia tốc có độ lớn $\SI{2}{\meter/\second^2}$. Biết thời điểm ban đầu vật ở gốc tọa độ và chuyển động ngược chiều dương của trục tọa độ. Viết phương trình chuyển động của vật.
	}
	{	\begin{center}
			\textbf{Hướng dẫn giải}
		\end{center}
		
		Chọn gốc thời gian là khi vật bắt đầu chuyển động.
		
		Vì vật chuyển động chậm dần đều ngược chiều dương nên
		\begin{equation*}
			\left\{
			\begin{array}{rcl}
				a\cdot v &<&0\\
				v &<& 0
			\end{array}
			\right.
			\quad
			\Rightarrow 
			\left\{
			\begin{array}{rcl}
				a &>& 0\\
				v &<& 0.
			\end{array}
			\right.
		\end{equation*}
		
		Kết hợp với các dữ kiện của đề bài, ta suy ra
			\begin{align*}
				\begin{cases}
					a=\SI{2}{\meter/\second^2}\\
					v=\SI{-3}{\meter/\second}\\
					x_0=\SI{0}{\meter} \qquad\text{(vì ban đầu vật ở gốc toạ độ.)}
				\end{cases}
			\end{align*}
		Do đó, phương trình chuyển động của vật có dạng
		$x=-3t+t^2$ (m, s).
	}
	\viduii{3}{Một đoạn dốc thẳng dài $\SI{62,5}{m}$, Nam đi xe đạp và khởi hành từ chân dốc đi lên với $v_0 =\SI{18}{km/h}$ chuyển động chậm dần đều với gia tốc có độ lớn $\SI{0,2}{m/s}^2$.
		\begin{enumerate}[label=\alph*.]
			\item Viết phương trình chuyển động của Nam.
			\item Nam đi hết đoạn dốc trong bao lâu?
		\end{enumerate}
	}
	{	\begin{center}
			\textbf{Hướng dẫn giải}
		\end{center}
		
		Đổi đơn vị $$\SI{18}{km/h} = \dfrac{\SI{18e3}{\meter}}{\SI{3600}{\second}}=\SI{5}{m/s}.$$
		
		Chọn gốc toạ độ tại chân dốc, chiều dương từ chân đến đỉnh dốc, gốc thời gian là khi Nam bắt đầu lên dốc.
		\begin{enumerate}[label=\alph*.]
			\item Khi nam lên dốc, Nam đi theo chiều dương nên $v>0$.
			
			Chuyển động chậm dần đều: 
			$$a\cdot v<0 \Rightarrow a<0.$$
			
			Phương trình chuyển động:
			$$x =x_0 +v_0t+\dfrac{1}{2}at^2 = 5t - \text{0,1}t^2.$$
			\item Thời gian đi hết đoạn dốc
			$$\text{62,5} =5t - \text{0,1}t^2 \Rightarrow t = \SI{25}{s}.$$
		\end{enumerate}
	}
\end{dang}

\begin{dang}{Xác định vị trí, thời điểm hai vật chuyển động thẳng biến đổi đều gặp nhau}
	\viduii{3}{Một xe ô tô bắt đầu chuyển động thẳng nhanh dần đều với gia tốc $\SI{0,5}{\meter/\second^{2}}$ đúng lúc một xe máy chuyển động thẳng đều với tốc độ $\SI{36}{\kilo\meter/\hour}$ vượt qua nó.
		Xác định thời điểm và vị trí hai xe gặp nhau lần nữa và vận tốc xe ô tô khi đó?
		Xác định thời điểm để hai xe cách nhau một quãng đường là $\SI{100}{\meter}$.
		
	}
	{	\begin{center}
			\textbf{Hướng dẫn giải}
		\end{center}
		Chọn chiều dương là chiều chuyển động của ô tô, gốc tọa độ tại vị trí xuất phát, gốc thời gian là lúc xe ô tô khởi hành.
		
		Xe ô tô có các thông số chuyển động 
		$$x_{10} = \SI{0}{\meter};\qquad  v_{10} = \SI{0}{\meter/\second};\qquad a_1 = \SI{0.5}{\meter
			/\second^{2}}$$	
		nên có phương trình chuyển động 
		$$x_1 = x_{10}+v_{10}t+\dfrac{1}{2}a_{1}t^{2}=\dfrac{1}{2}a_1t^2.$$
		
		Xe máy có các thông số chuyển động 
		$$x_{20} = 0;\qquad v_{20} =\SI{36}{\kilo\meter/\hour}=\SI{10}{\meter/\second};\qquad a_{2} = \SI{0}{\meter/\second^{2}}$$	
		nên có phương trình chuyển động 
		$$x_2 =x_{20}+v_{20}t+\dfrac{1}{2}a_2t^2=v_{20}t.$$
		
		Tọa độ hai xe bằng nhau khi hai xe gặp nhau
			$$x_1=x_2$$
			$$\Rightarrow\dfrac{1}{2}a_1t^2=v_{20}t$$
		\begin{align*}
			\begin{cases}
				t=\SI{0}{\second}\\
				t=\dfrac{2v_{20}}{a_1}=\dfrac{2\cdot\SI{10}{\meter/\second}}{\SI{0.5}{\meter/\second^{2}}}=\SI{40}{\second}
			\end{cases}
		\end{align*}
		trong đó nghiệm $t=0$ ứng với thời điểm hai xe gặp nhau lúc đầu, còn nghiệm $t=\SI{40}{\second}$ là nghiệm ta cần tìm. 
		
		Vị trí 2 xe gặp nhau 
		$$x_1=x_2=v_{20}t =\SI{10}{\meter/\second}\cdot\SI{40}{\second}= \SI{400}{m}.$$
		
		Vận tốc ô tô khi đó 
		$$v_1 = v_{10}+ a_1t = \SI{0}{\meter/\second}+\SI{0.5}{\meter/\second^{2}}\cdot\SI{40}{\second}=\SI{20}{m/s}.$$  
	}
	\viduii{3}{
		Trong một chuyến từ thiện của trung tâm A thì mọi người dừng lại bên đường uống nước. Sau đó, ngay thời điểm ô tô bắt đầu chuyển động nhanh dần đều với gia tốc $\SI{0,5}{m/s}^2$ thì có một xe khách vượt qua xe với tốc độ $\SI{18}{km/h}$ và gia tốc $\SI{0,3}{m/s}^2$. Hỏi ô tô đuổi kịp xe khách sau khi đi quãng đường bao xa, và tính vận tốc của ô tô lúc đó.
	}
	{	\begin{center}
			\textbf{Hướng dẫn giải}
		\end{center}
		
		Chọn chiều dương là chiều chuyển động của ô tô, gốc tọa độ tại vị trí uống nước, gốc thời gian là lúc xe ô tô khởi hành.
		
		Từ các thông số chuyển động của ô tô
		$$x_{10} = 0,\quad v_{10} = 0,\quad a_{1} = \SI{0,5}{m/s}^2,$$	
		ta suy ra phương trình chuyển động của ô tô
		$$x_1 = \text{0,25}t^2.$$
		
		Tương tự, phương trình chuyển động của xe khách cũng được suy ra từ các thông số chuyển động của xe khách
		\begin{equation*}
			\begin{gathered}
				x_{20} = 0,\quad v_{20} =\SI{18}{km/h}=\SI{5}{m/s},\quad a_{2} = \SI{0,3}{m/s}^2\\
				\Rightarrow\quad x_2 =5t+\text{0,15}t^2.
			\end{gathered}
		\end{equation*}
		
		Thời điểm hai xe gặp nhau được xác định từ phương trình 
		$$x_1=x_2 \quad\Rightarrow\quad \text{0,25}t^2=5t+\text{0,15}t^2 \quad\Rightarrow\quad t=\SI{0}{\second}\quad\vee\quad t =\SI{50}{s}.$$
		Ta chọn nghiệm $t =\SI{50}{s}$ là thời điểm gặp nhau sau khi ô tô đã xuất phát. 
		
		Vận tốc của ô tô khi đó
		$$v_1 = v_{10}+ a_1t = \SI{0}{\meter/\second}+\SI{0.5}{\meter/\second^{2}}\cdot\SI{50}{\second}=\SI{25}{m/s}.$$  
		
		Quãng đường ô tô đã đi được cho đến khi gặp nhau 
		$$s=x-x_0 =\text{0,25}t^2 = \SI{625}{\meter}.$$	
	}
\end{dang}
\begin{dang}{Xác định vận tốc, khoảng cách giữa hai vật\\ chuyển động thẳng biến đổi đều}
	\viduii{3}{Một xe ô tô bắt đầu chuyển động thẳng nhanh dần đều với gia tốc $\SI{0,5}{m/s}^2$ đúng lúc một xe máy chuyển động thẳng đều với vận tốc $\SI{36}{km/h}$ vượt qua nó. Xác định thời điểm để hai xe cách nhau một quãng đường là $\SI{100}{m}$.
	}
	{	\begin{center}
			\textbf{Hướng dẫn giải}
		\end{center}
		
		Chọn chiều dương là chiều chuyển động của ô tô, gốc tọa độ tại vị trí xuất phát, gốc thời gian là lúc xe ô tô khởi hành.
		
		Xe ô tô có các thông số chuyển động 
		$$x_{10} = \SI{0}{\meter};\qquad  v_{10} = \SI{0}{\meter/\second};\qquad a_1 = \SI{0.5}{\meter
			/\second^{2}}$$	
		nên có phương trình chuyển động 
		$$x_1 = x_{10}+v_{10}t+\dfrac{1}{2}a_{1}t^{2}=\dfrac{1}{2}a_1t^2=\SI{0.25}{}t^{2}.$$
		
		Xe máy có các thông số chuyển động 
		$$x_{20} = 0;\qquad v_{20} =\SI{36}{\kilo\meter/\hour}=\SI{10}{\meter/\second};\qquad a_{2} = \SI{0}{\meter/\second^{2}}$$	
		nên có phương trình chuyển động 
		$$x_2 =x_{20}+v_{20}t+\dfrac{1}{2}a_2t^2=v_{20}t=\SI{10}{}t.$$
		
		Để 2 xe cách nhau $\SI{40}{m}$ thì 
		$$|x_1-x_2| = 100.$$
		$$\Rightarrow \left[\begin{array}{ll}{x_1-x_2 =100}&\\{x_2-x_1 =100.}&\end{array}\right.$$
		
		$$\Rightarrow 
		\left[\begin{array}{ll}{\text{0,25}t^2 -10t=100 \Rightarrow t\approx\SI{48,28}{s}}&\\{10t-\text{0,25}t^2=100\Rightarrow t =\SI{20}{s}.}&\end{array}\right.$$
		
		\luuy{Đôi khi các phương trình cho ta nhiều nghiệm $t$, ta cần phân tích ý nghĩa của nghiệm và lựa chọn nghiệm phù hợp với thời điểm ta quan tâm. 
			
			Chẳng hạn trong  bài toán này, các phương trình cho ba nghiệm: $t_1\approx\SI{-8.28}{\second}, t_2=\SI{20}{\second},t_3\approx\SI{48.28}{\second}$. Nghiệm $t_1$ tương ứng với thời điểm trước khi xe hai xe gặp nhau lần đầu, lúc đó xe máy đang ở phía sau của ô tô và chuẩn bị vượt qua ô tô. Nghiệm $t_2$ ứng với thời điểm ô tô đang đuổi theo xe máy, và còn cách xe máy \SI{100}{\meter}. Nghiệm $t_3$ ứng với thời điểm ô tô đã vượt qua xe máy và đã bỏ xa xe máy \SI{100}{\meter}. Do đề bài chỉ cho ta biết về chuyển động hai xe kể từ thời điểm xe máy vượt qua ô tô, nên ta chỉ quan tâm các nghiệm $t>0$. 
		}
	}
\end{dang}
